\documentclass[12pt]{amsbook} 
\usepackage{amsmath, amssymb, graphics, setspace}

\newcommand{\mathsym}[1]{{}}
\newcommand{\unicode}[1]{{}}

\newcounter{mathematicapage}

\begin{document}
\section{How I tried to get $\tilde{\gamma}(r)$?}
I want that the Fourier transform of $\tilde{\gamma}(r)$ gives the proper scattering intensity. Often only $\gamma_0(r)=\tilde{\gamma}(r)/\tilde{\gamma}(0)$ is supplied in text books. For example for spherical particles one find
\begin{align}
\gamma_0(r) &=
\begin{cases}
  \left( 1-\frac{3}{4}\frac{r}{R}+\frac{1}{16}\left(\frac{r}{R}\right)^3\right) & \mbox{for } r\leq 2R \\
0 & \mbox{for }  r>2R
\end{cases}
\end{align}

The Fourier transform of $\gamma_0(r)$ yields
\begin{align}
\mathcal{F}[\gamma_0(r)] &= \int_0^\infty 4\pi r^2 \frac{\sin Qr}{Qr} \gamma_0(r) \mathrm{d}r 
\\ &= \frac43 \pi R^3 \left(3\frac{\sin QR -QR\cos QR}{(QR)^3}\right)^2
\end{align}
Here the factor $\Delta\eta^2\frac43 R^3$ is missing to get the correct scattering intensity of a single sphere.
It seems that  
$\tilde{\gamma}(r)=\Delta\eta^2 V_p\gamma_0(r)$ and 
\begin{align}
V_p=\int_0^\infty 4\pi r^2 \gamma_0(r) \mathrm{d}r
\end{align}
Therefore I used the following definition for $\tilde{\gamma}(r)$:
\begin{align}
\tilde{\gamma}(r)= \gamma_0(r) \Delta\eta^2 \int_0^\infty 4\pi r^2 \gamma_0(r) \mathrm{d}r
\end{align}



\section{Transformation cycle at the example of randomly distribute two phase system (DAB)}

Now lets have a look on the DAB example. It has the normalized correlation function $\gamma_{0,\mathrm{DAB}}(r)=\exp (-r/\xi)$. Following the above procedure I get
\begin{align}
\tilde{\gamma}_\mathrm{DAB}(r)= \gamma_{0,\mathrm{DAB}}(r) \Delta\eta^2 \int_0^\infty 4\pi r^2 \gamma_{0,\mathrm{DAB}}(r) \mathrm{d}r = 8\pi\xi^3\Delta\eta^2\exp(-r/\xi)
\end{align}
and from that the scattering intensity $I_\mathrm{DAB}(Q)$
\begin{align}
I_\mathrm{DAB}(Q) &= \mathcal{F}[\tilde{\gamma}_\mathrm{DAB}(r)] = \int_0^\infty 4\pi r^2 \frac{\sin Qr}{Qr} \tilde{\gamma}_\mathrm{DAB}(r) \mathrm{d}r \\
&= \frac{(8\pi\xi^3\Delta\eta)^2}{\left(1+Q^2\xi^2\right)^2}
\end{align}
From this function one also can calculate the Hankel transform $\mathcal{H}[I_\mathrm{DAB}(Q)]$ analytically. For this see the enclosed Mathematica file. So one can calculate $\tilde{G}(z)$ directly from the intensity which reads as
\begin{align}
\tilde{G}(z) &= 2\pi \mathcal{H}[I_\mathrm{DAB}(Q)] = 64\Delta\eta^2\pi^3\xi^3 z \mathrm{BesselK}_1(z/\xi)
\end{align}
However, one also can calculate $\tilde{G}(z)$ via the Abel transform of the correlation $\mathcal{A}[\tilde{\gamma}_\mathrm{DAB}(r)]$ which reads as
\begin{align}
\tilde{G}(z) &= (2\pi)^2 \mathcal{A}[\tilde{\gamma}_\mathrm{DAB}(r)] = 64\Delta\eta^2\pi^3\xi^3 z \mathrm{BesselK}_1(z/\xi)
\end{align}
I had to multiply the Abel transform by $(2\pi)^2$ to get consistency in the results. It seems, we now have closed the transformation cycle for a specific example. It would be interesting to proof it for any correlation function.


\section{Mathematica output}
\begin{doublespace}
\noindent\(\pmb{\text{gamma0DAB}[\text{r$\_$},\text{xi$\_$}]\text{:=}\text{Exp}[-r/\text{xi}]}\)
\end{doublespace}

\begin{doublespace}
\noindent\(\pmb{\text{gamma0DAB}[r,\text{xi}]}\)
\end{doublespace}

\begin{doublespace}
\noindent\(e^{-\frac{r}{\text{xi}}}\)
\end{doublespace}

\begin{doublespace}
\noindent\(\pmb{\text{Integrate}[\text{gamma0DAB}[r,\text{xi}]*4*\text{Pi}*r{}^{\wedge}2,\{r,0,\text{Infinity}\},}\\
\pmb{\text{Assumptions}\to \text{xi}>0]}\)
\end{doublespace}

\begin{doublespace}
\noindent\(8 \pi  \text{xi}^3\)
\end{doublespace}

\begin{doublespace}
\noindent\(\pmb{\text{gammaDAB}[\text{r$\_$},\text{xi$\_$},\text{eta$\_$}]\text{:=}}\\
\pmb{\text{Integrate}[\text{gamma0DAB}[r,\text{xi}]*4*\text{Pi}*r{}^{\wedge}2,\{r,0,\text{Infinity}\},}\\
\pmb{\text{Assumptions}\to \text{xi}>0]*\text{gamma0DAB}[r,\text{xi}]*\text{eta}{}^{\wedge}2}\)
\end{doublespace}

\begin{doublespace}
\noindent\(\pmb{\text{IDAB}=}\\
\pmb{\text{FullSimplify}[}\\
\pmb{\text{Integrate}[\text{gammaDAB}[r,\text{xi},\text{eta}]*\text{Sin}[Q*r]/(Q*r)*4*\text{Pi}*r{}^{\wedge}2,}\\
\pmb{\{r,0,\text{Infinity}\}],\text{Assumptions}\to \text{xi}>0\&\&Q>0]}\)
\end{doublespace}

\begin{doublespace}
\noindent\(\frac{64 \text{eta}^2 \pi ^2 \text{xi}^6}{\left(1+Q^2 \text{xi}^2\right)^2}\)
\end{doublespace}

\begin{doublespace}
\noindent\(\pmb{\text{GzHF}=\text{Integrate}[2*\text{Pi}*\text{IDAB}*\text{BesselJ}[0,Q*z]*Q,\{Q,0,\text{Infinity}\},}\\
\pmb{\text{Assumptions}\to z>0\&\&\text{xi}>0]}\)
\end{doublespace}

\begin{doublespace}
\noindent\(64 \text{eta}^2 \pi ^3 \text{xi}^3 z \text{BesselK}\left[1,\frac{z}{\text{xi}}\right]\)
\end{doublespace}

\begin{doublespace}
\noindent\(\pmb{\text{GzA}=\text{Integrate}[2*(2*\text{Pi}){}^{\wedge}2*\text{gammaDAB}[r,\text{xi},\text{eta}]r/\text{Sqrt}[r{}^{\wedge}2-z{}^{\wedge}2],}\\
\pmb{\{r,z,\text{Infinity}\}, \text{Assumptions}\to z>0\&\&\text{xi}>0]}\)
\end{doublespace}

\begin{doublespace}
\noindent\(64 \text{eta}^2 \pi ^3 \text{xi}^3 z \text{BesselK}\left[1,\frac{z}{\text{xi}}\right]\)
\end{doublespace}


\end{document} 