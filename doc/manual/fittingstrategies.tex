\clearpage
\section{Fitting strategies}

\clearpage
\section{Criteria for goodness-of-fit}
All criteria shown below for testing the goodness of a fit should be
considered with caution \cite{Bevington2003,Press2007}.
When you get data on a SAS instrument the
the measured intensities are measured with some statistical uncertainties.
Normally one assumes Poisson statistics to determine the uncertainty
in the counting statistics. The data reduction software should than perform a
proper error propagation analysis for all succeeding data treatment operations.
However, by this procedure only statistical uncertainties are taken into account.
All systematic uncertainties are than hopefully covered during the data treatment,
as fir example background correction, transmission correction etc \ref{ch:SASdatacorrection}.
~\\
\subsection{chi square test}
~\\
The method of least squares is built on the hypothesis that the
optimum description of a set of data is one which minimizes the
weighted sum of squares of deviations, between the data,
$I_\text{exp}(q_i)$ , and the fitting function $I_\text{th}(q_i)$:
\begin{equation}
\chi^2 = \displaystyle\sum_{i=1}^N
\left(\frac{I_\text{exp}(q_i)-I_\text{th}(q_i)}{\Delta
I(q_i)}\right)^2
\end{equation}
As a rule of thumb for chi-square fitting is the statement that a
�good fit� is achieved when the reduced chi-square equals one. The
reduced chi-square value, which equals the residual sum of square
divided by the degree of freedom can be computed by
\begin{equation}
\chi^2_\nu = \displaystyle \frac{1}{N-m} \sum_{i=1}^N
\left(\frac{I_\text{exp}(q_i)-I_\text{th}(q_i)}{\Delta
I(q_i)}\right)^2 = \frac{\chi^2}{N-m}
\end{equation}
where $N$ is the number of data points and $m$ the number of fit parameters.
$\nu=N-m$ is called the "number of degree of freedom". The reduced chi-square
is closely related to the variance of the fit $s^2$ by
\begin{equation}
s^2=\chi_\nu^2 \left( \frac{1}{N}\sum_{i=1}^N \frac{1}{\left(\Delta I_i^\text{exp}\right)^2}\right)^{-1}
\end{equation}

In the theory of hypothesis testing $\chi^2$ can be used to test
for goodness of a fit. The probability that a random set of $N$ data points
would yield a value of $\chi^2$ equal or greater than the measured
one is given by
\begin{equation}
\displaystyle Q_\text{factor} =
Q\left(\frac{N-m}{2},\frac{\chi^2}{2}\right) =
\frac{\Gamma\left(\frac{N-m}{2},\frac{\chi^2}{2}\right)}{\Gamma\left(\frac{N-m}{2}\right)}
\text{ with } \Gamma\left( a,x\right) =  \int_x^\infty t^{s-1}e^{-t}
dt
\end{equation}
For a fitting function being a good approximation to the data the experimental
value of $\chi^2_\nu$ should be close to one and the probability
$Q_\text{factor}$ somewhere between 0.01 and 0.5. For probability values
close to one the fit seems to be too good to be true.

~\\
\subsection{R-factor}
~\\
The crystallographers have introduced another parameter for the goodness of a fit.
They use the $R$ factor \cite{Rfactor,Hamilton1965} as a measure of model quality which is defines as
\begin{equation}\label{eqRfactor}
    \displaystyle  = \frac{\displaystyle\sum_{i=1}^N
\abs{I_\text{exp}(q_i) -
I_\text{th}(q_i)}}{\displaystyle\sum_{i=1}^N
\abs{I_\text{exp}(q_i)}}
\end{equation}
Theoretical values of $R$ range from zero (perfect
agreement of calculated and observed intensities) to infinity.  $R$
factors greater than 0.5 indicate in crystallography very poor
agreement between observed and calculated intensities. Models
refining to $R < 0.05$ are often considered to be good. However, the
$R$ factor must always be treated with caution, only as an indicator of
precision and not accuracy. In Crystallography partially incorrect
structures have been reported with $R$ values below $0.1$; many
imprecise but essentially correct structures have been reported with
higher $R$ values.

In practice, weighted $R$ factors $R_\text{w}$ are more often used
to track least-squares refinement, since the functions minimized are
weighted according to estimates of the precision of the measured
quantity. The weighted residuals are defined as:
\begin{equation} \displaystyle R_{w} =
\sqrt{\frac{\displaystyle\sum_{i=1}^N \left(\frac{I_\text{exp}(q_i)
- I_\text{th}(q_i)}{\Delta
I(q_i)}\right)^2}{\displaystyle\sum_{i=1}^N
\frac{I_\text{exp}^2(q_i)}{\Delta I(q_i)}}}
\end{equation}
