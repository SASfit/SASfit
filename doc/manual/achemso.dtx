% \iffalse meta-comment
%
% Copyright (C) 1996-98 by
%    Mats Dahlgren
% Copyright (C) 2007 by
%    Joseph Wright <joseph.wright@morningstar2.co.uk>
%
% This work may be distributed and/or modified under the
% conditions of the LaTeX Project Public License, either
% version 1.3 of this license or (at your option) any later
% version. The latest version of this license is in
%    http://www.latex-project.org/lppl.txt
% and version 1.3 or later is part of all distributions of
% LaTeX version 2003/12/01 or later.
%
% This work has the LPPL maintenance status ``maintained.''
%
% The current maintainer of this work is Joseph Wright.
%
% This work consists of the source file achemso.dtx
%                 and the derived files achemso.ins,
%                                       achemso.sty,
%                                       achemso.bib,
%                                       achemso.pdf,
%                                       achemso.bst,
%                                       achemsol.bst,
%                                       achemsnat.bst and
%                                       achemlnt.bst,
% Unpacking:
%    (a) If achemso.ins is present:
%           tex achemso.ins
%    (b) Without achemso.ins:
%           tex achemso.dtx
%    (c) If you use LaTeX to generate files:
%           latex \let\install=y% \iffalse meta-comment
%
% Copyright (C) 1996-98 by
%    Mats Dahlgren
% Copyright (C) 2007 by
%    Joseph Wright <joseph.wright@morningstar2.co.uk>
%
% This work may be distributed and/or modified under the
% conditions of the LaTeX Project Public License, either
% version 1.3 of this license or (at your option) any later
% version. The latest version of this license is in
%    http://www.latex-project.org/lppl.txt
% and version 1.3 or later is part of all distributions of
% LaTeX version 2003/12/01 or later.
%
% This work has the LPPL maintenance status ``maintained.''
%
% The current maintainer of this work is Joseph Wright.
%
% This work consists of the source file achemso.dtx
%                 and the derived files achemso.ins,
%                                       achemso.sty,
%                                       achemso.bib,
%                                       achemso.pdf,
%                                       achemso.bst,
%                                       achemsol.bst,
%                                       achemsnat.bst and
%                                       achemlnt.bst,
% Unpacking:
%    (a) If achemso.ins is present:
%           tex achemso.ins
%    (b) Without achemso.ins:
%           tex achemso.dtx
%    (c) If you use LaTeX to generate files:
%           latex \let\install=y% \iffalse meta-comment
%
% Copyright (C) 1996-98 by
%    Mats Dahlgren
% Copyright (C) 2007 by
%    Joseph Wright <joseph.wright@morningstar2.co.uk>
%
% This work may be distributed and/or modified under the
% conditions of the LaTeX Project Public License, either
% version 1.3 of this license or (at your option) any later
% version. The latest version of this license is in
%    http://www.latex-project.org/lppl.txt
% and version 1.3 or later is part of all distributions of
% LaTeX version 2003/12/01 or later.
%
% This work has the LPPL maintenance status ``maintained.''
%
% The current maintainer of this work is Joseph Wright.
%
% This work consists of the source file achemso.dtx
%                 and the derived files achemso.ins,
%                                       achemso.sty,
%                                       achemso.bib,
%                                       achemso.pdf,
%                                       achemso.bst,
%                                       achemsol.bst,
%                                       achemsnat.bst and
%                                       achemlnt.bst,
% Unpacking:
%    (a) If achemso.ins is present:
%           tex achemso.ins
%    (b) Without achemso.ins:
%           tex achemso.dtx
%    (c) If you use LaTeX to generate files:
%           latex \let\install=y% \iffalse meta-comment
%
% Copyright (C) 1996-98 by
%    Mats Dahlgren
% Copyright (C) 2007 by
%    Joseph Wright <joseph.wright@morningstar2.co.uk>
%
% This work may be distributed and/or modified under the
% conditions of the LaTeX Project Public License, either
% version 1.3 of this license or (at your option) any later
% version. The latest version of this license is in
%    http://www.latex-project.org/lppl.txt
% and version 1.3 or later is part of all distributions of
% LaTeX version 2003/12/01 or later.
%
% This work has the LPPL maintenance status ``maintained.''
%
% The current maintainer of this work is Joseph Wright.
%
% This work consists of the source file achemso.dtx
%                 and the derived files achemso.ins,
%                                       achemso.sty,
%                                       achemso.bib,
%                                       achemso.pdf,
%                                       achemso.bst,
%                                       achemsol.bst,
%                                       achemsnat.bst and
%                                       achemlnt.bst,
% Unpacking:
%    (a) If achemso.ins is present:
%           tex achemso.ins
%    (b) Without achemso.ins:
%           tex achemso.dtx
%    (c) If you use LaTeX to generate files:
%           latex \let\install=y\input{achemso.dtx}
%
% Documentation:
%    (a) Without write18 enabled:
%          pdflatex achemso.dtx
%          bibtex8 --wolfgang achemso.aux
%          makeindex -s gind.ist achemso.idx
%          makeindex -s gglo.ist -o achemso.gls  achemso.glo
%          pdflatex achemso.dtx
%          makeindex -s gind.ist achemso.idx
%          makeindex -s gglo.ist -o achemso.gls  achemso.glo
%          pdflatex achemso.dtx
%    (b) With write18 enabled:
%          pdflatex achemso.dtx
%          bibtex8 --wolfgang achemso.aux
%          pdflatex achemso.dtx
%          pdflatex achemso.dtx
%
% Installation:
%     Copy achemso.sty and the achmes*.bst files to a location
%     searched by TeX, and if required by your TeX installation,
%     run the appropriate command to build a hash of files
%     (texhash, mpm --update-db, etc.)
%
% Note:
%     The jawltxdoc.sty file is not needed for installation,
%     only for building the documentation.  It may be deleted.
%
%<*ignore>
% This is all taken verbatim from Heiko Oberdiek's packages
\begingroup
  \def\x{LaTeX2e}%
\expandafter\endgroup
\ifcase 0\ifx\install y1\fi\expandafter
         \ifx\csname processbatchFile\endcsname\relax\else1\fi
         \ifx\fmtname\x\else 1\fi\relax
\else\csname fi\endcsname
%</ignore>
%<*install>
\input docstrip.tex
\keepsilent
\askforoverwritefalse
\preamble
 ----------------------------------------------------------------
 The achemso package - LaTeX and BibTeX support for American
 Chemical Society publications
 Maintained by Joseph Wright
 E-mail: joseph.wright@morningstar2.co.uk
 Released under the LaTeX Project Public License v1.3 or later
 See http://www.latex-project.org/lppl.txt
 ----------------------------------------------------------------

\endpreamble
\Msg{Generating achemso files:}
\usedir{tex/latex/contib/achemso}
\generate{\file{\jobname.ins}{\from{\jobname.dtx}{install}}
          \file{\jobname.sty}{\from{\jobname.dtx}{package}}
          \file{jawltxdoc.sty}{\from{\jobname.dtx}{jawltxdoc}}
}
\declarepostamble\bibtexable
\endpostamble
\usedir{bibtex/bst/achemso}
\generate{\usepostamble\bibtexable
          \file{achemso.bst}{\from{achemso.dtx}{bib}}
          \file{achemnat.bst}{\from{achemso.dtx}{bib,nat}}
          \file{achemsol.bst}{\from{achemso.dtx}{bib,list}}
          \file{achemlnt.bst}{\from{achemso.dtx}{bib,list,nat}}
}
\generate{\usepostamble\empty\usepreamble\empty
          \file{achemso.bib}{\from{achemso.dtx}{database}}
}
\endbatchfile
%</install>
%<*ignore>
\fi
% Will Robertson's trick
\immediate\write18{makeindex -s gind.ist -o \jobname.ind  \jobname.idx}
\immediate\write18{makeindex -s gglo.ist -o \jobname.gls  \jobname.glo}
%</ignore>
%<*driver>
\PassOptionsToClass{a4paper}{article}
\documentclass{ltxdoc}
\EnableCrossrefs
\CodelineIndex
\RecordChanges
%\OnlyDescription
% The various formatting commands used in this file are collected
% together in |jawltxdoc|.
\usepackage{jawltxdoc}
\begin{document}
  \DocInput{\jobname.dtx}
\end{document}
%</driver>
% \fi
%
% \CheckSum{105}
%
% \CharacterTable
%  {Upper-case    \A\B\C\D\E\F\G\H\I\J\K\L\M\N\O\P\Q\R\S\T\U\V\W\X\Y\Z
%   Lower-case    \a\b\c\d\e\f\g\h\i\j\k\l\m\n\o\p\q\r\s\t\u\v\w\x\y\z
%   Digits        \0\1\2\3\4\5\6\7\8\9
%   Exclamation   \!     Double quote  \"     Hash (number) \#
%   Dollar        \$     Percent       \%     Ampersand     \&
%   Acute accent  \'     Left paren    \(     Right paren   \)
%   Asterisk      \*     Plus          \+     Comma         \,
%   Minus         \-     Point         \.     Solidus       \/
%   Colon         \:     Semicolon     \;     Less than     \<
%   Equals        \=     Greater than  \>     Question mark \?
%   Commercial at \@     Left bracket  \[     Backslash     \\
%   Right bracket \]     Circumflex    \^     Underscore    \_
%   Grave accent  \`     Left brace    \{     Vertical bar  \|
%   Right brace   \}     Tilde         \~}
%
%\GetFileInfo{\jobname.sty}
%
%\changes{v1.0}{1998/06/01}{Initial release of package by Mats
%   Dahlgren}
%\changes{v2.0}{2007/01/17}{Re-write of package by Joseph Wright}
%\changes{v2.0}{2007/01/17}{Several improvements to BibTeX style
%  files}
%\changes{v2.0}{2007/01/17}{License changed to LPPL}
%\changes{v2.1}{2007/02/15}{Updated documentation to reflect 3rd
%  edition of ACS Style Guide}
%\changes{v2.1}{2007/02/15}{BibTeX style improved to reflect 3rd
%  edition of ACS Style Guide}
%\changes{v2.2}{2007/06/05}{Added \texttt{natbib} support}
%\changes{v2.2a}{2007/07/08}{Fixed separation of editor names}
%\changes{v2.2a}{2007/07/08}{Bug fixes to \texttt{natbib} and list
% support}
%\changes{v2.2a}{2007/07/08}{\texttt{title} field included in output
% for \texttt{incollection} records}
%\changes{v2.2b}{2007/07/09}{Bug fix to name formatting}
%\changes{v2.2d}{2007/10/16}{Added \textsc{url} field to
%  \texttt{misc} output}
%\changes{v2.2d}{2007/10/16}{Package design improved}
%
%\DoNotIndex{\@biblabel,\@eha,\@gobble,\@ifpackageloaded,\@ifundefined}
%\DoNotIndex{\bibliographystyle,\bibname,\citeform,\citeleft}
%\DoNotIndex{\citenumfont,\citeright,\DeclareOption,\def,\else,\emph}
%\DoNotIndex{\fi,\ifx,\NeedsTeXFormat,\newcommand,\newif}
%\DoNotIndex{\OptionNotUsed,\PackageError,\PackageWarning}
%\DoNotIndex{\ProcessOptions,\ProvidesPackage,\refname,\relax}
%\DoNotIndex{\renewcommand,\RequirePackage,\textit,}
%
% \title{\texttt{achemso} --- LaTeX and BibTeX support for American
%   Chemical Society publications%
%   \thanks{This file describes version \fileversion, last revised
%           \filedate.}}
% \author{Joseph Wright%
%   \thanks{E-mail: joseph.wright@morningstar2.co.uk}}
% \date{Released \filedate}
%
%\maketitle
%
%\begin{abstract}
% The |achemso| package provides a BibTeX style in accordance with
% the requirements of the journals of the American Chemical Society,
% along with a supporting LaTeX package file. Also provided is a
% BibTeX style file to be used for bibliography database listings.
%\end{abstract}
%
% \section{Introduction}
%
% Synthetic chemists do not, in the main, use LaTeX for the
% preparation of journal articles. Some journals, mainly in the
% physical chemistry area, do accept LaTeX submissions.  Given the
% clear advantages of LaTeX over other methods, it would be
% nice to be able to use LaTeX for preparing reports. Thus the need
% for BibTeX styles for chemistry is real. The package |achemso|
% provides for a BibTeX style and other support for articles and
% reports in the style of the American Chemical Society (ACS).
%
% As describe in \emph{The ACS Style Guide} \cite{Coghill2006},
% almost all ACS publications use the same style for the formatting
% of references.  The reproduction of this style is the aim of the
% BibTeX style file provided here.  However, the ACS use different
% citation styles in different publications.  The |achemso| package
% provides support for the two numerical systems: superscript
% and italic in-text citations.  The majority of ACS journals use
% the superscript method (Table \ref{tbl:journals-super}), with a
% smaller number using the italic system (Table
% \ref{tbl:journals-inline}). The journal \emph{Biochemistry} does
% not use the standard ACS style for references, and so is not
% covered by the |achemso| package.
% \begin{table}
%   \centering
%   \small
%   \begin{tabular}{>{\itshape}l>{\itshape}l}
%     \toprule
%     \upshape{Journal Title} & \upshape{\emph{CASSI} Abbreviation} \\
%     \midrule
%     Accounts of Chemical Research & Acc.~Chem.~Res. \\
%     Analytical Chemistry & Anal.~Chem. \\
%     Biomacromolecules & Biomacromolecules \\
%     Chemical Reviews & Chem.~Rev. \\
%     Chemistry of Materials & Chem.~Mater. \\
%     Crystal Growth \& Design & Cryst.~Growth Des. \\
%     Energy \& Fuels & Energy Fuels \\
%     Industrial \& Engineering Chemistry Research & Ind.~Eng.~Chem.~Res. \\
%     Inorganic Chemistry & Inorg.Chem. \\
%     Journal of the American Chemical Society & J.~Am.~Chem.~Soc. \\
%     Journal of Chemical and Engineering Data & J.~Chem.~Eng.~Data \\
%     Journal of Chemical Theory and Computation & J.~Chem.~Theory Comput. \\
%     Journal of Chemical Information and Modeling & J.~Chem.~Inf.~Model. \\
%     Journal of Combinatorial Chemistry & J.~Comb.~Chem. \\
%     Journal of Medicinal Chemistry & J.~Med.~Chem. \\
%     Journal of Natural Products & J.~Nat.~Prod. \\
%     The Journal of Organic Chemistry & J.~Org.~Chem. \\
%     The Journal of Physical Chemistry A & J.~Phys.~Chem.~A \\
%     The Journal of Physical Chemistry B & J.~Phys.~Chem.~B \\
%     The Journal of Physical Chemistry C & J.~Phys.~Chem.~C \\
%     Journal of Proteome Research & J.~Proteome Res. \\
%     Langmuir & Langmuir \\
%     Macromolecules & Macromolecules \\
%     Molecular Pharmaceutics & Mol.~Pharm. \\
%     Nano Letters & Nano Lett. \\
%     Organic Letters & Org.~Lett. \\
%     Organic Process Research \& Design & Org.~Process Res.~Dev. \\
%     Organometallics & Organometallics \\
%     \bottomrule
%   \end{tabular}
%   \caption{Journals using the ACS reference style with superscript citations}
%   \label{tbl:journals-super}
% \end{table}
% \begin{table}
%   \small
%   \centering
%   \begin{tabular}{>{\itshape}l>{\itshape}l}
%     \toprule
%     \upshape{Journal Title} & \upshape{\emph{CASSI} Abbreviation} \\
%     \midrule
%     ACS Chemical Biology & ACS Chem.~Biol. \\
%     Bioconjugate Chemistry & Bioconjugate Chem. \\
%     Biotechnology Progress & Biotechnol.~Prog. \\
%     Chemical Research in Toxicology & Chem.~Res.~Toxicol. \\
%     Environmental Science and Technology & Envirn.~Sci.~Technol. \\
%     Journal of Agricultural and Food Chemistry & J.~Agric.~Food Chem. \\
%     \bottomrule
%   \end{tabular}
%   \caption{Journals using the ACS reference style with in-text citations}
%   \label{tbl:journals-inline}
% \end{table}
%
% This package consists of two BibTeX files (|achemso.bst|
% and |achemsol.bst|) along with a small LaTeX file |achemso.sty|.
% The naming of the package is slightly unusual, but follows from
% the need to pick a unique name.  To quote the documentation to the
% first version:
% \begin{quote}
%   there is already a LaTeX 2.09 and
%   BibTeX style package called |acsarticle| and
%   |acs.bst|, which are not ``ACS'' as in `American Chemical
%   Society' (rather, this package is
%   formatting the output according to the instructions of
%   \emph{Advances in Control Systems}).  Hence, \emph{this}
%   new package had to be given another name.  The name of choice
%   was then |achemso|, which is made from the words
%   ``\emph{A}merican \emph{Chem}ical \emph{So}ciety.''
% \end{quote}
%
% \subsection{Change of maintainer}
%
% This package was initially released by Mats Dahlgren.  He no
% longer has time to devote to LaTeX development.  With his permission,
% the package has therefore been taken over by Joseph
% Wright, the maintainer of the the |rsc| package.  The majority of
% the package has been rebuilt and the BibTeX style file has been
% totally overhauled.  Any mistakes are entirely the fault of the
% new maintainer!
%
% \section{The BibTeX style files}
%
% The BibTeX style files implement the bibliographic style specified
% by the ACS in \emph{The ACS Style Guide} \cite{Coghill2006},
% on the ACS website \cite{ACS2007}, and in current ACS publications.
% Some of this information can be contradictory, and \emph{The ACS
% Style Guide} sometimes gives more than one option as a model.
% In order to resolve cases where several possibilities are available
% current editions of the \emph{Journal of the American Chemical
% Society} have been consulted; the current consensus there has been
% taken as the correct approach.  In addition to the problem
% of picking the correct style, some of the BibTeX record types are
% difficult to match to standard references in ACS journals.  The
% ``best guess'' has been taken with these.
%
% \subsection{Additional record types}
%
% In general, the database record types supported here follow those
% in the standard BibTeX style files.  Four additional record types
% are provided:
% \begin{description}
%   \item[|patent|] A patent: formatting is similar to other record
%       types.  The data entry for this record type follows the
%       pattern used in |rsc.bst|: |journal| is used to hold
%       the patent type (\emph{e.g.}~``U.S.~Patent''), with the
%       patent number given in |pages|. Whilst this format is
%       non-standard, it is relatively easy to use and implement!
%   \item[|submitted|] Articles submitted to journals but not
%       yet accepted: appends ``submitted'' in a suitable fashion
%       to the entry.
%   \item[|inpress|] Articles in press: appends ``in press'' or,
%       if available, the DOI number assigned to the article.
%   \item[|remark|] A note with no other information to be
%       included.  Output consists purely of the |note| field.
% \end{description}
%
% \subsection{BibTeX database entry requirements}
%
% The requirements for entries in the BibTeX database are slightly
% different using |achemso.bst| to the standard style files. This
% is mainly because some fields are not cited in
% ACS bibliographies.  In particular, journal articles do not
% require a title (the |title| field is ignored).  Articles
% in books and ``collections'' only need the title of the book.
% If a chapter title is given for an |incollection| record, it will
% be printed, but not in the case of an |inbook| record.
%
% \subsection{References to software}
%
% Referencing software is always a little difficult.  The style files
% provided here follow the normal LaTeX convention of using the
% |manual| record type to cite software.  The only requirement is a
% |title|, but fields such as |organization| may be used for more
% detail.  The |edition| field is used to format the software version
% correctly: this will automatically be prefixed with ``version'' by
% the style file.
%
% \subsection{The \texttt{annotate} field}
%
% The standard BibTeX styles use the |note| field for notes to be
% added to the citation.  However, it is common to want personal
% notes about references.  This is catered for using the |annotate|
% field.  The style |achemso| ignores the |annotate| field, whilst
% the |achemsol| style appends the |annotate| information to the
% bibliographic output.  Thus |achemsol| is intended for use in
% database maintenance, whilst |achemso| is for production
% bibliographies.
%
% \DescribeMacro{\refin}
% For use in the |annotate| field the macro \cmd{\refin}
% is defined in |achemso.bst| and |achemsol.bst|.
% The command takes a single argument \marg{text}, and
% gives the output \textbf{Referenced in: text}.
% This command takes one argument (normally text) which is
% preceded by the text ``\textbf{Referenced in:} \meta{text}''.
% The \cmd{\refin} command is intended for tracking citations
% ``backward'' through the database.  For example, this could be
% used to link citations in a database to the writer's own papers.
%
% \subsection{Predefined journal abbreviations}
%
% A number of journal abbreviations are defined in the |.bst| files.
% The abbreviations cover a number ACS journals, several other
% physical chemistry publications and other journals listed as
% highly cited by \emph{Chem.\ Abs.}\ The interested user should
% consult the |.bst| files for full details.
%
% \subsection{\texttt{natbib} support}
%
% As of version 2.2, a |natbib| compatible style file, |achemnat| is
% provided.  The style file provides the appropriate option,
% |natbib|, to load this BibTeX file along with |natbib|, setting up
% the appropriate options.
%
% \section{The LaTeX Package}
%
% The current version of |achemso.sty| is a complete
% re-implementation of the functionality of the original file,
% designed to ensure greater compatibility with other packages. The
% only change for the user is that the bibliography section does
% \emph{not} start a new page when using the |article| document
% class. However, the package now supports all of the standard
% classes, and so the |report| class may be used to ensure a new
% page is started.
%
% \DescribeMacro{\bibliographystyle}
% Loading the |achemso| package adds the appropriate
% \cmd{\bibliographystyle} command to the LaTeX source.  As a result,
% subsequent \cmd{\bibliographystyle} statements will be ignored:
% a suitable warning is given.  The format of citations is altered
% (using the |cite| or |natbib| package as appropriate), and the
% package ensures that the bibliography will be named ``References''
%  in all standard document types.\footnote{This only works if the
% \texttt{babel} package is \emph{not} loaded.  Users wanting a
% system which works with \texttt{babel} should look at the
% \texttt{chemstyle package}. }
%
%  The |achemso| package has five options:
% \begin{description}
%   \item[|note|] If the bibliography contains notes as well
%       as citations, then the section heading should be
%       ``References and Notes''.  This is altered by the
%       |note| package option.
%   \item[|number|] This option numbers the bibliography
%       section (using the |tocbibind| package), and causes it to
%       be entered in the Table of Contents.
%   \item[|list|] This option is intended for creating a listing
%       of the entire BibTeX database.  The BibTeX style is
%       changed to |achemsol|, which will output the additional
%       database field |annotate|, intended for personal notes
%       about a particular database entry.  It also adds the
%       BibTeX key for each citation as a marginal note to the
%       output, using the |showkeys| package.
%    \item[|notsuper|] Switches from superscript citations
%       (\emph{e.g.}~Author \emph{et al.}$^3$) to
%        in-text ones in italics (\emph{e.g.}~Author
%        \emph{et al.}~(\emph{3})). There is a |super|
%        option for completeness, which simply gives the default
%        behaviour.
%     \item[|natbib|] Uses |natbib| rather than |cite| for citation
%        formatting; this also loads the |achemnat| style in place
%        of |achemso|.
% \end{description}
%
% \StopEventually{\bibliography{achemso}}
%
% \section{The Package Code}
%
%  The package code is not very complicated.  For the
%  interested reader(s), it is presented here.
%
% The usual setup code is executed.
%    \begin{macrocode}
%<*package>
\NeedsTeXFormat{LaTeX2e}
\ProvidesPackage{achemso}
  [2007/10/16 v2.2d LaTeX and BibTeX support for American
     Chemical Society publications]
%    \end{macrocode}
% \begin{macro}{\ACS@sctnnmbr}
% \begin{macro}{\ACS@lst}
% \begin{macro}{\ACS@note}
% \changes{v2.0}{2007/01/17}{Boolean values made internal to package}
% \begin{macro}{\ACS@super}
% \changes{v2.1}{2007/02/15}{New Boolean for citation control}
% \begin{macro}{\ACS@natbib}
% \changes{v2.2a}{2007/07/08}{New Boolean for |natbib| support}
% Boolean values are used to handle the options.
%    \begin{macrocode}
\newif \ifACS@sctnnmbr \ACS@sctnnmbrfalse
\newif \ifACS@list     \ACS@listfalse
\newif \ifACS@note     \ACS@notefalse
\newif \ifACS@super    \ACS@supertrue
\newif \ifACS@natbib   \ACS@natbibfalse
%    \end{macrocode}
% \end{macro}
% \end{macro}
% \end{macro}
% \end{macro}
% \end{macro}
% The options are processed.
%\changes{v2.2d}{2007/10/16}{Added \texttt{notes} option}
%    \begin{macrocode}
\DeclareOption{note}{\ExecuteOptions{notes}}
\DeclareOption{notes}{\ACS@notetrue}
\DeclareOption{number}{\ACS@sctnnmbrtrue}
\DeclareOption{super}{\ACS@supertrue}
\DeclareOption{list}{\ACS@listtrue}
\DeclareOption{notsuper}{\ACS@superfalse}
\DeclareOption{natbib}{\ACS@natbibtrue}
\DeclareOption*{\OptionNotUsed}
\ProcessOptions
%    \end{macrocode}
% \changes{v2.1}{2007/02/15}{|cite| package is loaded with different
% options depending on citation style requested}
% \changes{v2.2a}{2007/07/08}{|natbib| support added}
% The |cite| package is loaded to sort and compress references
% correctly. Depending upon the package option given, citations are
% either superscript or italic and in parentheses.
%    \begin{macrocode}
\ifACS@natbib
  \ifACS@super
    \RequirePackage[numbers,sort&compress,super]{natbib}
  \else
%    \end{macrocode}
% For in-line citations with |natbib|, we have to do a
% bit of work to get things to look right.  |natbib| uses
% \cmd{\citenumfont} to format the numbers, but it is not defined
% by default, so we have to use \cmd{\newcommand}.
%    \begin{macrocode}
    \RequirePackage[numbers,sort&compress,round]{natbib}
    \newcommand*{\citenumfont}{\textit}
  \fi
\else
  \ifACS@super
%    \end{macrocode}
%\changes{v2.2c}{2007/08/22}{Use the \texttt{overcite} alias for
%  \texttt{cite} as ACS have very old LaTeX system}
%    \begin{macrocode}
    \RequirePackage[nospace]{overcite}
  \else
%    \end{macrocode}
% Again in-line citations need some format changes.  In the case of
% |cite|, everything is defined initially.  Thus we can use
% \cmd{\renewcommand} for everything.
%    \begin{macrocode}
    \RequirePackage{cite}
    \renewcommand\citeleft{(}
    \renewcommand\citeright{)}
    \renewcommand\citeform[1]{\emph{#1}}
  \fi
\fi
%    \end{macrocode}
% If the |babel| package is loaded, the |note| option does not
% work.  So it is disabled here with a suitable warning.
%    \begin{macrocode}
\@ifpackageloaded{babel}
  {\ACS@notefalse\PackageWarning{achemso}%
    {babel package loaded - note option disabled}}
  {\relax}
%    \end{macrocode}
% \begin{macro}{\ACS@biberror}
% The function \cmd{\ACS@biberror} is defined here to provide an
% easy way of generating a warning if there is no name for a
% bibliography section.  This will only happen with non-standard
% class files.
%     \begin{macrocode}
\def\ACS@biberror{\PackageError{achemso}%
  {No bibliography name command defined}\@eha}
%    \end{macrocode}
% \end{macro}
% \begin{macro}{\refname}
% \begin{macro}{\bibname}
% The |note| option renames the references section to
% ``References and Notes''.  This applies for all standard
% document classes.
% The term ``Bibliography'' is not used in chemistry, the value of
% \cmd{\bibname} is redefined here in all cases where it exists.
%     \begin{macrocode}
\@ifundefined{refname}{%
  \@ifundefined{bibname}{%
    \ACS@biberror
  }{%
    \ifACS@note
      \renewcommand*{\bibname}{References and Notes}
    \else
      \renewcommand*{\bibname}{References}
    \fi
  }
}{%
  \ifACS@note
    \renewcommand*{\refname}{References and Notes}
  \fi
}
%    \end{macrocode}
% \end{macro}
% \end{macro}
% If the |number| option is set, the |tocbibind| package is
% used to number the bibliography.
% \changes{v2.0}{2007/01/17}{Switched to using \texttt{tocbibind}
% to produce number bibliography}
%    \begin{macrocode}
\ifACS@sctnnmbr
  \RequirePackage[numbib]{tocbibind}
\fi
%    \end{macrocode}
% \begin{macro}{\bibliographystyle}
% Depending on the package option, the bibliography style
% will either be |achemso| or |achemsol|.  The later is intended
% for listing the entire database.  The |list| option of the
% package selects this, and for listing also generates boxed
% labels for each reference.  The |showkeys| package provides
% this functionality.  If |natbib| is asked for, then the appropriate
% style files are used in place of the standard ones.
% \changes{v2.0}{2007/01/17}{Replaced custom code with
% \texttt{showkeys} package}
%    \begin{macrocode}
\ifACS@list
  \ifACS@natbib
    \bibliographystyle{achemlnt}
  \else
    \bibliographystyle{achemsol}
  \fi
  \RequirePackage[notcite]{showkeys}
\else
  \ifACS@natbib
    \bibliographystyle{achemnat}
  \else
    \bibliographystyle{achemso}
  \fi
\fi
%    \end{macrocode}
% \end{macro}
% \begin{macro}{\@biblabel}
% In order to re-format the bibliography labels, the easiest
% method is to redefine the \cmd{\@biblabel} macro from the LaTeX
% kernel.
%    \begin{macrocode}
\def\@biblabel#1{#1.}
%    \end{macrocode}
% \end{macro}
% \begin{macro}{\ACS@bibwarning}
% \begin{macro}{\bibliographystyle}
% To ensure that additional \cmd{\bibliographystyle} commands in the
% source are killed off.  The \cmd{\ACS@bibwarning} provides a clean
% method of generating the warning message.
% \changes{v2.0}{2007-01-17}{Command ignored in document body}
%    \begin{macrocode}
\def\ACS@bibwarning{\PackageWarning{achemso}%
  {Additional bibliographystyle command ignored}}
\def\bibliographystyle{\ACS@bibwarning\@gobble}
%    \end{macrocode}
% \end{macro}
% \end{macro}
% The package is complete.
%    \begin{macrocode}
%</package>
%    \end{macrocode}
% \PrintChanges
% \PrintIndex
% \Finale
% \iffalse
%<*bib>
ENTRY
  { address
%<list>    annotate
    author
    booktitle
    chapter
    doi
    edition
    editor
    howpublished
    institution
    journal
%<nat>    key
    note
    number
    organization
    pages
    publisher
    school
    series
    title
    type
    url
    volume
    year
  }
  {}
  {
  label
%<nat>    extra.label
%<nat>    short.list
  }

INTEGERS { output.state before.all mid.sentence after.sentence }
INTEGERS { after.block after.item author.or.editor }
INTEGERS { separate.by.semicolon }

FUNCTION {init.state.consts}
{ #0 'before.all :=
  #1 'mid.sentence :=
  #2 'after.sentence :=
  #3 'after.block :=
  #4 'after.item :=
}

FUNCTION {add.comma}
{ ", " * }

FUNCTION {add.semicolon}
{ "; " * }

%    \end{macrocode}
% For authors/editors we need to be able to add either a semi-colon
% or a comma.  This is done using a switching function, defined here.
%    \begin{macrocode}

FUNCTION {add.comma.or.semicolon}
{ #1 separate.by.semicolon =
    'add.semicolon
    'add.comma
  if$
}

FUNCTION {add.colon}
{ ": " * }

STRINGS { s t }

FUNCTION {output.nonnull}
{ 's :=
  output.state mid.sentence =
    { add.comma write$ }
    { output.state after.block =
      { add.semicolon write$
        newline$
        "\newblock " write$
      }
      { output.state before.all =
          'write$
          { output.state after.item =
            { " " * write$ }
            { add.period$ " " * write$ }
          if$
          }
        if$
        }
      if$
      mid.sentence 'output.state :=
    }
  if$
  s
}

FUNCTION {output}
{ duplicate$ empty$
    'pop$
    'output.nonnull
  if$
}

FUNCTION {output.check}
{ 't :=
  duplicate$ empty$
    { pop$ "Empty " t * " in " * cite$ * warning$ }
    'output.nonnull
  if$
}

%    \end{macrocode}
% For the standard file types, |output.bibitem| can come here.
% The same is not true for styles supporting |natbib|, and so
% |output.bibitem| occurs later for those styles.
% \iffalse
%<*!nat>
% \fi
%    \begin{macrocode}
FUNCTION {output.bibitem}
{ newline$
  "\bibitem{" write$
  cite$ write$
  "}" write$
  newline$
  ""
  before.all 'output.state :=
}

%    \end{macrocode}
% \iffalse
%</!nat>
% \fi
%    \begin{macrocode}
FUNCTION {new.block}
{ output.state before.all =
    'skip$
    { after.block 'output.state := }
  if$
}

FUNCTION {new.sentence}
{ output.state after.block =
    'skip$
    { output.state before.all =
        'skip$
        { after.sentence 'output.state := }
      if$
    }
  if$
}
%<*list>

FUNCTION {add.note}
{ annotate empty$
    'skip$
    { new.block
      "{\footnotesize " annotate * "}" * output }
  if$
}
%</list>

FUNCTION {fin.entry}
%<list>{ add.note
%<list>  add.period$
%<!list>{ add.period$
  write$
  newline$
}
FUNCTION {not}
{   { #0 }
    { #1 }
  if$
}

FUNCTION {and}
{   'skip$
    { pop$ #0 }
  if$
}

FUNCTION {or}
{   { pop$ #1 }
    'skip$
  if$
}

FUNCTION {field.or.null}
{ duplicate$ empty$
    { pop$ "" }
    'skip$
  if$
}

FUNCTION {emphasize}
{ duplicate$ empty$
    { pop$ "" }
    { "\emph{" swap$ * "}" * }
  if$
}

FUNCTION {boldface}
{ duplicate$ empty$
    { pop$ "" }
    { "\textbf{" swap$ * "}" * }
  if$
}

FUNCTION {paren}
{ duplicate$ empty$
    { pop$ "" }
    { "(" swap$ * ")" * }
  if$
}

FUNCTION {bbl.and}
{ "and" }

FUNCTION {bbl.chapter}
{ "Chapter" }

FUNCTION {bbl.editor}
{ "Ed." }

FUNCTION {bbl.editors}
{ "Eds." }

FUNCTION {bbl.edition}
{ "ed." }

FUNCTION {bbl.etal}
{ "et~al." }

FUNCTION {bbl.in}
{ "In" }

FUNCTION {bbl.inpress}
{ "in press" }

FUNCTION {bbl.msc}
{ "M.Sc.\ thesis" }

FUNCTION {bbl.page}
{ "p" }

FUNCTION {bbl.pages}
{ "pp" }

FUNCTION {bbl.phd}
{ "Ph.D.\ thesis" }

FUNCTION {bbl.submitted}
{ "submitted for publication" }

FUNCTION {bbl.techreport}
{ "Technical Report" }

FUNCTION {bbl.version}
{ "version" }

FUNCTION {bbl.volume}
{ "Vol." }

FUNCTION {bbl.first}
{ "1st" }

FUNCTION {bbl.second}
{ "2nd" }

FUNCTION {bbl.third}
{ "3rd" }

FUNCTION {bbl.fourth}
{ "4th" }

FUNCTION {bbl.fifth}
{ "5th" }

FUNCTION {bbl.st}
{ "st" }

FUNCTION {bbl.nd}
{ "nd" }

FUNCTION {bbl.rd}
{ "rd" }

FUNCTION {bbl.th}
{ "th" }

FUNCTION {eng.ord}
{ duplicate$ "1" swap$ *
  #-2 #1 substring$ "1" =
     { bbl.th * }
     { duplicate$ #-1 #1 substring$
       duplicate$ "1" =
         { pop$ bbl.st * }
         { duplicate$ "2" =
             { pop$ bbl.nd * }
             { "3" =
                 { bbl.rd * }
                 { bbl.th * }
               if$
             }
           if$
          }
       if$
     }
   if$
}

FUNCTION{is.a.digit}
{ duplicate$ "" =
    {pop$ #0}
    {chr.to.int$ #48 - duplicate$
     #0 < swap$ #9 > or not}
  if$
}

FUNCTION{is.a.number}
{
  { duplicate$ #1 #1 substring$ is.a.digit }
    {#2 global.max$ substring$}
  while$
  "" =
}

FUNCTION {extract.num}
{ duplicate$ 't :=
  "" 's :=
  { t empty$ not }
  { t #1 #1 substring$
    t #2 global.max$ substring$ 't :=
    duplicate$ is.a.number
      { s swap$ * 's := }
      { pop$ "" 't := }
    if$
  }
  while$
  s empty$
    'skip$
    { pop$ s }
  if$
}

FUNCTION {bibinfo.check}
{ swap$
  duplicate$ missing$
    { pop$ pop$
      ""
    }
    { duplicate$ empty$
        {
          swap$ pop$
        }
        { swap$
          pop$
        }
      if$
    }
  if$
}

FUNCTION {convert.edition}
{ extract.num "l" change.case$ 's :=
  s "first" = s "1" = or
    { bbl.first 't := }
    { s "second" = s "2" = or
        { bbl.second 't := }
        { s "third" = s "3" = or
            { bbl.third 't := }
            { s "fourth" = s "4" = or
                { bbl.fourth 't := }
                { s "fifth" = s "5" = or
                    { bbl.fifth 't := }
                    { s #1 #1 substring$ is.a.number
                        { s eng.ord 't := }
                        { edition 't := }
                      if$
                    }
                  if$
                }
              if$
            }
          if$
        }
      if$
    }
  if$
  t
}

FUNCTION {tie.or.space.connect}
{ duplicate$ text.length$ #3 <
    { "~" }
    { " " }
  if$
  swap$ * *
}

FUNCTION {space.connect}
{ " " swap$ * * }

INTEGERS { nameptr namesleft numnames }

FUNCTION {format.names}
{ 's :=
  #1 'nameptr :=
  s num.names$ 'numnames :=
  numnames 'namesleft :=
  numnames #15 >
    { s #1 "{vv~}{ll,}{~f.}{,~jj}" format.name$ 't :=
      t bbl.etal space.connect
    }
    {
       { namesleft #0 > }
       { s nameptr "{vv~}{ll,}{~f.}{,~jj}" format.name$ 't :=
           nameptr #1 >
             { namesleft #1 >
               { add.comma.or.semicolon t * }
               { numnames #2 >
                 { "" * }
                 'skip$
               if$
               t "others," =
                 { bbl.etal space.connect }
                 { add.comma.or.semicolon t * }
               if$
               }
             if$
             }
           't
         if$
         nameptr #1 + 'nameptr :=
         namesleft #1 - 'namesleft :=
         }
     while$
  }
  if$
}

FUNCTION {format.authors}
{ author empty$
    { "" }
    { #1 'author.or.editor :=
      #1 'separate.by.semicolon :=
      author format.names
    }
  if$
}

FUNCTION {format.editors}
{ editor empty$
    { "" }
    { #2 'author.or.editor :=
      #0 'separate.by.semicolon :=
      editor format.names
      add.comma
      editor num.names$ #1 >
        { bbl.editors }
        { bbl.editor }
      if$
      *
    }
  if$
}

FUNCTION {n.separate.multi}
{ 't :=
  ""
  #0 'numnames :=
  t text.length$ #4 > t is.a.number and
    {
      { t empty$ not }
      { t #-1 #1 substring$ is.a.number
          { numnames #1 + 'numnames := }
          { #0 'numnames := }
        if$
        t #-1 #1 substring$ swap$ *
        t #-2 global.max$ substring$ 't :=
        numnames #4 =
          { duplicate$ #1 #1 substring$ swap$
            #2 global.max$ substring$
            "," swap$ * *
            #1 'numnames :=
          }
          'skip$
        if$
      }
      while$
    }
    { t swap$ * }
  if$
}

FUNCTION {format.bvolume}
{ volume empty$
    { "" }
    { bbl.volume volume tie.or.space.connect }
  if$
}

FUNCTION {format.title.noemph}
{ 't :=
  t empty$
    { "" }
    { t }
  if$
}

FUNCTION {format.title}
{ 't :=
  t empty$
    { "" }
    { t emphasize }
  if$
}

FUNCTION {format.number.series}
{ volume empty$
    { number empty$
       { series field.or.null }
       { series empty$
         { "There is a number but no series in " cite$ * warning$ }
         { series number space.connect }
       if$
       }
      if$
    }
    { "" }
  if$
}

FUNCTION {format.url}
{ url empty$
    { "" }
    { new.sentence "\url{" url * "}" * }
  if$
}

% The specialised |output.bibitem| needed for |natbib| support now
% follows, along with the various support macros it needs.
% \iffalse
%<*nat>
% \fi
%    \begin{macrocode}
FUNCTION {format.full.names}
{'s :=
  #1 'nameptr :=
  s num.names$ 'numnames :=
  numnames 'namesleft :=
    { namesleft #0 > }
    { s nameptr
      "{vv~}{ll}" format.name$ 't :=
      nameptr #1 >
        {
          namesleft #1 >
            { ", " * t * }
            {
              numnames #2 >
                { "," * }
                'skip$
              if$
              t "others" =
                { bbl.etal * }
                { bbl.and space.connect t space.connect }
              if$
            }
          if$
        }
        't
      if$
      nameptr #1 + 'nameptr :=
      namesleft #1 - 'namesleft :=
    }
  while$
}

FUNCTION {author.editor.full}
{ author empty$
    { editor empty$
        { "" }
        { editor format.full.names }
      if$
    }
    { author format.full.names }
  if$
}

FUNCTION {author.full}
{ author empty$
    { "" }
    { author format.full.names }
  if$
}

FUNCTION {editor.full}
{ editor empty$
    { "" }
    { editor format.full.names }
  if$
}

FUNCTION {make.full.names}
{ type$ "book" =
  type$ "inbook" =
  or
    'author.editor.full
    { type$ "proceedings" =
        'editor.full
        'author.full
      if$
    }
  if$
}

FUNCTION {output.bibitem} { newline$
  "\bibitem[" write$
  label write$
  ")" make.full.names duplicate$ short.list =
     { pop$ }
     { * }
   if$
  "]{" * write$
  cite$ write$
  "}" write$
  newline$
  ""
  before.all 'output.state :=
}

%    \end{macrocode}
% \iffalse
%</nat>
% \fi
%    \begin{macrocode}
FUNCTION {n.dashify}
{ 't :=
  ""
    { t empty$ not }
    { t #1 #1 substring$ "-" =
    { t #1 #2 substring$ "--" = not
        { "--" *
          t #2 global.max$ substring$ 't :=
        }
        {   { t #1 #1 substring$ "-" = }
        { "-" *
          t #2 global.max$ substring$ 't :=
        }
          while$
        }
      if$
    }
    { t #1 #1 substring$ *
      t #2 global.max$ substring$ 't :=
    }
      if$
    }
  while$
}

FUNCTION {format.date}
{ year empty$
    { "" }
    { year boldface }
  if$
}

FUNCTION {format.bdate}
{ year empty$
    { "There's no year in " cite$ * warning$ }
    'year
  if$
}

FUNCTION {either.or.check}
{ empty$
    'pop$
    { "Can't use both " swap$ * " fields in " * cite$ * warning$ }
  if$
}

FUNCTION {format.edition}
{ edition duplicate$ empty$
    'skip$
    { convert.edition
      bbl.edition bibinfo.check
      " " * bbl.edition *
    }
  if$
}

INTEGERS { multiresult }

FUNCTION {multi.page.check}
{ 't :=
  #0 'multiresult :=
    { multiresult not
      t empty$ not
      and
    }
    { t #1 #1 substring$
      duplicate$ "-" =
      swap$ duplicate$ "," =
      swap$ "+" =
      or or
        { #1 'multiresult := }
        { t #2 global.max$ substring$ 't := }
      if$
    }
  while$
  multiresult
}

FUNCTION {format.pages}
{ pages empty$
    { "" }
    { pages multi.page.check
      { bbl.pages pages n.dashify tie.or.space.connect }
      { bbl.page pages tie.or.space.connect }
    if$
    }
  if$
}


FUNCTION {format.pages.required}
{ pages empty$
    { ""
      "There are no page numbers for " cite$ * warning$
      output
    }
    { pages multi.page.check
      { bbl.pages pages n.dashify tie.or.space.connect }
      { bbl.page pages tie.or.space.connect }
    if$
    }
  if$
}

FUNCTION {format.pages.nopp}
{ pages empty$
    { ""
      "There are no page numbers for " cite$ * warning$
      output
    }
    { pages multi.page.check
      { pages n.dashify space.connect }
      { pages space.connect }
    if$
    }
  if$
}

FUNCTION {format.pages.patent}
{ pages empty$
    { "There is no patent number for " cite$ * warning$ }
    { pages multi.page.check
      { pages n.dashify }
      { pages n.separate.multi }
      if$
    }
  if$
}

FUNCTION {format.vol.pages}
{ volume emphasize field.or.null
  duplicate$ empty$
    { pop$ format.pages.required }
    { add.comma pages n.dashify * }
  if$
}

FUNCTION {format.chapter.pages}
{ chapter empty$
    'format.pages
    { type empty$
    { bbl.chapter }
    { type "l" change.case$ }
      if$
      chapter tie.or.space.connect
      pages empty$
    'skip$
    { add.comma format.pages * }
      if$
    }
  if$
}

FUNCTION {format.title.in}
{ 's :=
  s empty$
    { "" }
    { editor empty$
      { bbl.in s format.title space.connect }
      { bbl.in s format.title space.connect
        add.semicolon format.editors *
      }
    if$
    }
  if$
}

FUNCTION {format.pub.address}
{ publisher empty$
    { "" }
    { address empty$
        { publisher }
        { publisher add.colon address *}
      if$
    }
  if$
}

FUNCTION {format.school.address}
{ school empty$
    { "" }
    { address empty$
        { school }
        { school add.colon address *}
      if$
    }
  if$
}

FUNCTION {format.organization.address}
{ organization empty$
    { "" }
    { address empty$
        { organization }
        { organization add.colon address *}
      if$
    }
  if$
}

FUNCTION {format.version}
{ edition empty$
    { "" }
    { bbl.version edition tie.or.space.connect }
  if$
}

FUNCTION {empty.misc.check}
{ author empty$ title empty$ howpublished empty$
  year empty$ note empty$ url empty$
  and and and and and
    { "all relevant fields are empty in " cite$ * warning$ }
    'skip$
  if$
}

FUNCTION {empty.doi.note}
{ doi empty$ note empty$ and
    { "Need either a note or DOI for " cite$ * warning$ }
    'skip$
  if$
}

FUNCTION {format.thesis.type}
{ type empty$
    'skip$
    { pop$
      type emphasize
    }
  if$
}

FUNCTION {article}
{ output.bibitem
  format.authors "author" output.check
  after.item 'output.state :=
  journal emphasize "journal" output.check
  after.item 'output.state :=
  format.date "year" output.check
  volume empty$
    { ""
      format.pages.nopp output
    }
    { format.vol.pages output }
  if$
  note output
  fin.entry
}

FUNCTION {book}
{ output.bibitem
  author empty$
    { booktitle empty$
        { title format.title "title" output.check }
        { booktitle format.title "booktitle" output.check }
      if$
      format.edition output
      new.block
      editor empty$
        { "Need either an author or editor for " cite$ * warning$ }
        { "" format.editors * "editor" output.check }
      if$
    }
    { format.authors output
      after.item 'output.state :=
      "author and editor" editor either.or.check
      booktitle empty$
        { title format.title "title" output.check }
        { booktitle format.title "booktitle" output.check }
      if$
      format.edition output
    }
  if$
  new.block
  format.number.series output
  new.block
  format.pub.address "publisher" output.check
  format.bdate "year" output.check
  new.block
  format.bvolume output
  pages empty$
    'skip$
    { format.pages output }
  if$
  note output
  fin.entry
}

FUNCTION {booklet}
{ output.bibitem
  format.authors output
  after.item 'output.state :=
  title format.title "title" output.check
  howpublished output
  address output
  format.date output
  note output
  fin.entry
}

FUNCTION {inbook}
{ output.bibitem
  author empty$
    { title format.title "title" output.check
      format.edition output
      new.block
      editor empty$
      { "Need at least an author or an editor for " cite$ * warning$ }
      { "" format.editors * "editor" output.check }
    if$
    }
    { format.authors output
      after.item 'output.state :=
      title format.title.in "title" output.check
      format.edition output
    }
  if$
  new.block
  format.number.series output
  new.block
  format.pub.address "publisher" output.check
  format.bdate "year" output.check
  new.block
  format.bvolume output
  format.chapter.pages "chapter and pages" output.check
  note output
  fin.entry
}

FUNCTION {incollection}
{ output.bibitem
  author empty$
    { booktitle format.title "booktitle" output.check
      format.edition output
      new.block
      editor empty$
        { "Need at least an author or an editor for " cite$ * warning$ }
        { "" format.editors * "editor" output.check }
      if$
    }
    { format.authors output
      after.item 'output.state :=
      title empty$
        'skip$
        { title format.title.noemph output }
      if$
      after.sentence 'output.state :=
      booktitle format.title.in "booktitle" output.check
      format.edition output
    }
  if$
  new.block
  format.number.series output
  new.block
  format.pub.address "publisher" output.check
  format.bdate "year" output.check
  new.block
  format.bvolume output
  format.chapter.pages "chapter and pages" output.check
  note output
  fin.entry
}

FUNCTION {inpress}
{ output.bibitem
  format.authors "author" output.check
  after.item 'output.state :=
  journal emphasize "journal" output.check
  doi empty$
    {  bbl.inpress output }
    {  after.item 'output.state :=
       format.date output
       "DOI:" doi tie.or.space.connect output
    }
  if$
  note output
  fin.entry
}

FUNCTION {inproceedings}
{ output.bibitem
  format.authors "author" output.check
  after.item 'output.state :=
  title empty$
    'skip$
    { title format.title.noemph output
      after.sentence 'output.state :=
    }
  if$
  booktitle format.title output
  address output
  format.bdate "year" output.check
  pages empty$
    'skip$
    { new.block
      format.pages output }
  if$
  note output
  fin.entry
}

FUNCTION {manual}
{ output.bibitem
  format.authors output
  after.item 'output.state :=
  title format.title "title" output.check
  format.version output
  new.block
  format.organization.address output
  format.bdate output
  note output
  fin.entry
}

FUNCTION {mastersthesis}
{ output.bibitem
  format.authors "author" output.check
  after.item 'output.state :=
  bbl.msc format.thesis.type output
  format.school.address "school" output.check
  format.bdate "year" output.check
  note output
  fin.entry
}

FUNCTION {misc}
{ output.bibitem
  format.authors output
  after.item 'output.state :=
  title empty$
    'skip$
    { title format.title output }
  if$
  howpublished output
  year output
  format.url output
  note output
  fin.entry
  empty.misc.check
}

FUNCTION {patent}
{ output.bibitem
  format.authors "author" output.check
  after.item 'output.state :=
  journal "journal" output.check
  after.item 'output.state :=
  format.pages.patent "pages" output.check
  format.bdate "year" output.check
  note output
  fin.entry
}

FUNCTION {phdthesis}
{ output.bibitem
  format.authors "author" output.check
  after.item 'output.state :=
  bbl.phd format.thesis.type output
  format.school.address "school" output.check
  format.bdate "year" output.check
  note output
  fin.entry
}

FUNCTION {proceedings}
{ output.bibitem
  title format.title.noemph "title" output.check
  address output
  format.bdate "year" output.check
  pages empty$
    'skip$
    { new.block
      format.pages output }
  if$
  note output
  fin.entry
}

FUNCTION {remark}
{ output.bibitem
  note "note" output.check
  fin.entry
}

FUNCTION {submitted}
{ output.bibitem
  format.authors "author" output.check
  bbl.submitted output
  fin.entry
}

FUNCTION {techreport}
{ output.bibitem
  format.authors "author" output.check
  after.item 'output.state :=
  title format.title "title" output.check
  new.block
  type empty$
    'bbl.techreport
    'type
  if$
  number empty$
    'skip$
    { number tie.or.space.connect }
  if$
  output
  format.pub.address output
  format.bdate "year" output.check
  pages empty$
    'skip$
    { new.block
      format.pages output }
  if$
  note output
  fin.entry
}

FUNCTION {unpublished}
{ output.bibitem
  format.authors "author" output.check
  after.item 'output.state :=
  journal empty$
    'skip$
    { journal emphasize "journal" output.check }
  if$
  doi empty$
    {  note output }
    {  after.item 'output.state :=
       format.date output
       "DOI:" doi tie.or.space.connect output
    }
  if$
  fin.entry
  empty.doi.note
}

FUNCTION {conference} {inproceedings}

FUNCTION {other} {patent}

FUNCTION {default.type} {misc}

MACRO {jan} {"Jan."}
MACRO {feb} {"Feb."}
MACRO {mar} {"Mar."}
MACRO {apr} {"Apr."}
MACRO {may} {"May"}
MACRO {jun} {"June"}
MACRO {jul} {"July"}
MACRO {aug} {"Aug."}
MACRO {sep} {"Sept."}
MACRO {oct} {"Oct."}
MACRO {nov} {"Nov."}
MACRO {dec} {"Dec."}

MACRO {acchemr} {"Acc.\ Chem.\ Res."}
MACRO {aacsa} {"Adv.\ {ACS} Abstr."}
MACRO {anchem} {"Anal.\ Chem."}
MACRO {bioch} {"Biochemistry"}
MACRO {bicoc} {"Bioconj.\ Chem."}  % ***
MACRO {bitech} {"Biotechnol.\ Progr."}  % ***
MACRO {chemeng} {"Chem.\ Eng.\ News"}
MACRO {chs} {"Chem.\ Health Safety"} % ***
MACRO {crt} {"Chem.\ Res.\ Toxicol."} % ***
MACRO {chemrev} {"Chem.\ Rev."} % ***
MACRO {cmat} {"Chem.\ Mater."} % ***
MACRO {chemtech} {"{CHEMTECH}"} % ***
MACRO {enfu} {"Energy Fuels"} % ***
MACRO {envst} {"Environ.\ Sci.\ Technol."}
MACRO {iecf} {"Ind.\ Eng.\ Chem.\ Fundam."}
MACRO {iecpdd} {"Ind.\ Eng.\ Chem.\ Proc.\ Des.\ Dev."}
MACRO {iecprd} {"Ind.\ Eng.\ Chem.\ Prod.\ Res.\ Dev."}
MACRO {iecr} {"Ind.\ Eng.\ Chem.\ Res."} % ***
MACRO {inor} {"Inorg.\ Chem."}
MACRO {jafc} {"J.~Agric.\ Food Chem."}
MACRO {jacs} {"J.~Am.\ Chem.\ Soc."}
MACRO {jced} {"J.~Chem.\ Eng.\ Data"}
MACRO {jcics} {"J.~Chem.\ Inf.\ Comput.\ Sci."}
MACRO {jmc} {"J.~Med.\ Chem."}
MACRO {joc} {"J.~Org.\ Chem."}
MACRO {jps} {"J.~Pharm.\ Sci."}
MACRO {jpcrd} {"J.~Phys.\ Chem.\ Ref.\ Data"} % ***
MACRO {jpc} {"J.~Phys.\ Chem."}
MACRO {jpca} {"J.~Phys.\ Chem.~A"}
MACRO {jpcb} {"J.~Phys.\ Chem.~B"}
MACRO {lang} {"Langmuir"}
MACRO {macro} {"Macromolecules"}
MACRO {orgmet} {"Organometallics"}
MACRO {orglett} {"Org.\ Lett."}

MACRO {jft} {"J.~Chem.\ Soc., Faraday Trans."}
MACRO {jft1} {"J.~Chem.\ Soc., Faraday Trans.\ 1"}
MACRO {jft2} {"J.~Chem.\ Soc., Faraday Trans.\ 2"}
MACRO {tfs} {"Trans.\ Faraday Soc."}
MACRO {jcis} {"J.~Colloid Interface Sci."}
MACRO {acis} {"Adv.~Colloid Interface Sci."}
MACRO {cs} {"Colloids Surf."}
MACRO {csa} {"Colloids Surf.\ A"}
MACRO {csb} {"Colloids Surf.\ B"}
MACRO {pcps} {"Progr.\ Colloid Polym.\ Sci."}
MACRO {jmr} {"J.~Magn.\ Reson."}
MACRO {jmra} {"J.~Magn.\ Reson.\ A"}
MACRO {jmrb} {"J.~Magn.\ Reson.\ B"}
MACRO {sci} {"Science"}
MACRO {nat} {"Nature"}
MACRO {jcch} {"J.~Comput.\ Chem."}
MACRO {cca} {"Croat.\ Chem.\ Acta"}
MACRO {angew} {"Angew.\ Chem., Int.\ Ed."}
MACRO {chemeurj} {"Chem.---Eur.\ J."}

MACRO {poly} {"Polymer"}
MACRO {ajp} {"Am.\ J.\ Phys."}
MACRO {rsi} {"Rev.\ Sci.\ Instrum."}
MACRO {jcp} {"J.~Chem.\ Phys."}
MACRO {cpl} {"Chem.\ Phys.\ Lett."}
MACRO {molph} {"Mol.\ Phys."}
MACRO {pac} {"Pure Appl.\ Chem."}
MACRO {jbc} {"J.~Biol.\ Chem."}
MACRO {tl} {"Tetrahedron Lett."}
MACRO {psisoe} {"Proc.\ SPIE-Int.\ Soc.\ Opt.\ Eng."}
MACRO {prb} {"Phys.\ Rev.\ B:\ Condens.\ Matter Mater. Phys."}
MACRO {jap} {"J.~Appl.\ Phys."}
MACRO {pnac} {"Proc.\ Natl.\ Acad.\ Sci.\ U.S.A."}
MACRO {bba} {"Biochim.\ Biophys.\ Acta"}
MACRO {nar} {"Nucleic.\ Acid Res."}

READ

%    \end{macrocode}
% The nature of the initialise code depends on whether we need to
% support |natbib|.  First the simple case is handled.
% \iffalse
%<*!nat>
% \fi
%    \begin{macrocode}
STRINGS { longest.label }

INTEGERS { number.label longest.label.width }

FUNCTION {initialize.longest.label}
{ "" 'longest.label :=
  #1 'number.label :=
  #0 'longest.label.width :=
}

FUNCTION {longest.label.pass}
{ number.label int.to.str$ 'label :=
  number.label #1 + 'number.label :=
  label width$ longest.label.width >
    { label 'longest.label :=
      label width$ 'longest.label.width :=
    }
    'skip$
  if$
}

EXECUTE {initialize.longest.label}

ITERATE {longest.label.pass}

%    \end{macrocode}
% \iffalse
%</!nat>
% \fi
% Now the |natbib| system is sorted out, basically by copying from
% |plainnat.bst|.
% \iffalse
%<*nat>
% \fi
%    \begin{macrocode}
INTEGERS { len }

FUNCTION {chop.word}
{ 's :=
  'len :=
  s #1 len substring$ =
    { s len #1 + global.max$ substring$ }
    's
  if$
}

FUNCTION {format.lab.names}
{ 's :=
  s #1 "{vv~}{ll}" format.name$
  s num.names$ duplicate$
  #2 >
    { pop$ bbl.etal space.connect }
    { #2 <
        'skip$
        { s #2 "{ff }{vv }{ll}{ jj}" format.name$ "others" =
            { bbl.etal space.connect }
            { bbl.and space.connect s #2 "{vv~}{ll}" format.name$ space.connect }
          if$
        }
      if$
    }
  if$
}

FUNCTION {author.key.label}
{ author empty$
    { key empty$
        { cite$ #1 #3 substring$ }
        'key
      if$
    }
    { author format.lab.names }
  if$
}

FUNCTION {author.editor.key.label}
{ author empty$
    { editor empty$
        { key empty$
            { cite$ #1 #3 substring$ }
            'key
          if$
        }
        { editor format.lab.names }
      if$
    }
    { author format.lab.names }
  if$
}

FUNCTION {author.key.organization.label}
{ author empty$
    { key empty$
        { organization empty$
            { cite$ #1 #3 substring$ }
            { "The " #4 organization chop.word #3 text.prefix$ }
          if$
        }
        'key
      if$
    }
    { author format.lab.names }
  if$
}

FUNCTION {editor.key.organization.label}
{ editor empty$
    { key empty$
        { organization empty$
            { cite$ #1 #3 substring$ }
            { "The " #4 organization chop.word #3 text.prefix$ }
          if$
        }
        'key
      if$
    }
    { editor format.lab.names }
  if$
}

FUNCTION {calc.short.authors}
{ type$ "book" =
  type$ "inbook" =
  or
    'author.editor.key.label
    { type$ "proceedings" =
        'editor.key.organization.label
        { type$ "manual" =
            'author.key.organization.label
            'author.key.label
          if$
        }
      if$
    }
  if$
  'short.list :=
}

FUNCTION {calc.label}
{ calc.short.authors
  short.list
  "("
  *
  year duplicate$ empty$
  short.list key field.or.null = or
     { pop$ "" }
     'skip$
  if$
  *
  'label :=
}

ITERATE {calc.label}

STRINGS { longest.label last.label next.extra }

INTEGERS { longest.label.width last.extra.num number.label }

FUNCTION {initialize.longest.label}
{ "" 'longest.label :=
  #0 int.to.chr$ 'last.label :=
  "" 'next.extra :=
  #0 'longest.label.width :=
  #0 'last.extra.num :=
  #0 'number.label :=
}

FUNCTION {forward.pass}
{ last.label label =
    { last.extra.num #1 + 'last.extra.num :=
      last.extra.num int.to.chr$ 'extra.label :=
    }
    { "a" chr.to.int$ 'last.extra.num :=
      "" 'extra.label :=
      label 'last.label :=
    }
  if$
  number.label #1 + 'number.label :=
}

EXECUTE {initialize.longest.label}

ITERATE {forward.pass}

%    \end{macrocode}
% \iffalse
%</nat>
% \fi

FUNCTION {begin.bib}
{ preamble$ empty$
    'skip$
    { preamble$ write$ newline$ }
  if$
  "\providecommand{\url}[1]{\texttt{#1}}"
  write$ newline$
  "\providecommand{\refin}[1]{\\ \textbf{Referenced in:} #1}"
  write$ newline$
%<nat>  "\providecommand{\natexlab}[1]{#1}"
%<nat>  write$ newline$
%<!nat>  "\begin{thebibliography}{"  longest.label  * "}" *
%<nat>  "\begin{thebibliography}{" number.label int.to.str$ * "}" *
  write$ newline$
}

EXECUTE {begin.bib}

EXECUTE {init.state.consts}

ITERATE {call.type$}

FUNCTION {end.bib}
{ newline$
  "\end{thebibliography}" write$ newline$
}

EXECUTE {end.bib}
%</bib>
%<*database>
@BOOK{Coghill2006,
  title = {{T}he {ACS} {S}tyle {G}uide},
  publisher = {{O}xford {U}niversity {P}ress, {I}nc. and
               {T}he {A}merican {C}hemical {S}ociety},
  year = {2006},
  editor = {Coghill, Anne M. and Garson, Lorrin R.},
  address = {{N}ew {Y}ork},
  edition = {3},
  subtitle = {{E}ffective {C}ommunication of {S}cientific {I}nformation},
}

@MISC{ACS2007,
  url = {http://pubs.acs.org/books/references.shtml},
}
%</database>
%<*jawltxdoc>
% The following is convenient method for collecting together package
% loading, formatting commands and new macros used to format |dtx|
% files written by the current author.  It is based on the similar
% files provided by Will Robertson in his packages and Heiko Oberdiek
% as a stand-alone package.  Notice that it is not intended for other
% users: there is no error checking!  However, it is covered by the
% LPPL in the same way as the rest of this package.
%
\NeedsTeXFormat{LaTeX2e}
\ProvidesPackage{jawltxdoc}
  [2007/10/14 v1.0b]
% First of all, a number of support packages are loaded.
\usepackage[T1]{fontenc}
\usepackage[english,UKenglish]{babel}
\usepackage[scaled=0.95]{helvet}
\usepackage[version=3]{mhchem}
\usepackage[final]{microtype}
\usepackage[osf]{mathpazo}
\usepackage{booktabs,array,url,graphicx,courier,unitsdef}
\usepackage{upgreek,ifpdf,listings}
% If using PDFLaTeX, the source will be attached to the PDF. This
% is basically the system used by Heiko Oberdiek, but with a check
% that PDF mode is enabled.
\ifpdf
  \usepackage{embedfile}
  \embedfile[%
    stringmethod=escape,%
    mimetype=plain/text,%
    desc={LaTeX docstrip source archive for package `\jobname'}%
    ]{\jobname.dtx}
\fi
\usepackage{\jobname}
\usepackage[numbered]{hypdoc}
%
% To typeset examples, a new environment is needed.  The code below
% is based on that in used by |listings|, but is modified to get
% better formatting for this context.  The formatting of the output
% is basically that in Will Robertson's |dtx-style| file.
\newlength\LaTeXwidth
\newlength\LaTeXoutdent
\newlength\LaTeXgap
\setlength\LaTeXgap{1em}
\setlength\LaTeXoutdent{-0.15\textwidth}
\def\typesetexampleandcode{%
  \begin{list}{}{%
    \setlength\itemindent{0pt}
    \setlength\leftmargin\LaTeXoutdent
    \setlength\rightmargin{0pt}
  }
  \item
    \setlength\LaTeXoutdent{-0.15\textwidth}
    \begin{minipage}[c]{\textwidth-\LaTeXwidth-\LaTeXoutdent-\LaTeXgap}
      \lst@sampleInput
    \end{minipage}%
    \hfill%
    \begin{minipage}[c]{\LaTeXwidth}%
      \hbox to\linewidth{\box\lst@samplebox\hss}%
    \end{minipage}%
  \end{list}
}
\def\typesetcodeandexample{%
  \begin{list}{}{%
    \setlength\itemindent{0pt}
    \setlength\leftmargin{0pt}
    \setlength\rightmargin{0pt}
  }
  \item
    \begin{minipage}[c]{\LaTeXwidth}%
      \hbox to\linewidth{\box\lst@samplebox\hss}%
    \end{minipage}%
    \lst@sampleInput
  \end{list}
}
\def\typesetfloatexample{%
  \begin{list}{}{%
    \setlength\itemindent{0pt}
    \setlength\leftmargin{0pt}
    \setlength\rightmargin{0pt}
  }
  \item
    \lst@sampleInput
    \begin{minipage}[c]{\LaTeXwidth}%
      \hbox to\linewidth{\box\lst@samplebox\hss}%
    \end{minipage}%
  \end{list}
}
\def\typesetcodeonly{%
  \begin{list}{}{%
    \setlength\itemindent{0pt}
    \setlength\leftmargin{0pt}
    \setlength\rightmargin{0pt}
  }
  \item
    \begin{minipage}[c]{\LaTeXwidth}%
      \hbox to\linewidth{\box\lst@samplebox\hss}%
    \end{minipage}%
  \end{list}
}
\edef\LaTeXexamplefile{\jobname.tmp}
\lst@RequireAspects{writefile}
\newbox\lst@samplebox
\lstnewenvironment{LaTeXexample}[1][\typesetexampleandcode]{%
  \let\typesetexample#1
  \global\let\lst@intname\@empty
  \setbox\lst@samplebox=\hbox\bgroup
  \setkeys{lst}{language=[LaTeX]{TeX},tabsize=4,gobble=2,%
    breakindent=0pt,basicstyle=\small\ttfamily,basewidth=0.51em,%
    keywordstyle=\color{blue},%
% Notice that new keywords should be added here. The list is simply
% macro names needed to typeset documentation of the package
% author.
    morekeywords={bibnote,citenote,bibnotetext,bibnotemark,%
      thebibnote,bibnotename,includegraphics,schemeref,%
      floatcontentsleft,floatcontentsright,floatcontentscentre,%
      schemerefmarker,compound,schemerefformat,color,%
      startchemical,stopchemical,chemical,setupchemical,bottext,%
      listofschemes}}
  \lst@BeginAlsoWriteFile{\LaTeXexamplefile}
}{%
  \lst@EndWriteFile\egroup
  \setlength\LaTeXwidth{\wd\lst@samplebox}
  \typesetexample%
}
\def\lst@sampleInput{%
  \MakePercentComment\catcode`\^^M=10\relax
  \small%
  {\setkeys{lst}{SelectCharTable=\lst@ReplaceInput{\^\^I}%
    {\lst@ProcessTabulator}}%
    \leavevmode \input{\LaTeXexamplefile}}%
  \MakePercentIgnore%
}
\hyphenation{PDF-LaTeX}
%</jawltxdoc>
%\fi

%
% Documentation:
%    (a) Without write18 enabled:
%          pdflatex achemso.dtx
%          bibtex8 --wolfgang achemso.aux
%          makeindex -s gind.ist achemso.idx
%          makeindex -s gglo.ist -o achemso.gls  achemso.glo
%          pdflatex achemso.dtx
%          makeindex -s gind.ist achemso.idx
%          makeindex -s gglo.ist -o achemso.gls  achemso.glo
%          pdflatex achemso.dtx
%    (b) With write18 enabled:
%          pdflatex achemso.dtx
%          bibtex8 --wolfgang achemso.aux
%          pdflatex achemso.dtx
%          pdflatex achemso.dtx
%
% Installation:
%     Copy achemso.sty and the achmes*.bst files to a location
%     searched by TeX, and if required by your TeX installation,
%     run the appropriate command to build a hash of files
%     (texhash, mpm --update-db, etc.)
%
% Note:
%     The jawltxdoc.sty file is not needed for installation,
%     only for building the documentation.  It may be deleted.
%
%<*ignore>
% This is all taken verbatim from Heiko Oberdiek's packages
\begingroup
  \def\x{LaTeX2e}%
\expandafter\endgroup
\ifcase 0\ifx\install y1\fi\expandafter
         \ifx\csname processbatchFile\endcsname\relax\else1\fi
         \ifx\fmtname\x\else 1\fi\relax
\else\csname fi\endcsname
%</ignore>
%<*install>
\input docstrip.tex
\keepsilent
\askforoverwritefalse
\preamble
 ----------------------------------------------------------------
 The achemso package - LaTeX and BibTeX support for American
 Chemical Society publications
 Maintained by Joseph Wright
 E-mail: joseph.wright@morningstar2.co.uk
 Released under the LaTeX Project Public License v1.3 or later
 See http://www.latex-project.org/lppl.txt
 ----------------------------------------------------------------

\endpreamble
\Msg{Generating achemso files:}
\usedir{tex/latex/contib/achemso}
\generate{\file{\jobname.ins}{\from{\jobname.dtx}{install}}
          \file{\jobname.sty}{\from{\jobname.dtx}{package}}
          \file{jawltxdoc.sty}{\from{\jobname.dtx}{jawltxdoc}}
}
\declarepostamble\bibtexable
\endpostamble
\usedir{bibtex/bst/achemso}
\generate{\usepostamble\bibtexable
          \file{achemso.bst}{\from{achemso.dtx}{bib}}
          \file{achemnat.bst}{\from{achemso.dtx}{bib,nat}}
          \file{achemsol.bst}{\from{achemso.dtx}{bib,list}}
          \file{achemlnt.bst}{\from{achemso.dtx}{bib,list,nat}}
}
\generate{\usepostamble\empty\usepreamble\empty
          \file{achemso.bib}{\from{achemso.dtx}{database}}
}
\endbatchfile
%</install>
%<*ignore>
\fi
% Will Robertson's trick
\immediate\write18{makeindex -s gind.ist -o \jobname.ind  \jobname.idx}
\immediate\write18{makeindex -s gglo.ist -o \jobname.gls  \jobname.glo}
%</ignore>
%<*driver>
\PassOptionsToClass{a4paper}{article}
\documentclass{ltxdoc}
\EnableCrossrefs
\CodelineIndex
\RecordChanges
%\OnlyDescription
% The various formatting commands used in this file are collected
% together in |jawltxdoc|.
\usepackage{jawltxdoc}
\begin{document}
  \DocInput{\jobname.dtx}
\end{document}
%</driver>
% \fi
%
% \CheckSum{105}
%
% \CharacterTable
%  {Upper-case    \A\B\C\D\E\F\G\H\I\J\K\L\M\N\O\P\Q\R\S\T\U\V\W\X\Y\Z
%   Lower-case    \a\b\c\d\e\f\g\h\i\j\k\l\m\n\o\p\q\r\s\t\u\v\w\x\y\z
%   Digits        \0\1\2\3\4\5\6\7\8\9
%   Exclamation   \!     Double quote  \"     Hash (number) \#
%   Dollar        \$     Percent       \%     Ampersand     \&
%   Acute accent  \'     Left paren    \(     Right paren   \)
%   Asterisk      \*     Plus          \+     Comma         \,
%   Minus         \-     Point         \.     Solidus       \/
%   Colon         \:     Semicolon     \;     Less than     \<
%   Equals        \=     Greater than  \>     Question mark \?
%   Commercial at \@     Left bracket  \[     Backslash     \\
%   Right bracket \]     Circumflex    \^     Underscore    \_
%   Grave accent  \`     Left brace    \{     Vertical bar  \|
%   Right brace   \}     Tilde         \~}
%
%\GetFileInfo{\jobname.sty}
%
%\changes{v1.0}{1998/06/01}{Initial release of package by Mats
%   Dahlgren}
%\changes{v2.0}{2007/01/17}{Re-write of package by Joseph Wright}
%\changes{v2.0}{2007/01/17}{Several improvements to BibTeX style
%  files}
%\changes{v2.0}{2007/01/17}{License changed to LPPL}
%\changes{v2.1}{2007/02/15}{Updated documentation to reflect 3rd
%  edition of ACS Style Guide}
%\changes{v2.1}{2007/02/15}{BibTeX style improved to reflect 3rd
%  edition of ACS Style Guide}
%\changes{v2.2}{2007/06/05}{Added \texttt{natbib} support}
%\changes{v2.2a}{2007/07/08}{Fixed separation of editor names}
%\changes{v2.2a}{2007/07/08}{Bug fixes to \texttt{natbib} and list
% support}
%\changes{v2.2a}{2007/07/08}{\texttt{title} field included in output
% for \texttt{incollection} records}
%\changes{v2.2b}{2007/07/09}{Bug fix to name formatting}
%\changes{v2.2d}{2007/10/16}{Added \textsc{url} field to
%  \texttt{misc} output}
%\changes{v2.2d}{2007/10/16}{Package design improved}
%
%\DoNotIndex{\@biblabel,\@eha,\@gobble,\@ifpackageloaded,\@ifundefined}
%\DoNotIndex{\bibliographystyle,\bibname,\citeform,\citeleft}
%\DoNotIndex{\citenumfont,\citeright,\DeclareOption,\def,\else,\emph}
%\DoNotIndex{\fi,\ifx,\NeedsTeXFormat,\newcommand,\newif}
%\DoNotIndex{\OptionNotUsed,\PackageError,\PackageWarning}
%\DoNotIndex{\ProcessOptions,\ProvidesPackage,\refname,\relax}
%\DoNotIndex{\renewcommand,\RequirePackage,\textit,}
%
% \title{\texttt{achemso} --- LaTeX and BibTeX support for American
%   Chemical Society publications%
%   \thanks{This file describes version \fileversion, last revised
%           \filedate.}}
% \author{Joseph Wright%
%   \thanks{E-mail: joseph.wright@morningstar2.co.uk}}
% \date{Released \filedate}
%
%\maketitle
%
%\begin{abstract}
% The |achemso| package provides a BibTeX style in accordance with
% the requirements of the journals of the American Chemical Society,
% along with a supporting LaTeX package file. Also provided is a
% BibTeX style file to be used for bibliography database listings.
%\end{abstract}
%
% \section{Introduction}
%
% Synthetic chemists do not, in the main, use LaTeX for the
% preparation of journal articles. Some journals, mainly in the
% physical chemistry area, do accept LaTeX submissions.  Given the
% clear advantages of LaTeX over other methods, it would be
% nice to be able to use LaTeX for preparing reports. Thus the need
% for BibTeX styles for chemistry is real. The package |achemso|
% provides for a BibTeX style and other support for articles and
% reports in the style of the American Chemical Society (ACS).
%
% As describe in \emph{The ACS Style Guide} \cite{Coghill2006},
% almost all ACS publications use the same style for the formatting
% of references.  The reproduction of this style is the aim of the
% BibTeX style file provided here.  However, the ACS use different
% citation styles in different publications.  The |achemso| package
% provides support for the two numerical systems: superscript
% and italic in-text citations.  The majority of ACS journals use
% the superscript method (Table \ref{tbl:journals-super}), with a
% smaller number using the italic system (Table
% \ref{tbl:journals-inline}). The journal \emph{Biochemistry} does
% not use the standard ACS style for references, and so is not
% covered by the |achemso| package.
% \begin{table}
%   \centering
%   \small
%   \begin{tabular}{>{\itshape}l>{\itshape}l}
%     \toprule
%     \upshape{Journal Title} & \upshape{\emph{CASSI} Abbreviation} \\
%     \midrule
%     Accounts of Chemical Research & Acc.~Chem.~Res. \\
%     Analytical Chemistry & Anal.~Chem. \\
%     Biomacromolecules & Biomacromolecules \\
%     Chemical Reviews & Chem.~Rev. \\
%     Chemistry of Materials & Chem.~Mater. \\
%     Crystal Growth \& Design & Cryst.~Growth Des. \\
%     Energy \& Fuels & Energy Fuels \\
%     Industrial \& Engineering Chemistry Research & Ind.~Eng.~Chem.~Res. \\
%     Inorganic Chemistry & Inorg.Chem. \\
%     Journal of the American Chemical Society & J.~Am.~Chem.~Soc. \\
%     Journal of Chemical and Engineering Data & J.~Chem.~Eng.~Data \\
%     Journal of Chemical Theory and Computation & J.~Chem.~Theory Comput. \\
%     Journal of Chemical Information and Modeling & J.~Chem.~Inf.~Model. \\
%     Journal of Combinatorial Chemistry & J.~Comb.~Chem. \\
%     Journal of Medicinal Chemistry & J.~Med.~Chem. \\
%     Journal of Natural Products & J.~Nat.~Prod. \\
%     The Journal of Organic Chemistry & J.~Org.~Chem. \\
%     The Journal of Physical Chemistry A & J.~Phys.~Chem.~A \\
%     The Journal of Physical Chemistry B & J.~Phys.~Chem.~B \\
%     The Journal of Physical Chemistry C & J.~Phys.~Chem.~C \\
%     Journal of Proteome Research & J.~Proteome Res. \\
%     Langmuir & Langmuir \\
%     Macromolecules & Macromolecules \\
%     Molecular Pharmaceutics & Mol.~Pharm. \\
%     Nano Letters & Nano Lett. \\
%     Organic Letters & Org.~Lett. \\
%     Organic Process Research \& Design & Org.~Process Res.~Dev. \\
%     Organometallics & Organometallics \\
%     \bottomrule
%   \end{tabular}
%   \caption{Journals using the ACS reference style with superscript citations}
%   \label{tbl:journals-super}
% \end{table}
% \begin{table}
%   \small
%   \centering
%   \begin{tabular}{>{\itshape}l>{\itshape}l}
%     \toprule
%     \upshape{Journal Title} & \upshape{\emph{CASSI} Abbreviation} \\
%     \midrule
%     ACS Chemical Biology & ACS Chem.~Biol. \\
%     Bioconjugate Chemistry & Bioconjugate Chem. \\
%     Biotechnology Progress & Biotechnol.~Prog. \\
%     Chemical Research in Toxicology & Chem.~Res.~Toxicol. \\
%     Environmental Science and Technology & Envirn.~Sci.~Technol. \\
%     Journal of Agricultural and Food Chemistry & J.~Agric.~Food Chem. \\
%     \bottomrule
%   \end{tabular}
%   \caption{Journals using the ACS reference style with in-text citations}
%   \label{tbl:journals-inline}
% \end{table}
%
% This package consists of two BibTeX files (|achemso.bst|
% and |achemsol.bst|) along with a small LaTeX file |achemso.sty|.
% The naming of the package is slightly unusual, but follows from
% the need to pick a unique name.  To quote the documentation to the
% first version:
% \begin{quote}
%   there is already a LaTeX 2.09 and
%   BibTeX style package called |acsarticle| and
%   |acs.bst|, which are not ``ACS'' as in `American Chemical
%   Society' (rather, this package is
%   formatting the output according to the instructions of
%   \emph{Advances in Control Systems}).  Hence, \emph{this}
%   new package had to be given another name.  The name of choice
%   was then |achemso|, which is made from the words
%   ``\emph{A}merican \emph{Chem}ical \emph{So}ciety.''
% \end{quote}
%
% \subsection{Change of maintainer}
%
% This package was initially released by Mats Dahlgren.  He no
% longer has time to devote to LaTeX development.  With his permission,
% the package has therefore been taken over by Joseph
% Wright, the maintainer of the the |rsc| package.  The majority of
% the package has been rebuilt and the BibTeX style file has been
% totally overhauled.  Any mistakes are entirely the fault of the
% new maintainer!
%
% \section{The BibTeX style files}
%
% The BibTeX style files implement the bibliographic style specified
% by the ACS in \emph{The ACS Style Guide} \cite{Coghill2006},
% on the ACS website \cite{ACS2007}, and in current ACS publications.
% Some of this information can be contradictory, and \emph{The ACS
% Style Guide} sometimes gives more than one option as a model.
% In order to resolve cases where several possibilities are available
% current editions of the \emph{Journal of the American Chemical
% Society} have been consulted; the current consensus there has been
% taken as the correct approach.  In addition to the problem
% of picking the correct style, some of the BibTeX record types are
% difficult to match to standard references in ACS journals.  The
% ``best guess'' has been taken with these.
%
% \subsection{Additional record types}
%
% In general, the database record types supported here follow those
% in the standard BibTeX style files.  Four additional record types
% are provided:
% \begin{description}
%   \item[|patent|] A patent: formatting is similar to other record
%       types.  The data entry for this record type follows the
%       pattern used in |rsc.bst|: |journal| is used to hold
%       the patent type (\emph{e.g.}~``U.S.~Patent''), with the
%       patent number given in |pages|. Whilst this format is
%       non-standard, it is relatively easy to use and implement!
%   \item[|submitted|] Articles submitted to journals but not
%       yet accepted: appends ``submitted'' in a suitable fashion
%       to the entry.
%   \item[|inpress|] Articles in press: appends ``in press'' or,
%       if available, the DOI number assigned to the article.
%   \item[|remark|] A note with no other information to be
%       included.  Output consists purely of the |note| field.
% \end{description}
%
% \subsection{BibTeX database entry requirements}
%
% The requirements for entries in the BibTeX database are slightly
% different using |achemso.bst| to the standard style files. This
% is mainly because some fields are not cited in
% ACS bibliographies.  In particular, journal articles do not
% require a title (the |title| field is ignored).  Articles
% in books and ``collections'' only need the title of the book.
% If a chapter title is given for an |incollection| record, it will
% be printed, but not in the case of an |inbook| record.
%
% \subsection{References to software}
%
% Referencing software is always a little difficult.  The style files
% provided here follow the normal LaTeX convention of using the
% |manual| record type to cite software.  The only requirement is a
% |title|, but fields such as |organization| may be used for more
% detail.  The |edition| field is used to format the software version
% correctly: this will automatically be prefixed with ``version'' by
% the style file.
%
% \subsection{The \texttt{annotate} field}
%
% The standard BibTeX styles use the |note| field for notes to be
% added to the citation.  However, it is common to want personal
% notes about references.  This is catered for using the |annotate|
% field.  The style |achemso| ignores the |annotate| field, whilst
% the |achemsol| style appends the |annotate| information to the
% bibliographic output.  Thus |achemsol| is intended for use in
% database maintenance, whilst |achemso| is for production
% bibliographies.
%
% \DescribeMacro{\refin}
% For use in the |annotate| field the macro \cmd{\refin}
% is defined in |achemso.bst| and |achemsol.bst|.
% The command takes a single argument \marg{text}, and
% gives the output \textbf{Referenced in: text}.
% This command takes one argument (normally text) which is
% preceded by the text ``\textbf{Referenced in:} \meta{text}''.
% The \cmd{\refin} command is intended for tracking citations
% ``backward'' through the database.  For example, this could be
% used to link citations in a database to the writer's own papers.
%
% \subsection{Predefined journal abbreviations}
%
% A number of journal abbreviations are defined in the |.bst| files.
% The abbreviations cover a number ACS journals, several other
% physical chemistry publications and other journals listed as
% highly cited by \emph{Chem.\ Abs.}\ The interested user should
% consult the |.bst| files for full details.
%
% \subsection{\texttt{natbib} support}
%
% As of version 2.2, a |natbib| compatible style file, |achemnat| is
% provided.  The style file provides the appropriate option,
% |natbib|, to load this BibTeX file along with |natbib|, setting up
% the appropriate options.
%
% \section{The LaTeX Package}
%
% The current version of |achemso.sty| is a complete
% re-implementation of the functionality of the original file,
% designed to ensure greater compatibility with other packages. The
% only change for the user is that the bibliography section does
% \emph{not} start a new page when using the |article| document
% class. However, the package now supports all of the standard
% classes, and so the |report| class may be used to ensure a new
% page is started.
%
% \DescribeMacro{\bibliographystyle}
% Loading the |achemso| package adds the appropriate
% \cmd{\bibliographystyle} command to the LaTeX source.  As a result,
% subsequent \cmd{\bibliographystyle} statements will be ignored:
% a suitable warning is given.  The format of citations is altered
% (using the |cite| or |natbib| package as appropriate), and the
% package ensures that the bibliography will be named ``References''
%  in all standard document types.\footnote{This only works if the
% \texttt{babel} package is \emph{not} loaded.  Users wanting a
% system which works with \texttt{babel} should look at the
% \texttt{chemstyle package}. }
%
%  The |achemso| package has five options:
% \begin{description}
%   \item[|note|] If the bibliography contains notes as well
%       as citations, then the section heading should be
%       ``References and Notes''.  This is altered by the
%       |note| package option.
%   \item[|number|] This option numbers the bibliography
%       section (using the |tocbibind| package), and causes it to
%       be entered in the Table of Contents.
%   \item[|list|] This option is intended for creating a listing
%       of the entire BibTeX database.  The BibTeX style is
%       changed to |achemsol|, which will output the additional
%       database field |annotate|, intended for personal notes
%       about a particular database entry.  It also adds the
%       BibTeX key for each citation as a marginal note to the
%       output, using the |showkeys| package.
%    \item[|notsuper|] Switches from superscript citations
%       (\emph{e.g.}~Author \emph{et al.}$^3$) to
%        in-text ones in italics (\emph{e.g.}~Author
%        \emph{et al.}~(\emph{3})). There is a |super|
%        option for completeness, which simply gives the default
%        behaviour.
%     \item[|natbib|] Uses |natbib| rather than |cite| for citation
%        formatting; this also loads the |achemnat| style in place
%        of |achemso|.
% \end{description}
%
% \StopEventually{\bibliography{achemso}}
%
% \section{The Package Code}
%
%  The package code is not very complicated.  For the
%  interested reader(s), it is presented here.
%
% The usual setup code is executed.
%    \begin{macrocode}
%<*package>
\NeedsTeXFormat{LaTeX2e}
\ProvidesPackage{achemso}
  [2007/10/16 v2.2d LaTeX and BibTeX support for American
     Chemical Society publications]
%    \end{macrocode}
% \begin{macro}{\ACS@sctnnmbr}
% \begin{macro}{\ACS@lst}
% \begin{macro}{\ACS@note}
% \changes{v2.0}{2007/01/17}{Boolean values made internal to package}
% \begin{macro}{\ACS@super}
% \changes{v2.1}{2007/02/15}{New Boolean for citation control}
% \begin{macro}{\ACS@natbib}
% \changes{v2.2a}{2007/07/08}{New Boolean for |natbib| support}
% Boolean values are used to handle the options.
%    \begin{macrocode}
\newif \ifACS@sctnnmbr \ACS@sctnnmbrfalse
\newif \ifACS@list     \ACS@listfalse
\newif \ifACS@note     \ACS@notefalse
\newif \ifACS@super    \ACS@supertrue
\newif \ifACS@natbib   \ACS@natbibfalse
%    \end{macrocode}
% \end{macro}
% \end{macro}
% \end{macro}
% \end{macro}
% \end{macro}
% The options are processed.
%\changes{v2.2d}{2007/10/16}{Added \texttt{notes} option}
%    \begin{macrocode}
\DeclareOption{note}{\ExecuteOptions{notes}}
\DeclareOption{notes}{\ACS@notetrue}
\DeclareOption{number}{\ACS@sctnnmbrtrue}
\DeclareOption{super}{\ACS@supertrue}
\DeclareOption{list}{\ACS@listtrue}
\DeclareOption{notsuper}{\ACS@superfalse}
\DeclareOption{natbib}{\ACS@natbibtrue}
\DeclareOption*{\OptionNotUsed}
\ProcessOptions
%    \end{macrocode}
% \changes{v2.1}{2007/02/15}{|cite| package is loaded with different
% options depending on citation style requested}
% \changes{v2.2a}{2007/07/08}{|natbib| support added}
% The |cite| package is loaded to sort and compress references
% correctly. Depending upon the package option given, citations are
% either superscript or italic and in parentheses.
%    \begin{macrocode}
\ifACS@natbib
  \ifACS@super
    \RequirePackage[numbers,sort&compress,super]{natbib}
  \else
%    \end{macrocode}
% For in-line citations with |natbib|, we have to do a
% bit of work to get things to look right.  |natbib| uses
% \cmd{\citenumfont} to format the numbers, but it is not defined
% by default, so we have to use \cmd{\newcommand}.
%    \begin{macrocode}
    \RequirePackage[numbers,sort&compress,round]{natbib}
    \newcommand*{\citenumfont}{\textit}
  \fi
\else
  \ifACS@super
%    \end{macrocode}
%\changes{v2.2c}{2007/08/22}{Use the \texttt{overcite} alias for
%  \texttt{cite} as ACS have very old LaTeX system}
%    \begin{macrocode}
    \RequirePackage[nospace]{overcite}
  \else
%    \end{macrocode}
% Again in-line citations need some format changes.  In the case of
% |cite|, everything is defined initially.  Thus we can use
% \cmd{\renewcommand} for everything.
%    \begin{macrocode}
    \RequirePackage{cite}
    \renewcommand\citeleft{(}
    \renewcommand\citeright{)}
    \renewcommand\citeform[1]{\emph{#1}}
  \fi
\fi
%    \end{macrocode}
% If the |babel| package is loaded, the |note| option does not
% work.  So it is disabled here with a suitable warning.
%    \begin{macrocode}
\@ifpackageloaded{babel}
  {\ACS@notefalse\PackageWarning{achemso}%
    {babel package loaded - note option disabled}}
  {\relax}
%    \end{macrocode}
% \begin{macro}{\ACS@biberror}
% The function \cmd{\ACS@biberror} is defined here to provide an
% easy way of generating a warning if there is no name for a
% bibliography section.  This will only happen with non-standard
% class files.
%     \begin{macrocode}
\def\ACS@biberror{\PackageError{achemso}%
  {No bibliography name command defined}\@eha}
%    \end{macrocode}
% \end{macro}
% \begin{macro}{\refname}
% \begin{macro}{\bibname}
% The |note| option renames the references section to
% ``References and Notes''.  This applies for all standard
% document classes.
% The term ``Bibliography'' is not used in chemistry, the value of
% \cmd{\bibname} is redefined here in all cases where it exists.
%     \begin{macrocode}
\@ifundefined{refname}{%
  \@ifundefined{bibname}{%
    \ACS@biberror
  }{%
    \ifACS@note
      \renewcommand*{\bibname}{References and Notes}
    \else
      \renewcommand*{\bibname}{References}
    \fi
  }
}{%
  \ifACS@note
    \renewcommand*{\refname}{References and Notes}
  \fi
}
%    \end{macrocode}
% \end{macro}
% \end{macro}
% If the |number| option is set, the |tocbibind| package is
% used to number the bibliography.
% \changes{v2.0}{2007/01/17}{Switched to using \texttt{tocbibind}
% to produce number bibliography}
%    \begin{macrocode}
\ifACS@sctnnmbr
  \RequirePackage[numbib]{tocbibind}
\fi
%    \end{macrocode}
% \begin{macro}{\bibliographystyle}
% Depending on the package option, the bibliography style
% will either be |achemso| or |achemsol|.  The later is intended
% for listing the entire database.  The |list| option of the
% package selects this, and for listing also generates boxed
% labels for each reference.  The |showkeys| package provides
% this functionality.  If |natbib| is asked for, then the appropriate
% style files are used in place of the standard ones.
% \changes{v2.0}{2007/01/17}{Replaced custom code with
% \texttt{showkeys} package}
%    \begin{macrocode}
\ifACS@list
  \ifACS@natbib
    \bibliographystyle{achemlnt}
  \else
    \bibliographystyle{achemsol}
  \fi
  \RequirePackage[notcite]{showkeys}
\else
  \ifACS@natbib
    \bibliographystyle{achemnat}
  \else
    \bibliographystyle{achemso}
  \fi
\fi
%    \end{macrocode}
% \end{macro}
% \begin{macro}{\@biblabel}
% In order to re-format the bibliography labels, the easiest
% method is to redefine the \cmd{\@biblabel} macro from the LaTeX
% kernel.
%    \begin{macrocode}
\def\@biblabel#1{#1.}
%    \end{macrocode}
% \end{macro}
% \begin{macro}{\ACS@bibwarning}
% \begin{macro}{\bibliographystyle}
% To ensure that additional \cmd{\bibliographystyle} commands in the
% source are killed off.  The \cmd{\ACS@bibwarning} provides a clean
% method of generating the warning message.
% \changes{v2.0}{2007-01-17}{Command ignored in document body}
%    \begin{macrocode}
\def\ACS@bibwarning{\PackageWarning{achemso}%
  {Additional bibliographystyle command ignored}}
\def\bibliographystyle{\ACS@bibwarning\@gobble}
%    \end{macrocode}
% \end{macro}
% \end{macro}
% The package is complete.
%    \begin{macrocode}
%</package>
%    \end{macrocode}
% \PrintChanges
% \PrintIndex
% \Finale
% \iffalse
%<*bib>
ENTRY
  { address
%<list>    annotate
    author
    booktitle
    chapter
    doi
    edition
    editor
    howpublished
    institution
    journal
%<nat>    key
    note
    number
    organization
    pages
    publisher
    school
    series
    title
    type
    url
    volume
    year
  }
  {}
  {
  label
%<nat>    extra.label
%<nat>    short.list
  }

INTEGERS { output.state before.all mid.sentence after.sentence }
INTEGERS { after.block after.item author.or.editor }
INTEGERS { separate.by.semicolon }

FUNCTION {init.state.consts}
{ #0 'before.all :=
  #1 'mid.sentence :=
  #2 'after.sentence :=
  #3 'after.block :=
  #4 'after.item :=
}

FUNCTION {add.comma}
{ ", " * }

FUNCTION {add.semicolon}
{ "; " * }

%    \end{macrocode}
% For authors/editors we need to be able to add either a semi-colon
% or a comma.  This is done using a switching function, defined here.
%    \begin{macrocode}

FUNCTION {add.comma.or.semicolon}
{ #1 separate.by.semicolon =
    'add.semicolon
    'add.comma
  if$
}

FUNCTION {add.colon}
{ ": " * }

STRINGS { s t }

FUNCTION {output.nonnull}
{ 's :=
  output.state mid.sentence =
    { add.comma write$ }
    { output.state after.block =
      { add.semicolon write$
        newline$
        "\newblock " write$
      }
      { output.state before.all =
          'write$
          { output.state after.item =
            { " " * write$ }
            { add.period$ " " * write$ }
          if$
          }
        if$
        }
      if$
      mid.sentence 'output.state :=
    }
  if$
  s
}

FUNCTION {output}
{ duplicate$ empty$
    'pop$
    'output.nonnull
  if$
}

FUNCTION {output.check}
{ 't :=
  duplicate$ empty$
    { pop$ "Empty " t * " in " * cite$ * warning$ }
    'output.nonnull
  if$
}

%    \end{macrocode}
% For the standard file types, |output.bibitem| can come here.
% The same is not true for styles supporting |natbib|, and so
% |output.bibitem| occurs later for those styles.
% \iffalse
%<*!nat>
% \fi
%    \begin{macrocode}
FUNCTION {output.bibitem}
{ newline$
  "\bibitem{" write$
  cite$ write$
  "}" write$
  newline$
  ""
  before.all 'output.state :=
}

%    \end{macrocode}
% \iffalse
%</!nat>
% \fi
%    \begin{macrocode}
FUNCTION {new.block}
{ output.state before.all =
    'skip$
    { after.block 'output.state := }
  if$
}

FUNCTION {new.sentence}
{ output.state after.block =
    'skip$
    { output.state before.all =
        'skip$
        { after.sentence 'output.state := }
      if$
    }
  if$
}
%<*list>

FUNCTION {add.note}
{ annotate empty$
    'skip$
    { new.block
      "{\footnotesize " annotate * "}" * output }
  if$
}
%</list>

FUNCTION {fin.entry}
%<list>{ add.note
%<list>  add.period$
%<!list>{ add.period$
  write$
  newline$
}
FUNCTION {not}
{   { #0 }
    { #1 }
  if$
}

FUNCTION {and}
{   'skip$
    { pop$ #0 }
  if$
}

FUNCTION {or}
{   { pop$ #1 }
    'skip$
  if$
}

FUNCTION {field.or.null}
{ duplicate$ empty$
    { pop$ "" }
    'skip$
  if$
}

FUNCTION {emphasize}
{ duplicate$ empty$
    { pop$ "" }
    { "\emph{" swap$ * "}" * }
  if$
}

FUNCTION {boldface}
{ duplicate$ empty$
    { pop$ "" }
    { "\textbf{" swap$ * "}" * }
  if$
}

FUNCTION {paren}
{ duplicate$ empty$
    { pop$ "" }
    { "(" swap$ * ")" * }
  if$
}

FUNCTION {bbl.and}
{ "and" }

FUNCTION {bbl.chapter}
{ "Chapter" }

FUNCTION {bbl.editor}
{ "Ed." }

FUNCTION {bbl.editors}
{ "Eds." }

FUNCTION {bbl.edition}
{ "ed." }

FUNCTION {bbl.etal}
{ "et~al." }

FUNCTION {bbl.in}
{ "In" }

FUNCTION {bbl.inpress}
{ "in press" }

FUNCTION {bbl.msc}
{ "M.Sc.\ thesis" }

FUNCTION {bbl.page}
{ "p" }

FUNCTION {bbl.pages}
{ "pp" }

FUNCTION {bbl.phd}
{ "Ph.D.\ thesis" }

FUNCTION {bbl.submitted}
{ "submitted for publication" }

FUNCTION {bbl.techreport}
{ "Technical Report" }

FUNCTION {bbl.version}
{ "version" }

FUNCTION {bbl.volume}
{ "Vol." }

FUNCTION {bbl.first}
{ "1st" }

FUNCTION {bbl.second}
{ "2nd" }

FUNCTION {bbl.third}
{ "3rd" }

FUNCTION {bbl.fourth}
{ "4th" }

FUNCTION {bbl.fifth}
{ "5th" }

FUNCTION {bbl.st}
{ "st" }

FUNCTION {bbl.nd}
{ "nd" }

FUNCTION {bbl.rd}
{ "rd" }

FUNCTION {bbl.th}
{ "th" }

FUNCTION {eng.ord}
{ duplicate$ "1" swap$ *
  #-2 #1 substring$ "1" =
     { bbl.th * }
     { duplicate$ #-1 #1 substring$
       duplicate$ "1" =
         { pop$ bbl.st * }
         { duplicate$ "2" =
             { pop$ bbl.nd * }
             { "3" =
                 { bbl.rd * }
                 { bbl.th * }
               if$
             }
           if$
          }
       if$
     }
   if$
}

FUNCTION{is.a.digit}
{ duplicate$ "" =
    {pop$ #0}
    {chr.to.int$ #48 - duplicate$
     #0 < swap$ #9 > or not}
  if$
}

FUNCTION{is.a.number}
{
  { duplicate$ #1 #1 substring$ is.a.digit }
    {#2 global.max$ substring$}
  while$
  "" =
}

FUNCTION {extract.num}
{ duplicate$ 't :=
  "" 's :=
  { t empty$ not }
  { t #1 #1 substring$
    t #2 global.max$ substring$ 't :=
    duplicate$ is.a.number
      { s swap$ * 's := }
      { pop$ "" 't := }
    if$
  }
  while$
  s empty$
    'skip$
    { pop$ s }
  if$
}

FUNCTION {bibinfo.check}
{ swap$
  duplicate$ missing$
    { pop$ pop$
      ""
    }
    { duplicate$ empty$
        {
          swap$ pop$
        }
        { swap$
          pop$
        }
      if$
    }
  if$
}

FUNCTION {convert.edition}
{ extract.num "l" change.case$ 's :=
  s "first" = s "1" = or
    { bbl.first 't := }
    { s "second" = s "2" = or
        { bbl.second 't := }
        { s "third" = s "3" = or
            { bbl.third 't := }
            { s "fourth" = s "4" = or
                { bbl.fourth 't := }
                { s "fifth" = s "5" = or
                    { bbl.fifth 't := }
                    { s #1 #1 substring$ is.a.number
                        { s eng.ord 't := }
                        { edition 't := }
                      if$
                    }
                  if$
                }
              if$
            }
          if$
        }
      if$
    }
  if$
  t
}

FUNCTION {tie.or.space.connect}
{ duplicate$ text.length$ #3 <
    { "~" }
    { " " }
  if$
  swap$ * *
}

FUNCTION {space.connect}
{ " " swap$ * * }

INTEGERS { nameptr namesleft numnames }

FUNCTION {format.names}
{ 's :=
  #1 'nameptr :=
  s num.names$ 'numnames :=
  numnames 'namesleft :=
  numnames #15 >
    { s #1 "{vv~}{ll,}{~f.}{,~jj}" format.name$ 't :=
      t bbl.etal space.connect
    }
    {
       { namesleft #0 > }
       { s nameptr "{vv~}{ll,}{~f.}{,~jj}" format.name$ 't :=
           nameptr #1 >
             { namesleft #1 >
               { add.comma.or.semicolon t * }
               { numnames #2 >
                 { "" * }
                 'skip$
               if$
               t "others," =
                 { bbl.etal space.connect }
                 { add.comma.or.semicolon t * }
               if$
               }
             if$
             }
           't
         if$
         nameptr #1 + 'nameptr :=
         namesleft #1 - 'namesleft :=
         }
     while$
  }
  if$
}

FUNCTION {format.authors}
{ author empty$
    { "" }
    { #1 'author.or.editor :=
      #1 'separate.by.semicolon :=
      author format.names
    }
  if$
}

FUNCTION {format.editors}
{ editor empty$
    { "" }
    { #2 'author.or.editor :=
      #0 'separate.by.semicolon :=
      editor format.names
      add.comma
      editor num.names$ #1 >
        { bbl.editors }
        { bbl.editor }
      if$
      *
    }
  if$
}

FUNCTION {n.separate.multi}
{ 't :=
  ""
  #0 'numnames :=
  t text.length$ #4 > t is.a.number and
    {
      { t empty$ not }
      { t #-1 #1 substring$ is.a.number
          { numnames #1 + 'numnames := }
          { #0 'numnames := }
        if$
        t #-1 #1 substring$ swap$ *
        t #-2 global.max$ substring$ 't :=
        numnames #4 =
          { duplicate$ #1 #1 substring$ swap$
            #2 global.max$ substring$
            "," swap$ * *
            #1 'numnames :=
          }
          'skip$
        if$
      }
      while$
    }
    { t swap$ * }
  if$
}

FUNCTION {format.bvolume}
{ volume empty$
    { "" }
    { bbl.volume volume tie.or.space.connect }
  if$
}

FUNCTION {format.title.noemph}
{ 't :=
  t empty$
    { "" }
    { t }
  if$
}

FUNCTION {format.title}
{ 't :=
  t empty$
    { "" }
    { t emphasize }
  if$
}

FUNCTION {format.number.series}
{ volume empty$
    { number empty$
       { series field.or.null }
       { series empty$
         { "There is a number but no series in " cite$ * warning$ }
         { series number space.connect }
       if$
       }
      if$
    }
    { "" }
  if$
}

FUNCTION {format.url}
{ url empty$
    { "" }
    { new.sentence "\url{" url * "}" * }
  if$
}

% The specialised |output.bibitem| needed for |natbib| support now
% follows, along with the various support macros it needs.
% \iffalse
%<*nat>
% \fi
%    \begin{macrocode}
FUNCTION {format.full.names}
{'s :=
  #1 'nameptr :=
  s num.names$ 'numnames :=
  numnames 'namesleft :=
    { namesleft #0 > }
    { s nameptr
      "{vv~}{ll}" format.name$ 't :=
      nameptr #1 >
        {
          namesleft #1 >
            { ", " * t * }
            {
              numnames #2 >
                { "," * }
                'skip$
              if$
              t "others" =
                { bbl.etal * }
                { bbl.and space.connect t space.connect }
              if$
            }
          if$
        }
        't
      if$
      nameptr #1 + 'nameptr :=
      namesleft #1 - 'namesleft :=
    }
  while$
}

FUNCTION {author.editor.full}
{ author empty$
    { editor empty$
        { "" }
        { editor format.full.names }
      if$
    }
    { author format.full.names }
  if$
}

FUNCTION {author.full}
{ author empty$
    { "" }
    { author format.full.names }
  if$
}

FUNCTION {editor.full}
{ editor empty$
    { "" }
    { editor format.full.names }
  if$
}

FUNCTION {make.full.names}
{ type$ "book" =
  type$ "inbook" =
  or
    'author.editor.full
    { type$ "proceedings" =
        'editor.full
        'author.full
      if$
    }
  if$
}

FUNCTION {output.bibitem} { newline$
  "\bibitem[" write$
  label write$
  ")" make.full.names duplicate$ short.list =
     { pop$ }
     { * }
   if$
  "]{" * write$
  cite$ write$
  "}" write$
  newline$
  ""
  before.all 'output.state :=
}

%    \end{macrocode}
% \iffalse
%</nat>
% \fi
%    \begin{macrocode}
FUNCTION {n.dashify}
{ 't :=
  ""
    { t empty$ not }
    { t #1 #1 substring$ "-" =
    { t #1 #2 substring$ "--" = not
        { "--" *
          t #2 global.max$ substring$ 't :=
        }
        {   { t #1 #1 substring$ "-" = }
        { "-" *
          t #2 global.max$ substring$ 't :=
        }
          while$
        }
      if$
    }
    { t #1 #1 substring$ *
      t #2 global.max$ substring$ 't :=
    }
      if$
    }
  while$
}

FUNCTION {format.date}
{ year empty$
    { "" }
    { year boldface }
  if$
}

FUNCTION {format.bdate}
{ year empty$
    { "There's no year in " cite$ * warning$ }
    'year
  if$
}

FUNCTION {either.or.check}
{ empty$
    'pop$
    { "Can't use both " swap$ * " fields in " * cite$ * warning$ }
  if$
}

FUNCTION {format.edition}
{ edition duplicate$ empty$
    'skip$
    { convert.edition
      bbl.edition bibinfo.check
      " " * bbl.edition *
    }
  if$
}

INTEGERS { multiresult }

FUNCTION {multi.page.check}
{ 't :=
  #0 'multiresult :=
    { multiresult not
      t empty$ not
      and
    }
    { t #1 #1 substring$
      duplicate$ "-" =
      swap$ duplicate$ "," =
      swap$ "+" =
      or or
        { #1 'multiresult := }
        { t #2 global.max$ substring$ 't := }
      if$
    }
  while$
  multiresult
}

FUNCTION {format.pages}
{ pages empty$
    { "" }
    { pages multi.page.check
      { bbl.pages pages n.dashify tie.or.space.connect }
      { bbl.page pages tie.or.space.connect }
    if$
    }
  if$
}


FUNCTION {format.pages.required}
{ pages empty$
    { ""
      "There are no page numbers for " cite$ * warning$
      output
    }
    { pages multi.page.check
      { bbl.pages pages n.dashify tie.or.space.connect }
      { bbl.page pages tie.or.space.connect }
    if$
    }
  if$
}

FUNCTION {format.pages.nopp}
{ pages empty$
    { ""
      "There are no page numbers for " cite$ * warning$
      output
    }
    { pages multi.page.check
      { pages n.dashify space.connect }
      { pages space.connect }
    if$
    }
  if$
}

FUNCTION {format.pages.patent}
{ pages empty$
    { "There is no patent number for " cite$ * warning$ }
    { pages multi.page.check
      { pages n.dashify }
      { pages n.separate.multi }
      if$
    }
  if$
}

FUNCTION {format.vol.pages}
{ volume emphasize field.or.null
  duplicate$ empty$
    { pop$ format.pages.required }
    { add.comma pages n.dashify * }
  if$
}

FUNCTION {format.chapter.pages}
{ chapter empty$
    'format.pages
    { type empty$
    { bbl.chapter }
    { type "l" change.case$ }
      if$
      chapter tie.or.space.connect
      pages empty$
    'skip$
    { add.comma format.pages * }
      if$
    }
  if$
}

FUNCTION {format.title.in}
{ 's :=
  s empty$
    { "" }
    { editor empty$
      { bbl.in s format.title space.connect }
      { bbl.in s format.title space.connect
        add.semicolon format.editors *
      }
    if$
    }
  if$
}

FUNCTION {format.pub.address}
{ publisher empty$
    { "" }
    { address empty$
        { publisher }
        { publisher add.colon address *}
      if$
    }
  if$
}

FUNCTION {format.school.address}
{ school empty$
    { "" }
    { address empty$
        { school }
        { school add.colon address *}
      if$
    }
  if$
}

FUNCTION {format.organization.address}
{ organization empty$
    { "" }
    { address empty$
        { organization }
        { organization add.colon address *}
      if$
    }
  if$
}

FUNCTION {format.version}
{ edition empty$
    { "" }
    { bbl.version edition tie.or.space.connect }
  if$
}

FUNCTION {empty.misc.check}
{ author empty$ title empty$ howpublished empty$
  year empty$ note empty$ url empty$
  and and and and and
    { "all relevant fields are empty in " cite$ * warning$ }
    'skip$
  if$
}

FUNCTION {empty.doi.note}
{ doi empty$ note empty$ and
    { "Need either a note or DOI for " cite$ * warning$ }
    'skip$
  if$
}

FUNCTION {format.thesis.type}
{ type empty$
    'skip$
    { pop$
      type emphasize
    }
  if$
}

FUNCTION {article}
{ output.bibitem
  format.authors "author" output.check
  after.item 'output.state :=
  journal emphasize "journal" output.check
  after.item 'output.state :=
  format.date "year" output.check
  volume empty$
    { ""
      format.pages.nopp output
    }
    { format.vol.pages output }
  if$
  note output
  fin.entry
}

FUNCTION {book}
{ output.bibitem
  author empty$
    { booktitle empty$
        { title format.title "title" output.check }
        { booktitle format.title "booktitle" output.check }
      if$
      format.edition output
      new.block
      editor empty$
        { "Need either an author or editor for " cite$ * warning$ }
        { "" format.editors * "editor" output.check }
      if$
    }
    { format.authors output
      after.item 'output.state :=
      "author and editor" editor either.or.check
      booktitle empty$
        { title format.title "title" output.check }
        { booktitle format.title "booktitle" output.check }
      if$
      format.edition output
    }
  if$
  new.block
  format.number.series output
  new.block
  format.pub.address "publisher" output.check
  format.bdate "year" output.check
  new.block
  format.bvolume output
  pages empty$
    'skip$
    { format.pages output }
  if$
  note output
  fin.entry
}

FUNCTION {booklet}
{ output.bibitem
  format.authors output
  after.item 'output.state :=
  title format.title "title" output.check
  howpublished output
  address output
  format.date output
  note output
  fin.entry
}

FUNCTION {inbook}
{ output.bibitem
  author empty$
    { title format.title "title" output.check
      format.edition output
      new.block
      editor empty$
      { "Need at least an author or an editor for " cite$ * warning$ }
      { "" format.editors * "editor" output.check }
    if$
    }
    { format.authors output
      after.item 'output.state :=
      title format.title.in "title" output.check
      format.edition output
    }
  if$
  new.block
  format.number.series output
  new.block
  format.pub.address "publisher" output.check
  format.bdate "year" output.check
  new.block
  format.bvolume output
  format.chapter.pages "chapter and pages" output.check
  note output
  fin.entry
}

FUNCTION {incollection}
{ output.bibitem
  author empty$
    { booktitle format.title "booktitle" output.check
      format.edition output
      new.block
      editor empty$
        { "Need at least an author or an editor for " cite$ * warning$ }
        { "" format.editors * "editor" output.check }
      if$
    }
    { format.authors output
      after.item 'output.state :=
      title empty$
        'skip$
        { title format.title.noemph output }
      if$
      after.sentence 'output.state :=
      booktitle format.title.in "booktitle" output.check
      format.edition output
    }
  if$
  new.block
  format.number.series output
  new.block
  format.pub.address "publisher" output.check
  format.bdate "year" output.check
  new.block
  format.bvolume output
  format.chapter.pages "chapter and pages" output.check
  note output
  fin.entry
}

FUNCTION {inpress}
{ output.bibitem
  format.authors "author" output.check
  after.item 'output.state :=
  journal emphasize "journal" output.check
  doi empty$
    {  bbl.inpress output }
    {  after.item 'output.state :=
       format.date output
       "DOI:" doi tie.or.space.connect output
    }
  if$
  note output
  fin.entry
}

FUNCTION {inproceedings}
{ output.bibitem
  format.authors "author" output.check
  after.item 'output.state :=
  title empty$
    'skip$
    { title format.title.noemph output
      after.sentence 'output.state :=
    }
  if$
  booktitle format.title output
  address output
  format.bdate "year" output.check
  pages empty$
    'skip$
    { new.block
      format.pages output }
  if$
  note output
  fin.entry
}

FUNCTION {manual}
{ output.bibitem
  format.authors output
  after.item 'output.state :=
  title format.title "title" output.check
  format.version output
  new.block
  format.organization.address output
  format.bdate output
  note output
  fin.entry
}

FUNCTION {mastersthesis}
{ output.bibitem
  format.authors "author" output.check
  after.item 'output.state :=
  bbl.msc format.thesis.type output
  format.school.address "school" output.check
  format.bdate "year" output.check
  note output
  fin.entry
}

FUNCTION {misc}
{ output.bibitem
  format.authors output
  after.item 'output.state :=
  title empty$
    'skip$
    { title format.title output }
  if$
  howpublished output
  year output
  format.url output
  note output
  fin.entry
  empty.misc.check
}

FUNCTION {patent}
{ output.bibitem
  format.authors "author" output.check
  after.item 'output.state :=
  journal "journal" output.check
  after.item 'output.state :=
  format.pages.patent "pages" output.check
  format.bdate "year" output.check
  note output
  fin.entry
}

FUNCTION {phdthesis}
{ output.bibitem
  format.authors "author" output.check
  after.item 'output.state :=
  bbl.phd format.thesis.type output
  format.school.address "school" output.check
  format.bdate "year" output.check
  note output
  fin.entry
}

FUNCTION {proceedings}
{ output.bibitem
  title format.title.noemph "title" output.check
  address output
  format.bdate "year" output.check
  pages empty$
    'skip$
    { new.block
      format.pages output }
  if$
  note output
  fin.entry
}

FUNCTION {remark}
{ output.bibitem
  note "note" output.check
  fin.entry
}

FUNCTION {submitted}
{ output.bibitem
  format.authors "author" output.check
  bbl.submitted output
  fin.entry
}

FUNCTION {techreport}
{ output.bibitem
  format.authors "author" output.check
  after.item 'output.state :=
  title format.title "title" output.check
  new.block
  type empty$
    'bbl.techreport
    'type
  if$
  number empty$
    'skip$
    { number tie.or.space.connect }
  if$
  output
  format.pub.address output
  format.bdate "year" output.check
  pages empty$
    'skip$
    { new.block
      format.pages output }
  if$
  note output
  fin.entry
}

FUNCTION {unpublished}
{ output.bibitem
  format.authors "author" output.check
  after.item 'output.state :=
  journal empty$
    'skip$
    { journal emphasize "journal" output.check }
  if$
  doi empty$
    {  note output }
    {  after.item 'output.state :=
       format.date output
       "DOI:" doi tie.or.space.connect output
    }
  if$
  fin.entry
  empty.doi.note
}

FUNCTION {conference} {inproceedings}

FUNCTION {other} {patent}

FUNCTION {default.type} {misc}

MACRO {jan} {"Jan."}
MACRO {feb} {"Feb."}
MACRO {mar} {"Mar."}
MACRO {apr} {"Apr."}
MACRO {may} {"May"}
MACRO {jun} {"June"}
MACRO {jul} {"July"}
MACRO {aug} {"Aug."}
MACRO {sep} {"Sept."}
MACRO {oct} {"Oct."}
MACRO {nov} {"Nov."}
MACRO {dec} {"Dec."}

MACRO {acchemr} {"Acc.\ Chem.\ Res."}
MACRO {aacsa} {"Adv.\ {ACS} Abstr."}
MACRO {anchem} {"Anal.\ Chem."}
MACRO {bioch} {"Biochemistry"}
MACRO {bicoc} {"Bioconj.\ Chem."}  % ***
MACRO {bitech} {"Biotechnol.\ Progr."}  % ***
MACRO {chemeng} {"Chem.\ Eng.\ News"}
MACRO {chs} {"Chem.\ Health Safety"} % ***
MACRO {crt} {"Chem.\ Res.\ Toxicol."} % ***
MACRO {chemrev} {"Chem.\ Rev."} % ***
MACRO {cmat} {"Chem.\ Mater."} % ***
MACRO {chemtech} {"{CHEMTECH}"} % ***
MACRO {enfu} {"Energy Fuels"} % ***
MACRO {envst} {"Environ.\ Sci.\ Technol."}
MACRO {iecf} {"Ind.\ Eng.\ Chem.\ Fundam."}
MACRO {iecpdd} {"Ind.\ Eng.\ Chem.\ Proc.\ Des.\ Dev."}
MACRO {iecprd} {"Ind.\ Eng.\ Chem.\ Prod.\ Res.\ Dev."}
MACRO {iecr} {"Ind.\ Eng.\ Chem.\ Res."} % ***
MACRO {inor} {"Inorg.\ Chem."}
MACRO {jafc} {"J.~Agric.\ Food Chem."}
MACRO {jacs} {"J.~Am.\ Chem.\ Soc."}
MACRO {jced} {"J.~Chem.\ Eng.\ Data"}
MACRO {jcics} {"J.~Chem.\ Inf.\ Comput.\ Sci."}
MACRO {jmc} {"J.~Med.\ Chem."}
MACRO {joc} {"J.~Org.\ Chem."}
MACRO {jps} {"J.~Pharm.\ Sci."}
MACRO {jpcrd} {"J.~Phys.\ Chem.\ Ref.\ Data"} % ***
MACRO {jpc} {"J.~Phys.\ Chem."}
MACRO {jpca} {"J.~Phys.\ Chem.~A"}
MACRO {jpcb} {"J.~Phys.\ Chem.~B"}
MACRO {lang} {"Langmuir"}
MACRO {macro} {"Macromolecules"}
MACRO {orgmet} {"Organometallics"}
MACRO {orglett} {"Org.\ Lett."}

MACRO {jft} {"J.~Chem.\ Soc., Faraday Trans."}
MACRO {jft1} {"J.~Chem.\ Soc., Faraday Trans.\ 1"}
MACRO {jft2} {"J.~Chem.\ Soc., Faraday Trans.\ 2"}
MACRO {tfs} {"Trans.\ Faraday Soc."}
MACRO {jcis} {"J.~Colloid Interface Sci."}
MACRO {acis} {"Adv.~Colloid Interface Sci."}
MACRO {cs} {"Colloids Surf."}
MACRO {csa} {"Colloids Surf.\ A"}
MACRO {csb} {"Colloids Surf.\ B"}
MACRO {pcps} {"Progr.\ Colloid Polym.\ Sci."}
MACRO {jmr} {"J.~Magn.\ Reson."}
MACRO {jmra} {"J.~Magn.\ Reson.\ A"}
MACRO {jmrb} {"J.~Magn.\ Reson.\ B"}
MACRO {sci} {"Science"}
MACRO {nat} {"Nature"}
MACRO {jcch} {"J.~Comput.\ Chem."}
MACRO {cca} {"Croat.\ Chem.\ Acta"}
MACRO {angew} {"Angew.\ Chem., Int.\ Ed."}
MACRO {chemeurj} {"Chem.---Eur.\ J."}

MACRO {poly} {"Polymer"}
MACRO {ajp} {"Am.\ J.\ Phys."}
MACRO {rsi} {"Rev.\ Sci.\ Instrum."}
MACRO {jcp} {"J.~Chem.\ Phys."}
MACRO {cpl} {"Chem.\ Phys.\ Lett."}
MACRO {molph} {"Mol.\ Phys."}
MACRO {pac} {"Pure Appl.\ Chem."}
MACRO {jbc} {"J.~Biol.\ Chem."}
MACRO {tl} {"Tetrahedron Lett."}
MACRO {psisoe} {"Proc.\ SPIE-Int.\ Soc.\ Opt.\ Eng."}
MACRO {prb} {"Phys.\ Rev.\ B:\ Condens.\ Matter Mater. Phys."}
MACRO {jap} {"J.~Appl.\ Phys."}
MACRO {pnac} {"Proc.\ Natl.\ Acad.\ Sci.\ U.S.A."}
MACRO {bba} {"Biochim.\ Biophys.\ Acta"}
MACRO {nar} {"Nucleic.\ Acid Res."}

READ

%    \end{macrocode}
% The nature of the initialise code depends on whether we need to
% support |natbib|.  First the simple case is handled.
% \iffalse
%<*!nat>
% \fi
%    \begin{macrocode}
STRINGS { longest.label }

INTEGERS { number.label longest.label.width }

FUNCTION {initialize.longest.label}
{ "" 'longest.label :=
  #1 'number.label :=
  #0 'longest.label.width :=
}

FUNCTION {longest.label.pass}
{ number.label int.to.str$ 'label :=
  number.label #1 + 'number.label :=
  label width$ longest.label.width >
    { label 'longest.label :=
      label width$ 'longest.label.width :=
    }
    'skip$
  if$
}

EXECUTE {initialize.longest.label}

ITERATE {longest.label.pass}

%    \end{macrocode}
% \iffalse
%</!nat>
% \fi
% Now the |natbib| system is sorted out, basically by copying from
% |plainnat.bst|.
% \iffalse
%<*nat>
% \fi
%    \begin{macrocode}
INTEGERS { len }

FUNCTION {chop.word}
{ 's :=
  'len :=
  s #1 len substring$ =
    { s len #1 + global.max$ substring$ }
    's
  if$
}

FUNCTION {format.lab.names}
{ 's :=
  s #1 "{vv~}{ll}" format.name$
  s num.names$ duplicate$
  #2 >
    { pop$ bbl.etal space.connect }
    { #2 <
        'skip$
        { s #2 "{ff }{vv }{ll}{ jj}" format.name$ "others" =
            { bbl.etal space.connect }
            { bbl.and space.connect s #2 "{vv~}{ll}" format.name$ space.connect }
          if$
        }
      if$
    }
  if$
}

FUNCTION {author.key.label}
{ author empty$
    { key empty$
        { cite$ #1 #3 substring$ }
        'key
      if$
    }
    { author format.lab.names }
  if$
}

FUNCTION {author.editor.key.label}
{ author empty$
    { editor empty$
        { key empty$
            { cite$ #1 #3 substring$ }
            'key
          if$
        }
        { editor format.lab.names }
      if$
    }
    { author format.lab.names }
  if$
}

FUNCTION {author.key.organization.label}
{ author empty$
    { key empty$
        { organization empty$
            { cite$ #1 #3 substring$ }
            { "The " #4 organization chop.word #3 text.prefix$ }
          if$
        }
        'key
      if$
    }
    { author format.lab.names }
  if$
}

FUNCTION {editor.key.organization.label}
{ editor empty$
    { key empty$
        { organization empty$
            { cite$ #1 #3 substring$ }
            { "The " #4 organization chop.word #3 text.prefix$ }
          if$
        }
        'key
      if$
    }
    { editor format.lab.names }
  if$
}

FUNCTION {calc.short.authors}
{ type$ "book" =
  type$ "inbook" =
  or
    'author.editor.key.label
    { type$ "proceedings" =
        'editor.key.organization.label
        { type$ "manual" =
            'author.key.organization.label
            'author.key.label
          if$
        }
      if$
    }
  if$
  'short.list :=
}

FUNCTION {calc.label}
{ calc.short.authors
  short.list
  "("
  *
  year duplicate$ empty$
  short.list key field.or.null = or
     { pop$ "" }
     'skip$
  if$
  *
  'label :=
}

ITERATE {calc.label}

STRINGS { longest.label last.label next.extra }

INTEGERS { longest.label.width last.extra.num number.label }

FUNCTION {initialize.longest.label}
{ "" 'longest.label :=
  #0 int.to.chr$ 'last.label :=
  "" 'next.extra :=
  #0 'longest.label.width :=
  #0 'last.extra.num :=
  #0 'number.label :=
}

FUNCTION {forward.pass}
{ last.label label =
    { last.extra.num #1 + 'last.extra.num :=
      last.extra.num int.to.chr$ 'extra.label :=
    }
    { "a" chr.to.int$ 'last.extra.num :=
      "" 'extra.label :=
      label 'last.label :=
    }
  if$
  number.label #1 + 'number.label :=
}

EXECUTE {initialize.longest.label}

ITERATE {forward.pass}

%    \end{macrocode}
% \iffalse
%</nat>
% \fi

FUNCTION {begin.bib}
{ preamble$ empty$
    'skip$
    { preamble$ write$ newline$ }
  if$
  "\providecommand{\url}[1]{\texttt{#1}}"
  write$ newline$
  "\providecommand{\refin}[1]{\\ \textbf{Referenced in:} #1}"
  write$ newline$
%<nat>  "\providecommand{\natexlab}[1]{#1}"
%<nat>  write$ newline$
%<!nat>  "\begin{thebibliography}{"  longest.label  * "}" *
%<nat>  "\begin{thebibliography}{" number.label int.to.str$ * "}" *
  write$ newline$
}

EXECUTE {begin.bib}

EXECUTE {init.state.consts}

ITERATE {call.type$}

FUNCTION {end.bib}
{ newline$
  "\end{thebibliography}" write$ newline$
}

EXECUTE {end.bib}
%</bib>
%<*database>
@BOOK{Coghill2006,
  title = {{T}he {ACS} {S}tyle {G}uide},
  publisher = {{O}xford {U}niversity {P}ress, {I}nc. and
               {T}he {A}merican {C}hemical {S}ociety},
  year = {2006},
  editor = {Coghill, Anne M. and Garson, Lorrin R.},
  address = {{N}ew {Y}ork},
  edition = {3},
  subtitle = {{E}ffective {C}ommunication of {S}cientific {I}nformation},
}

@MISC{ACS2007,
  url = {http://pubs.acs.org/books/references.shtml},
}
%</database>
%<*jawltxdoc>
% The following is convenient method for collecting together package
% loading, formatting commands and new macros used to format |dtx|
% files written by the current author.  It is based on the similar
% files provided by Will Robertson in his packages and Heiko Oberdiek
% as a stand-alone package.  Notice that it is not intended for other
% users: there is no error checking!  However, it is covered by the
% LPPL in the same way as the rest of this package.
%
\NeedsTeXFormat{LaTeX2e}
\ProvidesPackage{jawltxdoc}
  [2007/10/14 v1.0b]
% First of all, a number of support packages are loaded.
\usepackage[T1]{fontenc}
\usepackage[english,UKenglish]{babel}
\usepackage[scaled=0.95]{helvet}
\usepackage[version=3]{mhchem}
\usepackage[final]{microtype}
\usepackage[osf]{mathpazo}
\usepackage{booktabs,array,url,graphicx,courier,unitsdef}
\usepackage{upgreek,ifpdf,listings}
% If using PDFLaTeX, the source will be attached to the PDF. This
% is basically the system used by Heiko Oberdiek, but with a check
% that PDF mode is enabled.
\ifpdf
  \usepackage{embedfile}
  \embedfile[%
    stringmethod=escape,%
    mimetype=plain/text,%
    desc={LaTeX docstrip source archive for package `\jobname'}%
    ]{\jobname.dtx}
\fi
\usepackage{\jobname}
\usepackage[numbered]{hypdoc}
%
% To typeset examples, a new environment is needed.  The code below
% is based on that in used by |listings|, but is modified to get
% better formatting for this context.  The formatting of the output
% is basically that in Will Robertson's |dtx-style| file.
\newlength\LaTeXwidth
\newlength\LaTeXoutdent
\newlength\LaTeXgap
\setlength\LaTeXgap{1em}
\setlength\LaTeXoutdent{-0.15\textwidth}
\def\typesetexampleandcode{%
  \begin{list}{}{%
    \setlength\itemindent{0pt}
    \setlength\leftmargin\LaTeXoutdent
    \setlength\rightmargin{0pt}
  }
  \item
    \setlength\LaTeXoutdent{-0.15\textwidth}
    \begin{minipage}[c]{\textwidth-\LaTeXwidth-\LaTeXoutdent-\LaTeXgap}
      \lst@sampleInput
    \end{minipage}%
    \hfill%
    \begin{minipage}[c]{\LaTeXwidth}%
      \hbox to\linewidth{\box\lst@samplebox\hss}%
    \end{minipage}%
  \end{list}
}
\def\typesetcodeandexample{%
  \begin{list}{}{%
    \setlength\itemindent{0pt}
    \setlength\leftmargin{0pt}
    \setlength\rightmargin{0pt}
  }
  \item
    \begin{minipage}[c]{\LaTeXwidth}%
      \hbox to\linewidth{\box\lst@samplebox\hss}%
    \end{minipage}%
    \lst@sampleInput
  \end{list}
}
\def\typesetfloatexample{%
  \begin{list}{}{%
    \setlength\itemindent{0pt}
    \setlength\leftmargin{0pt}
    \setlength\rightmargin{0pt}
  }
  \item
    \lst@sampleInput
    \begin{minipage}[c]{\LaTeXwidth}%
      \hbox to\linewidth{\box\lst@samplebox\hss}%
    \end{minipage}%
  \end{list}
}
\def\typesetcodeonly{%
  \begin{list}{}{%
    \setlength\itemindent{0pt}
    \setlength\leftmargin{0pt}
    \setlength\rightmargin{0pt}
  }
  \item
    \begin{minipage}[c]{\LaTeXwidth}%
      \hbox to\linewidth{\box\lst@samplebox\hss}%
    \end{minipage}%
  \end{list}
}
\edef\LaTeXexamplefile{\jobname.tmp}
\lst@RequireAspects{writefile}
\newbox\lst@samplebox
\lstnewenvironment{LaTeXexample}[1][\typesetexampleandcode]{%
  \let\typesetexample#1
  \global\let\lst@intname\@empty
  \setbox\lst@samplebox=\hbox\bgroup
  \setkeys{lst}{language=[LaTeX]{TeX},tabsize=4,gobble=2,%
    breakindent=0pt,basicstyle=\small\ttfamily,basewidth=0.51em,%
    keywordstyle=\color{blue},%
% Notice that new keywords should be added here. The list is simply
% macro names needed to typeset documentation of the package
% author.
    morekeywords={bibnote,citenote,bibnotetext,bibnotemark,%
      thebibnote,bibnotename,includegraphics,schemeref,%
      floatcontentsleft,floatcontentsright,floatcontentscentre,%
      schemerefmarker,compound,schemerefformat,color,%
      startchemical,stopchemical,chemical,setupchemical,bottext,%
      listofschemes}}
  \lst@BeginAlsoWriteFile{\LaTeXexamplefile}
}{%
  \lst@EndWriteFile\egroup
  \setlength\LaTeXwidth{\wd\lst@samplebox}
  \typesetexample%
}
\def\lst@sampleInput{%
  \MakePercentComment\catcode`\^^M=10\relax
  \small%
  {\setkeys{lst}{SelectCharTable=\lst@ReplaceInput{\^\^I}%
    {\lst@ProcessTabulator}}%
    \leavevmode \input{\LaTeXexamplefile}}%
  \MakePercentIgnore%
}
\hyphenation{PDF-LaTeX}
%</jawltxdoc>
%\fi

%
% Documentation:
%    (a) Without write18 enabled:
%          pdflatex achemso.dtx
%          bibtex8 --wolfgang achemso.aux
%          makeindex -s gind.ist achemso.idx
%          makeindex -s gglo.ist -o achemso.gls  achemso.glo
%          pdflatex achemso.dtx
%          makeindex -s gind.ist achemso.idx
%          makeindex -s gglo.ist -o achemso.gls  achemso.glo
%          pdflatex achemso.dtx
%    (b) With write18 enabled:
%          pdflatex achemso.dtx
%          bibtex8 --wolfgang achemso.aux
%          pdflatex achemso.dtx
%          pdflatex achemso.dtx
%
% Installation:
%     Copy achemso.sty and the achmes*.bst files to a location
%     searched by TeX, and if required by your TeX installation,
%     run the appropriate command to build a hash of files
%     (texhash, mpm --update-db, etc.)
%
% Note:
%     The jawltxdoc.sty file is not needed for installation,
%     only for building the documentation.  It may be deleted.
%
%<*ignore>
% This is all taken verbatim from Heiko Oberdiek's packages
\begingroup
  \def\x{LaTeX2e}%
\expandafter\endgroup
\ifcase 0\ifx\install y1\fi\expandafter
         \ifx\csname processbatchFile\endcsname\relax\else1\fi
         \ifx\fmtname\x\else 1\fi\relax
\else\csname fi\endcsname
%</ignore>
%<*install>
\input docstrip.tex
\keepsilent
\askforoverwritefalse
\preamble
 ----------------------------------------------------------------
 The achemso package - LaTeX and BibTeX support for American
 Chemical Society publications
 Maintained by Joseph Wright
 E-mail: joseph.wright@morningstar2.co.uk
 Released under the LaTeX Project Public License v1.3 or later
 See http://www.latex-project.org/lppl.txt
 ----------------------------------------------------------------

\endpreamble
\Msg{Generating achemso files:}
\usedir{tex/latex/contib/achemso}
\generate{\file{\jobname.ins}{\from{\jobname.dtx}{install}}
          \file{\jobname.sty}{\from{\jobname.dtx}{package}}
          \file{jawltxdoc.sty}{\from{\jobname.dtx}{jawltxdoc}}
}
\declarepostamble\bibtexable
\endpostamble
\usedir{bibtex/bst/achemso}
\generate{\usepostamble\bibtexable
          \file{achemso.bst}{\from{achemso.dtx}{bib}}
          \file{achemnat.bst}{\from{achemso.dtx}{bib,nat}}
          \file{achemsol.bst}{\from{achemso.dtx}{bib,list}}
          \file{achemlnt.bst}{\from{achemso.dtx}{bib,list,nat}}
}
\generate{\usepostamble\empty\usepreamble\empty
          \file{achemso.bib}{\from{achemso.dtx}{database}}
}
\endbatchfile
%</install>
%<*ignore>
\fi
% Will Robertson's trick
\immediate\write18{makeindex -s gind.ist -o \jobname.ind  \jobname.idx}
\immediate\write18{makeindex -s gglo.ist -o \jobname.gls  \jobname.glo}
%</ignore>
%<*driver>
\PassOptionsToClass{a4paper}{article}
\documentclass{ltxdoc}
\EnableCrossrefs
\CodelineIndex
\RecordChanges
%\OnlyDescription
% The various formatting commands used in this file are collected
% together in |jawltxdoc|.
\usepackage{jawltxdoc}
\begin{document}
  \DocInput{\jobname.dtx}
\end{document}
%</driver>
% \fi
%
% \CheckSum{105}
%
% \CharacterTable
%  {Upper-case    \A\B\C\D\E\F\G\H\I\J\K\L\M\N\O\P\Q\R\S\T\U\V\W\X\Y\Z
%   Lower-case    \a\b\c\d\e\f\g\h\i\j\k\l\m\n\o\p\q\r\s\t\u\v\w\x\y\z
%   Digits        \0\1\2\3\4\5\6\7\8\9
%   Exclamation   \!     Double quote  \"     Hash (number) \#
%   Dollar        \$     Percent       \%     Ampersand     \&
%   Acute accent  \'     Left paren    \(     Right paren   \)
%   Asterisk      \*     Plus          \+     Comma         \,
%   Minus         \-     Point         \.     Solidus       \/
%   Colon         \:     Semicolon     \;     Less than     \<
%   Equals        \=     Greater than  \>     Question mark \?
%   Commercial at \@     Left bracket  \[     Backslash     \\
%   Right bracket \]     Circumflex    \^     Underscore    \_
%   Grave accent  \`     Left brace    \{     Vertical bar  \|
%   Right brace   \}     Tilde         \~}
%
%\GetFileInfo{\jobname.sty}
%
%\changes{v1.0}{1998/06/01}{Initial release of package by Mats
%   Dahlgren}
%\changes{v2.0}{2007/01/17}{Re-write of package by Joseph Wright}
%\changes{v2.0}{2007/01/17}{Several improvements to BibTeX style
%  files}
%\changes{v2.0}{2007/01/17}{License changed to LPPL}
%\changes{v2.1}{2007/02/15}{Updated documentation to reflect 3rd
%  edition of ACS Style Guide}
%\changes{v2.1}{2007/02/15}{BibTeX style improved to reflect 3rd
%  edition of ACS Style Guide}
%\changes{v2.2}{2007/06/05}{Added \texttt{natbib} support}
%\changes{v2.2a}{2007/07/08}{Fixed separation of editor names}
%\changes{v2.2a}{2007/07/08}{Bug fixes to \texttt{natbib} and list
% support}
%\changes{v2.2a}{2007/07/08}{\texttt{title} field included in output
% for \texttt{incollection} records}
%\changes{v2.2b}{2007/07/09}{Bug fix to name formatting}
%\changes{v2.2d}{2007/10/16}{Added \textsc{url} field to
%  \texttt{misc} output}
%\changes{v2.2d}{2007/10/16}{Package design improved}
%
%\DoNotIndex{\@biblabel,\@eha,\@gobble,\@ifpackageloaded,\@ifundefined}
%\DoNotIndex{\bibliographystyle,\bibname,\citeform,\citeleft}
%\DoNotIndex{\citenumfont,\citeright,\DeclareOption,\def,\else,\emph}
%\DoNotIndex{\fi,\ifx,\NeedsTeXFormat,\newcommand,\newif}
%\DoNotIndex{\OptionNotUsed,\PackageError,\PackageWarning}
%\DoNotIndex{\ProcessOptions,\ProvidesPackage,\refname,\relax}
%\DoNotIndex{\renewcommand,\RequirePackage,\textit,}
%
% \title{\texttt{achemso} --- LaTeX and BibTeX support for American
%   Chemical Society publications%
%   \thanks{This file describes version \fileversion, last revised
%           \filedate.}}
% \author{Joseph Wright%
%   \thanks{E-mail: joseph.wright@morningstar2.co.uk}}
% \date{Released \filedate}
%
%\maketitle
%
%\begin{abstract}
% The |achemso| package provides a BibTeX style in accordance with
% the requirements of the journals of the American Chemical Society,
% along with a supporting LaTeX package file. Also provided is a
% BibTeX style file to be used for bibliography database listings.
%\end{abstract}
%
% \section{Introduction}
%
% Synthetic chemists do not, in the main, use LaTeX for the
% preparation of journal articles. Some journals, mainly in the
% physical chemistry area, do accept LaTeX submissions.  Given the
% clear advantages of LaTeX over other methods, it would be
% nice to be able to use LaTeX for preparing reports. Thus the need
% for BibTeX styles for chemistry is real. The package |achemso|
% provides for a BibTeX style and other support for articles and
% reports in the style of the American Chemical Society (ACS).
%
% As describe in \emph{The ACS Style Guide} \cite{Coghill2006},
% almost all ACS publications use the same style for the formatting
% of references.  The reproduction of this style is the aim of the
% BibTeX style file provided here.  However, the ACS use different
% citation styles in different publications.  The |achemso| package
% provides support for the two numerical systems: superscript
% and italic in-text citations.  The majority of ACS journals use
% the superscript method (Table \ref{tbl:journals-super}), with a
% smaller number using the italic system (Table
% \ref{tbl:journals-inline}). The journal \emph{Biochemistry} does
% not use the standard ACS style for references, and so is not
% covered by the |achemso| package.
% \begin{table}
%   \centering
%   \small
%   \begin{tabular}{>{\itshape}l>{\itshape}l}
%     \toprule
%     \upshape{Journal Title} & \upshape{\emph{CASSI} Abbreviation} \\
%     \midrule
%     Accounts of Chemical Research & Acc.~Chem.~Res. \\
%     Analytical Chemistry & Anal.~Chem. \\
%     Biomacromolecules & Biomacromolecules \\
%     Chemical Reviews & Chem.~Rev. \\
%     Chemistry of Materials & Chem.~Mater. \\
%     Crystal Growth \& Design & Cryst.~Growth Des. \\
%     Energy \& Fuels & Energy Fuels \\
%     Industrial \& Engineering Chemistry Research & Ind.~Eng.~Chem.~Res. \\
%     Inorganic Chemistry & Inorg.Chem. \\
%     Journal of the American Chemical Society & J.~Am.~Chem.~Soc. \\
%     Journal of Chemical and Engineering Data & J.~Chem.~Eng.~Data \\
%     Journal of Chemical Theory and Computation & J.~Chem.~Theory Comput. \\
%     Journal of Chemical Information and Modeling & J.~Chem.~Inf.~Model. \\
%     Journal of Combinatorial Chemistry & J.~Comb.~Chem. \\
%     Journal of Medicinal Chemistry & J.~Med.~Chem. \\
%     Journal of Natural Products & J.~Nat.~Prod. \\
%     The Journal of Organic Chemistry & J.~Org.~Chem. \\
%     The Journal of Physical Chemistry A & J.~Phys.~Chem.~A \\
%     The Journal of Physical Chemistry B & J.~Phys.~Chem.~B \\
%     The Journal of Physical Chemistry C & J.~Phys.~Chem.~C \\
%     Journal of Proteome Research & J.~Proteome Res. \\
%     Langmuir & Langmuir \\
%     Macromolecules & Macromolecules \\
%     Molecular Pharmaceutics & Mol.~Pharm. \\
%     Nano Letters & Nano Lett. \\
%     Organic Letters & Org.~Lett. \\
%     Organic Process Research \& Design & Org.~Process Res.~Dev. \\
%     Organometallics & Organometallics \\
%     \bottomrule
%   \end{tabular}
%   \caption{Journals using the ACS reference style with superscript citations}
%   \label{tbl:journals-super}
% \end{table}
% \begin{table}
%   \small
%   \centering
%   \begin{tabular}{>{\itshape}l>{\itshape}l}
%     \toprule
%     \upshape{Journal Title} & \upshape{\emph{CASSI} Abbreviation} \\
%     \midrule
%     ACS Chemical Biology & ACS Chem.~Biol. \\
%     Bioconjugate Chemistry & Bioconjugate Chem. \\
%     Biotechnology Progress & Biotechnol.~Prog. \\
%     Chemical Research in Toxicology & Chem.~Res.~Toxicol. \\
%     Environmental Science and Technology & Envirn.~Sci.~Technol. \\
%     Journal of Agricultural and Food Chemistry & J.~Agric.~Food Chem. \\
%     \bottomrule
%   \end{tabular}
%   \caption{Journals using the ACS reference style with in-text citations}
%   \label{tbl:journals-inline}
% \end{table}
%
% This package consists of two BibTeX files (|achemso.bst|
% and |achemsol.bst|) along with a small LaTeX file |achemso.sty|.
% The naming of the package is slightly unusual, but follows from
% the need to pick a unique name.  To quote the documentation to the
% first version:
% \begin{quote}
%   there is already a LaTeX 2.09 and
%   BibTeX style package called |acsarticle| and
%   |acs.bst|, which are not ``ACS'' as in `American Chemical
%   Society' (rather, this package is
%   formatting the output according to the instructions of
%   \emph{Advances in Control Systems}).  Hence, \emph{this}
%   new package had to be given another name.  The name of choice
%   was then |achemso|, which is made from the words
%   ``\emph{A}merican \emph{Chem}ical \emph{So}ciety.''
% \end{quote}
%
% \subsection{Change of maintainer}
%
% This package was initially released by Mats Dahlgren.  He no
% longer has time to devote to LaTeX development.  With his permission,
% the package has therefore been taken over by Joseph
% Wright, the maintainer of the the |rsc| package.  The majority of
% the package has been rebuilt and the BibTeX style file has been
% totally overhauled.  Any mistakes are entirely the fault of the
% new maintainer!
%
% \section{The BibTeX style files}
%
% The BibTeX style files implement the bibliographic style specified
% by the ACS in \emph{The ACS Style Guide} \cite{Coghill2006},
% on the ACS website \cite{ACS2007}, and in current ACS publications.
% Some of this information can be contradictory, and \emph{The ACS
% Style Guide} sometimes gives more than one option as a model.
% In order to resolve cases where several possibilities are available
% current editions of the \emph{Journal of the American Chemical
% Society} have been consulted; the current consensus there has been
% taken as the correct approach.  In addition to the problem
% of picking the correct style, some of the BibTeX record types are
% difficult to match to standard references in ACS journals.  The
% ``best guess'' has been taken with these.
%
% \subsection{Additional record types}
%
% In general, the database record types supported here follow those
% in the standard BibTeX style files.  Four additional record types
% are provided:
% \begin{description}
%   \item[|patent|] A patent: formatting is similar to other record
%       types.  The data entry for this record type follows the
%       pattern used in |rsc.bst|: |journal| is used to hold
%       the patent type (\emph{e.g.}~``U.S.~Patent''), with the
%       patent number given in |pages|. Whilst this format is
%       non-standard, it is relatively easy to use and implement!
%   \item[|submitted|] Articles submitted to journals but not
%       yet accepted: appends ``submitted'' in a suitable fashion
%       to the entry.
%   \item[|inpress|] Articles in press: appends ``in press'' or,
%       if available, the DOI number assigned to the article.
%   \item[|remark|] A note with no other information to be
%       included.  Output consists purely of the |note| field.
% \end{description}
%
% \subsection{BibTeX database entry requirements}
%
% The requirements for entries in the BibTeX database are slightly
% different using |achemso.bst| to the standard style files. This
% is mainly because some fields are not cited in
% ACS bibliographies.  In particular, journal articles do not
% require a title (the |title| field is ignored).  Articles
% in books and ``collections'' only need the title of the book.
% If a chapter title is given for an |incollection| record, it will
% be printed, but not in the case of an |inbook| record.
%
% \subsection{References to software}
%
% Referencing software is always a little difficult.  The style files
% provided here follow the normal LaTeX convention of using the
% |manual| record type to cite software.  The only requirement is a
% |title|, but fields such as |organization| may be used for more
% detail.  The |edition| field is used to format the software version
% correctly: this will automatically be prefixed with ``version'' by
% the style file.
%
% \subsection{The \texttt{annotate} field}
%
% The standard BibTeX styles use the |note| field for notes to be
% added to the citation.  However, it is common to want personal
% notes about references.  This is catered for using the |annotate|
% field.  The style |achemso| ignores the |annotate| field, whilst
% the |achemsol| style appends the |annotate| information to the
% bibliographic output.  Thus |achemsol| is intended for use in
% database maintenance, whilst |achemso| is for production
% bibliographies.
%
% \DescribeMacro{\refin}
% For use in the |annotate| field the macro \cmd{\refin}
% is defined in |achemso.bst| and |achemsol.bst|.
% The command takes a single argument \marg{text}, and
% gives the output \textbf{Referenced in: text}.
% This command takes one argument (normally text) which is
% preceded by the text ``\textbf{Referenced in:} \meta{text}''.
% The \cmd{\refin} command is intended for tracking citations
% ``backward'' through the database.  For example, this could be
% used to link citations in a database to the writer's own papers.
%
% \subsection{Predefined journal abbreviations}
%
% A number of journal abbreviations are defined in the |.bst| files.
% The abbreviations cover a number ACS journals, several other
% physical chemistry publications and other journals listed as
% highly cited by \emph{Chem.\ Abs.}\ The interested user should
% consult the |.bst| files for full details.
%
% \subsection{\texttt{natbib} support}
%
% As of version 2.2, a |natbib| compatible style file, |achemnat| is
% provided.  The style file provides the appropriate option,
% |natbib|, to load this BibTeX file along with |natbib|, setting up
% the appropriate options.
%
% \section{The LaTeX Package}
%
% The current version of |achemso.sty| is a complete
% re-implementation of the functionality of the original file,
% designed to ensure greater compatibility with other packages. The
% only change for the user is that the bibliography section does
% \emph{not} start a new page when using the |article| document
% class. However, the package now supports all of the standard
% classes, and so the |report| class may be used to ensure a new
% page is started.
%
% \DescribeMacro{\bibliographystyle}
% Loading the |achemso| package adds the appropriate
% \cmd{\bibliographystyle} command to the LaTeX source.  As a result,
% subsequent \cmd{\bibliographystyle} statements will be ignored:
% a suitable warning is given.  The format of citations is altered
% (using the |cite| or |natbib| package as appropriate), and the
% package ensures that the bibliography will be named ``References''
%  in all standard document types.\footnote{This only works if the
% \texttt{babel} package is \emph{not} loaded.  Users wanting a
% system which works with \texttt{babel} should look at the
% \texttt{chemstyle package}. }
%
%  The |achemso| package has five options:
% \begin{description}
%   \item[|note|] If the bibliography contains notes as well
%       as citations, then the section heading should be
%       ``References and Notes''.  This is altered by the
%       |note| package option.
%   \item[|number|] This option numbers the bibliography
%       section (using the |tocbibind| package), and causes it to
%       be entered in the Table of Contents.
%   \item[|list|] This option is intended for creating a listing
%       of the entire BibTeX database.  The BibTeX style is
%       changed to |achemsol|, which will output the additional
%       database field |annotate|, intended for personal notes
%       about a particular database entry.  It also adds the
%       BibTeX key for each citation as a marginal note to the
%       output, using the |showkeys| package.
%    \item[|notsuper|] Switches from superscript citations
%       (\emph{e.g.}~Author \emph{et al.}$^3$) to
%        in-text ones in italics (\emph{e.g.}~Author
%        \emph{et al.}~(\emph{3})). There is a |super|
%        option for completeness, which simply gives the default
%        behaviour.
%     \item[|natbib|] Uses |natbib| rather than |cite| for citation
%        formatting; this also loads the |achemnat| style in place
%        of |achemso|.
% \end{description}
%
% \StopEventually{\bibliography{achemso}}
%
% \section{The Package Code}
%
%  The package code is not very complicated.  For the
%  interested reader(s), it is presented here.
%
% The usual setup code is executed.
%    \begin{macrocode}
%<*package>
\NeedsTeXFormat{LaTeX2e}
\ProvidesPackage{achemso}
  [2007/10/16 v2.2d LaTeX and BibTeX support for American
     Chemical Society publications]
%    \end{macrocode}
% \begin{macro}{\ACS@sctnnmbr}
% \begin{macro}{\ACS@lst}
% \begin{macro}{\ACS@note}
% \changes{v2.0}{2007/01/17}{Boolean values made internal to package}
% \begin{macro}{\ACS@super}
% \changes{v2.1}{2007/02/15}{New Boolean for citation control}
% \begin{macro}{\ACS@natbib}
% \changes{v2.2a}{2007/07/08}{New Boolean for |natbib| support}
% Boolean values are used to handle the options.
%    \begin{macrocode}
\newif \ifACS@sctnnmbr \ACS@sctnnmbrfalse
\newif \ifACS@list     \ACS@listfalse
\newif \ifACS@note     \ACS@notefalse
\newif \ifACS@super    \ACS@supertrue
\newif \ifACS@natbib   \ACS@natbibfalse
%    \end{macrocode}
% \end{macro}
% \end{macro}
% \end{macro}
% \end{macro}
% \end{macro}
% The options are processed.
%\changes{v2.2d}{2007/10/16}{Added \texttt{notes} option}
%    \begin{macrocode}
\DeclareOption{note}{\ExecuteOptions{notes}}
\DeclareOption{notes}{\ACS@notetrue}
\DeclareOption{number}{\ACS@sctnnmbrtrue}
\DeclareOption{super}{\ACS@supertrue}
\DeclareOption{list}{\ACS@listtrue}
\DeclareOption{notsuper}{\ACS@superfalse}
\DeclareOption{natbib}{\ACS@natbibtrue}
\DeclareOption*{\OptionNotUsed}
\ProcessOptions
%    \end{macrocode}
% \changes{v2.1}{2007/02/15}{|cite| package is loaded with different
% options depending on citation style requested}
% \changes{v2.2a}{2007/07/08}{|natbib| support added}
% The |cite| package is loaded to sort and compress references
% correctly. Depending upon the package option given, citations are
% either superscript or italic and in parentheses.
%    \begin{macrocode}
\ifACS@natbib
  \ifACS@super
    \RequirePackage[numbers,sort&compress,super]{natbib}
  \else
%    \end{macrocode}
% For in-line citations with |natbib|, we have to do a
% bit of work to get things to look right.  |natbib| uses
% \cmd{\citenumfont} to format the numbers, but it is not defined
% by default, so we have to use \cmd{\newcommand}.
%    \begin{macrocode}
    \RequirePackage[numbers,sort&compress,round]{natbib}
    \newcommand*{\citenumfont}{\textit}
  \fi
\else
  \ifACS@super
%    \end{macrocode}
%\changes{v2.2c}{2007/08/22}{Use the \texttt{overcite} alias for
%  \texttt{cite} as ACS have very old LaTeX system}
%    \begin{macrocode}
    \RequirePackage[nospace]{overcite}
  \else
%    \end{macrocode}
% Again in-line citations need some format changes.  In the case of
% |cite|, everything is defined initially.  Thus we can use
% \cmd{\renewcommand} for everything.
%    \begin{macrocode}
    \RequirePackage{cite}
    \renewcommand\citeleft{(}
    \renewcommand\citeright{)}
    \renewcommand\citeform[1]{\emph{#1}}
  \fi
\fi
%    \end{macrocode}
% If the |babel| package is loaded, the |note| option does not
% work.  So it is disabled here with a suitable warning.
%    \begin{macrocode}
\@ifpackageloaded{babel}
  {\ACS@notefalse\PackageWarning{achemso}%
    {babel package loaded - note option disabled}}
  {\relax}
%    \end{macrocode}
% \begin{macro}{\ACS@biberror}
% The function \cmd{\ACS@biberror} is defined here to provide an
% easy way of generating a warning if there is no name for a
% bibliography section.  This will only happen with non-standard
% class files.
%     \begin{macrocode}
\def\ACS@biberror{\PackageError{achemso}%
  {No bibliography name command defined}\@eha}
%    \end{macrocode}
% \end{macro}
% \begin{macro}{\refname}
% \begin{macro}{\bibname}
% The |note| option renames the references section to
% ``References and Notes''.  This applies for all standard
% document classes.
% The term ``Bibliography'' is not used in chemistry, the value of
% \cmd{\bibname} is redefined here in all cases where it exists.
%     \begin{macrocode}
\@ifundefined{refname}{%
  \@ifundefined{bibname}{%
    \ACS@biberror
  }{%
    \ifACS@note
      \renewcommand*{\bibname}{References and Notes}
    \else
      \renewcommand*{\bibname}{References}
    \fi
  }
}{%
  \ifACS@note
    \renewcommand*{\refname}{References and Notes}
  \fi
}
%    \end{macrocode}
% \end{macro}
% \end{macro}
% If the |number| option is set, the |tocbibind| package is
% used to number the bibliography.
% \changes{v2.0}{2007/01/17}{Switched to using \texttt{tocbibind}
% to produce number bibliography}
%    \begin{macrocode}
\ifACS@sctnnmbr
  \RequirePackage[numbib]{tocbibind}
\fi
%    \end{macrocode}
% \begin{macro}{\bibliographystyle}
% Depending on the package option, the bibliography style
% will either be |achemso| or |achemsol|.  The later is intended
% for listing the entire database.  The |list| option of the
% package selects this, and for listing also generates boxed
% labels for each reference.  The |showkeys| package provides
% this functionality.  If |natbib| is asked for, then the appropriate
% style files are used in place of the standard ones.
% \changes{v2.0}{2007/01/17}{Replaced custom code with
% \texttt{showkeys} package}
%    \begin{macrocode}
\ifACS@list
  \ifACS@natbib
    \bibliographystyle{achemlnt}
  \else
    \bibliographystyle{achemsol}
  \fi
  \RequirePackage[notcite]{showkeys}
\else
  \ifACS@natbib
    \bibliographystyle{achemnat}
  \else
    \bibliographystyle{achemso}
  \fi
\fi
%    \end{macrocode}
% \end{macro}
% \begin{macro}{\@biblabel}
% In order to re-format the bibliography labels, the easiest
% method is to redefine the \cmd{\@biblabel} macro from the LaTeX
% kernel.
%    \begin{macrocode}
\def\@biblabel#1{#1.}
%    \end{macrocode}
% \end{macro}
% \begin{macro}{\ACS@bibwarning}
% \begin{macro}{\bibliographystyle}
% To ensure that additional \cmd{\bibliographystyle} commands in the
% source are killed off.  The \cmd{\ACS@bibwarning} provides a clean
% method of generating the warning message.
% \changes{v2.0}{2007-01-17}{Command ignored in document body}
%    \begin{macrocode}
\def\ACS@bibwarning{\PackageWarning{achemso}%
  {Additional bibliographystyle command ignored}}
\def\bibliographystyle{\ACS@bibwarning\@gobble}
%    \end{macrocode}
% \end{macro}
% \end{macro}
% The package is complete.
%    \begin{macrocode}
%</package>
%    \end{macrocode}
% \PrintChanges
% \PrintIndex
% \Finale
% \iffalse
%<*bib>
ENTRY
  { address
%<list>    annotate
    author
    booktitle
    chapter
    doi
    edition
    editor
    howpublished
    institution
    journal
%<nat>    key
    note
    number
    organization
    pages
    publisher
    school
    series
    title
    type
    url
    volume
    year
  }
  {}
  {
  label
%<nat>    extra.label
%<nat>    short.list
  }

INTEGERS { output.state before.all mid.sentence after.sentence }
INTEGERS { after.block after.item author.or.editor }
INTEGERS { separate.by.semicolon }

FUNCTION {init.state.consts}
{ #0 'before.all :=
  #1 'mid.sentence :=
  #2 'after.sentence :=
  #3 'after.block :=
  #4 'after.item :=
}

FUNCTION {add.comma}
{ ", " * }

FUNCTION {add.semicolon}
{ "; " * }

%    \end{macrocode}
% For authors/editors we need to be able to add either a semi-colon
% or a comma.  This is done using a switching function, defined here.
%    \begin{macrocode}

FUNCTION {add.comma.or.semicolon}
{ #1 separate.by.semicolon =
    'add.semicolon
    'add.comma
  if$
}

FUNCTION {add.colon}
{ ": " * }

STRINGS { s t }

FUNCTION {output.nonnull}
{ 's :=
  output.state mid.sentence =
    { add.comma write$ }
    { output.state after.block =
      { add.semicolon write$
        newline$
        "\newblock " write$
      }
      { output.state before.all =
          'write$
          { output.state after.item =
            { " " * write$ }
            { add.period$ " " * write$ }
          if$
          }
        if$
        }
      if$
      mid.sentence 'output.state :=
    }
  if$
  s
}

FUNCTION {output}
{ duplicate$ empty$
    'pop$
    'output.nonnull
  if$
}

FUNCTION {output.check}
{ 't :=
  duplicate$ empty$
    { pop$ "Empty " t * " in " * cite$ * warning$ }
    'output.nonnull
  if$
}

%    \end{macrocode}
% For the standard file types, |output.bibitem| can come here.
% The same is not true for styles supporting |natbib|, and so
% |output.bibitem| occurs later for those styles.
% \iffalse
%<*!nat>
% \fi
%    \begin{macrocode}
FUNCTION {output.bibitem}
{ newline$
  "\bibitem{" write$
  cite$ write$
  "}" write$
  newline$
  ""
  before.all 'output.state :=
}

%    \end{macrocode}
% \iffalse
%</!nat>
% \fi
%    \begin{macrocode}
FUNCTION {new.block}
{ output.state before.all =
    'skip$
    { after.block 'output.state := }
  if$
}

FUNCTION {new.sentence}
{ output.state after.block =
    'skip$
    { output.state before.all =
        'skip$
        { after.sentence 'output.state := }
      if$
    }
  if$
}
%<*list>

FUNCTION {add.note}
{ annotate empty$
    'skip$
    { new.block
      "{\footnotesize " annotate * "}" * output }
  if$
}
%</list>

FUNCTION {fin.entry}
%<list>{ add.note
%<list>  add.period$
%<!list>{ add.period$
  write$
  newline$
}
FUNCTION {not}
{   { #0 }
    { #1 }
  if$
}

FUNCTION {and}
{   'skip$
    { pop$ #0 }
  if$
}

FUNCTION {or}
{   { pop$ #1 }
    'skip$
  if$
}

FUNCTION {field.or.null}
{ duplicate$ empty$
    { pop$ "" }
    'skip$
  if$
}

FUNCTION {emphasize}
{ duplicate$ empty$
    { pop$ "" }
    { "\emph{" swap$ * "}" * }
  if$
}

FUNCTION {boldface}
{ duplicate$ empty$
    { pop$ "" }
    { "\textbf{" swap$ * "}" * }
  if$
}

FUNCTION {paren}
{ duplicate$ empty$
    { pop$ "" }
    { "(" swap$ * ")" * }
  if$
}

FUNCTION {bbl.and}
{ "and" }

FUNCTION {bbl.chapter}
{ "Chapter" }

FUNCTION {bbl.editor}
{ "Ed." }

FUNCTION {bbl.editors}
{ "Eds." }

FUNCTION {bbl.edition}
{ "ed." }

FUNCTION {bbl.etal}
{ "et~al." }

FUNCTION {bbl.in}
{ "In" }

FUNCTION {bbl.inpress}
{ "in press" }

FUNCTION {bbl.msc}
{ "M.Sc.\ thesis" }

FUNCTION {bbl.page}
{ "p" }

FUNCTION {bbl.pages}
{ "pp" }

FUNCTION {bbl.phd}
{ "Ph.D.\ thesis" }

FUNCTION {bbl.submitted}
{ "submitted for publication" }

FUNCTION {bbl.techreport}
{ "Technical Report" }

FUNCTION {bbl.version}
{ "version" }

FUNCTION {bbl.volume}
{ "Vol." }

FUNCTION {bbl.first}
{ "1st" }

FUNCTION {bbl.second}
{ "2nd" }

FUNCTION {bbl.third}
{ "3rd" }

FUNCTION {bbl.fourth}
{ "4th" }

FUNCTION {bbl.fifth}
{ "5th" }

FUNCTION {bbl.st}
{ "st" }

FUNCTION {bbl.nd}
{ "nd" }

FUNCTION {bbl.rd}
{ "rd" }

FUNCTION {bbl.th}
{ "th" }

FUNCTION {eng.ord}
{ duplicate$ "1" swap$ *
  #-2 #1 substring$ "1" =
     { bbl.th * }
     { duplicate$ #-1 #1 substring$
       duplicate$ "1" =
         { pop$ bbl.st * }
         { duplicate$ "2" =
             { pop$ bbl.nd * }
             { "3" =
                 { bbl.rd * }
                 { bbl.th * }
               if$
             }
           if$
          }
       if$
     }
   if$
}

FUNCTION{is.a.digit}
{ duplicate$ "" =
    {pop$ #0}
    {chr.to.int$ #48 - duplicate$
     #0 < swap$ #9 > or not}
  if$
}

FUNCTION{is.a.number}
{
  { duplicate$ #1 #1 substring$ is.a.digit }
    {#2 global.max$ substring$}
  while$
  "" =
}

FUNCTION {extract.num}
{ duplicate$ 't :=
  "" 's :=
  { t empty$ not }
  { t #1 #1 substring$
    t #2 global.max$ substring$ 't :=
    duplicate$ is.a.number
      { s swap$ * 's := }
      { pop$ "" 't := }
    if$
  }
  while$
  s empty$
    'skip$
    { pop$ s }
  if$
}

FUNCTION {bibinfo.check}
{ swap$
  duplicate$ missing$
    { pop$ pop$
      ""
    }
    { duplicate$ empty$
        {
          swap$ pop$
        }
        { swap$
          pop$
        }
      if$
    }
  if$
}

FUNCTION {convert.edition}
{ extract.num "l" change.case$ 's :=
  s "first" = s "1" = or
    { bbl.first 't := }
    { s "second" = s "2" = or
        { bbl.second 't := }
        { s "third" = s "3" = or
            { bbl.third 't := }
            { s "fourth" = s "4" = or
                { bbl.fourth 't := }
                { s "fifth" = s "5" = or
                    { bbl.fifth 't := }
                    { s #1 #1 substring$ is.a.number
                        { s eng.ord 't := }
                        { edition 't := }
                      if$
                    }
                  if$
                }
              if$
            }
          if$
        }
      if$
    }
  if$
  t
}

FUNCTION {tie.or.space.connect}
{ duplicate$ text.length$ #3 <
    { "~" }
    { " " }
  if$
  swap$ * *
}

FUNCTION {space.connect}
{ " " swap$ * * }

INTEGERS { nameptr namesleft numnames }

FUNCTION {format.names}
{ 's :=
  #1 'nameptr :=
  s num.names$ 'numnames :=
  numnames 'namesleft :=
  numnames #15 >
    { s #1 "{vv~}{ll,}{~f.}{,~jj}" format.name$ 't :=
      t bbl.etal space.connect
    }
    {
       { namesleft #0 > }
       { s nameptr "{vv~}{ll,}{~f.}{,~jj}" format.name$ 't :=
           nameptr #1 >
             { namesleft #1 >
               { add.comma.or.semicolon t * }
               { numnames #2 >
                 { "" * }
                 'skip$
               if$
               t "others," =
                 { bbl.etal space.connect }
                 { add.comma.or.semicolon t * }
               if$
               }
             if$
             }
           't
         if$
         nameptr #1 + 'nameptr :=
         namesleft #1 - 'namesleft :=
         }
     while$
  }
  if$
}

FUNCTION {format.authors}
{ author empty$
    { "" }
    { #1 'author.or.editor :=
      #1 'separate.by.semicolon :=
      author format.names
    }
  if$
}

FUNCTION {format.editors}
{ editor empty$
    { "" }
    { #2 'author.or.editor :=
      #0 'separate.by.semicolon :=
      editor format.names
      add.comma
      editor num.names$ #1 >
        { bbl.editors }
        { bbl.editor }
      if$
      *
    }
  if$
}

FUNCTION {n.separate.multi}
{ 't :=
  ""
  #0 'numnames :=
  t text.length$ #4 > t is.a.number and
    {
      { t empty$ not }
      { t #-1 #1 substring$ is.a.number
          { numnames #1 + 'numnames := }
          { #0 'numnames := }
        if$
        t #-1 #1 substring$ swap$ *
        t #-2 global.max$ substring$ 't :=
        numnames #4 =
          { duplicate$ #1 #1 substring$ swap$
            #2 global.max$ substring$
            "," swap$ * *
            #1 'numnames :=
          }
          'skip$
        if$
      }
      while$
    }
    { t swap$ * }
  if$
}

FUNCTION {format.bvolume}
{ volume empty$
    { "" }
    { bbl.volume volume tie.or.space.connect }
  if$
}

FUNCTION {format.title.noemph}
{ 't :=
  t empty$
    { "" }
    { t }
  if$
}

FUNCTION {format.title}
{ 't :=
  t empty$
    { "" }
    { t emphasize }
  if$
}

FUNCTION {format.number.series}
{ volume empty$
    { number empty$
       { series field.or.null }
       { series empty$
         { "There is a number but no series in " cite$ * warning$ }
         { series number space.connect }
       if$
       }
      if$
    }
    { "" }
  if$
}

FUNCTION {format.url}
{ url empty$
    { "" }
    { new.sentence "\url{" url * "}" * }
  if$
}

% The specialised |output.bibitem| needed for |natbib| support now
% follows, along with the various support macros it needs.
% \iffalse
%<*nat>
% \fi
%    \begin{macrocode}
FUNCTION {format.full.names}
{'s :=
  #1 'nameptr :=
  s num.names$ 'numnames :=
  numnames 'namesleft :=
    { namesleft #0 > }
    { s nameptr
      "{vv~}{ll}" format.name$ 't :=
      nameptr #1 >
        {
          namesleft #1 >
            { ", " * t * }
            {
              numnames #2 >
                { "," * }
                'skip$
              if$
              t "others" =
                { bbl.etal * }
                { bbl.and space.connect t space.connect }
              if$
            }
          if$
        }
        't
      if$
      nameptr #1 + 'nameptr :=
      namesleft #1 - 'namesleft :=
    }
  while$
}

FUNCTION {author.editor.full}
{ author empty$
    { editor empty$
        { "" }
        { editor format.full.names }
      if$
    }
    { author format.full.names }
  if$
}

FUNCTION {author.full}
{ author empty$
    { "" }
    { author format.full.names }
  if$
}

FUNCTION {editor.full}
{ editor empty$
    { "" }
    { editor format.full.names }
  if$
}

FUNCTION {make.full.names}
{ type$ "book" =
  type$ "inbook" =
  or
    'author.editor.full
    { type$ "proceedings" =
        'editor.full
        'author.full
      if$
    }
  if$
}

FUNCTION {output.bibitem} { newline$
  "\bibitem[" write$
  label write$
  ")" make.full.names duplicate$ short.list =
     { pop$ }
     { * }
   if$
  "]{" * write$
  cite$ write$
  "}" write$
  newline$
  ""
  before.all 'output.state :=
}

%    \end{macrocode}
% \iffalse
%</nat>
% \fi
%    \begin{macrocode}
FUNCTION {n.dashify}
{ 't :=
  ""
    { t empty$ not }
    { t #1 #1 substring$ "-" =
    { t #1 #2 substring$ "--" = not
        { "--" *
          t #2 global.max$ substring$ 't :=
        }
        {   { t #1 #1 substring$ "-" = }
        { "-" *
          t #2 global.max$ substring$ 't :=
        }
          while$
        }
      if$
    }
    { t #1 #1 substring$ *
      t #2 global.max$ substring$ 't :=
    }
      if$
    }
  while$
}

FUNCTION {format.date}
{ year empty$
    { "" }
    { year boldface }
  if$
}

FUNCTION {format.bdate}
{ year empty$
    { "There's no year in " cite$ * warning$ }
    'year
  if$
}

FUNCTION {either.or.check}
{ empty$
    'pop$
    { "Can't use both " swap$ * " fields in " * cite$ * warning$ }
  if$
}

FUNCTION {format.edition}
{ edition duplicate$ empty$
    'skip$
    { convert.edition
      bbl.edition bibinfo.check
      " " * bbl.edition *
    }
  if$
}

INTEGERS { multiresult }

FUNCTION {multi.page.check}
{ 't :=
  #0 'multiresult :=
    { multiresult not
      t empty$ not
      and
    }
    { t #1 #1 substring$
      duplicate$ "-" =
      swap$ duplicate$ "," =
      swap$ "+" =
      or or
        { #1 'multiresult := }
        { t #2 global.max$ substring$ 't := }
      if$
    }
  while$
  multiresult
}

FUNCTION {format.pages}
{ pages empty$
    { "" }
    { pages multi.page.check
      { bbl.pages pages n.dashify tie.or.space.connect }
      { bbl.page pages tie.or.space.connect }
    if$
    }
  if$
}


FUNCTION {format.pages.required}
{ pages empty$
    { ""
      "There are no page numbers for " cite$ * warning$
      output
    }
    { pages multi.page.check
      { bbl.pages pages n.dashify tie.or.space.connect }
      { bbl.page pages tie.or.space.connect }
    if$
    }
  if$
}

FUNCTION {format.pages.nopp}
{ pages empty$
    { ""
      "There are no page numbers for " cite$ * warning$
      output
    }
    { pages multi.page.check
      { pages n.dashify space.connect }
      { pages space.connect }
    if$
    }
  if$
}

FUNCTION {format.pages.patent}
{ pages empty$
    { "There is no patent number for " cite$ * warning$ }
    { pages multi.page.check
      { pages n.dashify }
      { pages n.separate.multi }
      if$
    }
  if$
}

FUNCTION {format.vol.pages}
{ volume emphasize field.or.null
  duplicate$ empty$
    { pop$ format.pages.required }
    { add.comma pages n.dashify * }
  if$
}

FUNCTION {format.chapter.pages}
{ chapter empty$
    'format.pages
    { type empty$
    { bbl.chapter }
    { type "l" change.case$ }
      if$
      chapter tie.or.space.connect
      pages empty$
    'skip$
    { add.comma format.pages * }
      if$
    }
  if$
}

FUNCTION {format.title.in}
{ 's :=
  s empty$
    { "" }
    { editor empty$
      { bbl.in s format.title space.connect }
      { bbl.in s format.title space.connect
        add.semicolon format.editors *
      }
    if$
    }
  if$
}

FUNCTION {format.pub.address}
{ publisher empty$
    { "" }
    { address empty$
        { publisher }
        { publisher add.colon address *}
      if$
    }
  if$
}

FUNCTION {format.school.address}
{ school empty$
    { "" }
    { address empty$
        { school }
        { school add.colon address *}
      if$
    }
  if$
}

FUNCTION {format.organization.address}
{ organization empty$
    { "" }
    { address empty$
        { organization }
        { organization add.colon address *}
      if$
    }
  if$
}

FUNCTION {format.version}
{ edition empty$
    { "" }
    { bbl.version edition tie.or.space.connect }
  if$
}

FUNCTION {empty.misc.check}
{ author empty$ title empty$ howpublished empty$
  year empty$ note empty$ url empty$
  and and and and and
    { "all relevant fields are empty in " cite$ * warning$ }
    'skip$
  if$
}

FUNCTION {empty.doi.note}
{ doi empty$ note empty$ and
    { "Need either a note or DOI for " cite$ * warning$ }
    'skip$
  if$
}

FUNCTION {format.thesis.type}
{ type empty$
    'skip$
    { pop$
      type emphasize
    }
  if$
}

FUNCTION {article}
{ output.bibitem
  format.authors "author" output.check
  after.item 'output.state :=
  journal emphasize "journal" output.check
  after.item 'output.state :=
  format.date "year" output.check
  volume empty$
    { ""
      format.pages.nopp output
    }
    { format.vol.pages output }
  if$
  note output
  fin.entry
}

FUNCTION {book}
{ output.bibitem
  author empty$
    { booktitle empty$
        { title format.title "title" output.check }
        { booktitle format.title "booktitle" output.check }
      if$
      format.edition output
      new.block
      editor empty$
        { "Need either an author or editor for " cite$ * warning$ }
        { "" format.editors * "editor" output.check }
      if$
    }
    { format.authors output
      after.item 'output.state :=
      "author and editor" editor either.or.check
      booktitle empty$
        { title format.title "title" output.check }
        { booktitle format.title "booktitle" output.check }
      if$
      format.edition output
    }
  if$
  new.block
  format.number.series output
  new.block
  format.pub.address "publisher" output.check
  format.bdate "year" output.check
  new.block
  format.bvolume output
  pages empty$
    'skip$
    { format.pages output }
  if$
  note output
  fin.entry
}

FUNCTION {booklet}
{ output.bibitem
  format.authors output
  after.item 'output.state :=
  title format.title "title" output.check
  howpublished output
  address output
  format.date output
  note output
  fin.entry
}

FUNCTION {inbook}
{ output.bibitem
  author empty$
    { title format.title "title" output.check
      format.edition output
      new.block
      editor empty$
      { "Need at least an author or an editor for " cite$ * warning$ }
      { "" format.editors * "editor" output.check }
    if$
    }
    { format.authors output
      after.item 'output.state :=
      title format.title.in "title" output.check
      format.edition output
    }
  if$
  new.block
  format.number.series output
  new.block
  format.pub.address "publisher" output.check
  format.bdate "year" output.check
  new.block
  format.bvolume output
  format.chapter.pages "chapter and pages" output.check
  note output
  fin.entry
}

FUNCTION {incollection}
{ output.bibitem
  author empty$
    { booktitle format.title "booktitle" output.check
      format.edition output
      new.block
      editor empty$
        { "Need at least an author or an editor for " cite$ * warning$ }
        { "" format.editors * "editor" output.check }
      if$
    }
    { format.authors output
      after.item 'output.state :=
      title empty$
        'skip$
        { title format.title.noemph output }
      if$
      after.sentence 'output.state :=
      booktitle format.title.in "booktitle" output.check
      format.edition output
    }
  if$
  new.block
  format.number.series output
  new.block
  format.pub.address "publisher" output.check
  format.bdate "year" output.check
  new.block
  format.bvolume output
  format.chapter.pages "chapter and pages" output.check
  note output
  fin.entry
}

FUNCTION {inpress}
{ output.bibitem
  format.authors "author" output.check
  after.item 'output.state :=
  journal emphasize "journal" output.check
  doi empty$
    {  bbl.inpress output }
    {  after.item 'output.state :=
       format.date output
       "DOI:" doi tie.or.space.connect output
    }
  if$
  note output
  fin.entry
}

FUNCTION {inproceedings}
{ output.bibitem
  format.authors "author" output.check
  after.item 'output.state :=
  title empty$
    'skip$
    { title format.title.noemph output
      after.sentence 'output.state :=
    }
  if$
  booktitle format.title output
  address output
  format.bdate "year" output.check
  pages empty$
    'skip$
    { new.block
      format.pages output }
  if$
  note output
  fin.entry
}

FUNCTION {manual}
{ output.bibitem
  format.authors output
  after.item 'output.state :=
  title format.title "title" output.check
  format.version output
  new.block
  format.organization.address output
  format.bdate output
  note output
  fin.entry
}

FUNCTION {mastersthesis}
{ output.bibitem
  format.authors "author" output.check
  after.item 'output.state :=
  bbl.msc format.thesis.type output
  format.school.address "school" output.check
  format.bdate "year" output.check
  note output
  fin.entry
}

FUNCTION {misc}
{ output.bibitem
  format.authors output
  after.item 'output.state :=
  title empty$
    'skip$
    { title format.title output }
  if$
  howpublished output
  year output
  format.url output
  note output
  fin.entry
  empty.misc.check
}

FUNCTION {patent}
{ output.bibitem
  format.authors "author" output.check
  after.item 'output.state :=
  journal "journal" output.check
  after.item 'output.state :=
  format.pages.patent "pages" output.check
  format.bdate "year" output.check
  note output
  fin.entry
}

FUNCTION {phdthesis}
{ output.bibitem
  format.authors "author" output.check
  after.item 'output.state :=
  bbl.phd format.thesis.type output
  format.school.address "school" output.check
  format.bdate "year" output.check
  note output
  fin.entry
}

FUNCTION {proceedings}
{ output.bibitem
  title format.title.noemph "title" output.check
  address output
  format.bdate "year" output.check
  pages empty$
    'skip$
    { new.block
      format.pages output }
  if$
  note output
  fin.entry
}

FUNCTION {remark}
{ output.bibitem
  note "note" output.check
  fin.entry
}

FUNCTION {submitted}
{ output.bibitem
  format.authors "author" output.check
  bbl.submitted output
  fin.entry
}

FUNCTION {techreport}
{ output.bibitem
  format.authors "author" output.check
  after.item 'output.state :=
  title format.title "title" output.check
  new.block
  type empty$
    'bbl.techreport
    'type
  if$
  number empty$
    'skip$
    { number tie.or.space.connect }
  if$
  output
  format.pub.address output
  format.bdate "year" output.check
  pages empty$
    'skip$
    { new.block
      format.pages output }
  if$
  note output
  fin.entry
}

FUNCTION {unpublished}
{ output.bibitem
  format.authors "author" output.check
  after.item 'output.state :=
  journal empty$
    'skip$
    { journal emphasize "journal" output.check }
  if$
  doi empty$
    {  note output }
    {  after.item 'output.state :=
       format.date output
       "DOI:" doi tie.or.space.connect output
    }
  if$
  fin.entry
  empty.doi.note
}

FUNCTION {conference} {inproceedings}

FUNCTION {other} {patent}

FUNCTION {default.type} {misc}

MACRO {jan} {"Jan."}
MACRO {feb} {"Feb."}
MACRO {mar} {"Mar."}
MACRO {apr} {"Apr."}
MACRO {may} {"May"}
MACRO {jun} {"June"}
MACRO {jul} {"July"}
MACRO {aug} {"Aug."}
MACRO {sep} {"Sept."}
MACRO {oct} {"Oct."}
MACRO {nov} {"Nov."}
MACRO {dec} {"Dec."}

MACRO {acchemr} {"Acc.\ Chem.\ Res."}
MACRO {aacsa} {"Adv.\ {ACS} Abstr."}
MACRO {anchem} {"Anal.\ Chem."}
MACRO {bioch} {"Biochemistry"}
MACRO {bicoc} {"Bioconj.\ Chem."}  % ***
MACRO {bitech} {"Biotechnol.\ Progr."}  % ***
MACRO {chemeng} {"Chem.\ Eng.\ News"}
MACRO {chs} {"Chem.\ Health Safety"} % ***
MACRO {crt} {"Chem.\ Res.\ Toxicol."} % ***
MACRO {chemrev} {"Chem.\ Rev."} % ***
MACRO {cmat} {"Chem.\ Mater."} % ***
MACRO {chemtech} {"{CHEMTECH}"} % ***
MACRO {enfu} {"Energy Fuels"} % ***
MACRO {envst} {"Environ.\ Sci.\ Technol."}
MACRO {iecf} {"Ind.\ Eng.\ Chem.\ Fundam."}
MACRO {iecpdd} {"Ind.\ Eng.\ Chem.\ Proc.\ Des.\ Dev."}
MACRO {iecprd} {"Ind.\ Eng.\ Chem.\ Prod.\ Res.\ Dev."}
MACRO {iecr} {"Ind.\ Eng.\ Chem.\ Res."} % ***
MACRO {inor} {"Inorg.\ Chem."}
MACRO {jafc} {"J.~Agric.\ Food Chem."}
MACRO {jacs} {"J.~Am.\ Chem.\ Soc."}
MACRO {jced} {"J.~Chem.\ Eng.\ Data"}
MACRO {jcics} {"J.~Chem.\ Inf.\ Comput.\ Sci."}
MACRO {jmc} {"J.~Med.\ Chem."}
MACRO {joc} {"J.~Org.\ Chem."}
MACRO {jps} {"J.~Pharm.\ Sci."}
MACRO {jpcrd} {"J.~Phys.\ Chem.\ Ref.\ Data"} % ***
MACRO {jpc} {"J.~Phys.\ Chem."}
MACRO {jpca} {"J.~Phys.\ Chem.~A"}
MACRO {jpcb} {"J.~Phys.\ Chem.~B"}
MACRO {lang} {"Langmuir"}
MACRO {macro} {"Macromolecules"}
MACRO {orgmet} {"Organometallics"}
MACRO {orglett} {"Org.\ Lett."}

MACRO {jft} {"J.~Chem.\ Soc., Faraday Trans."}
MACRO {jft1} {"J.~Chem.\ Soc., Faraday Trans.\ 1"}
MACRO {jft2} {"J.~Chem.\ Soc., Faraday Trans.\ 2"}
MACRO {tfs} {"Trans.\ Faraday Soc."}
MACRO {jcis} {"J.~Colloid Interface Sci."}
MACRO {acis} {"Adv.~Colloid Interface Sci."}
MACRO {cs} {"Colloids Surf."}
MACRO {csa} {"Colloids Surf.\ A"}
MACRO {csb} {"Colloids Surf.\ B"}
MACRO {pcps} {"Progr.\ Colloid Polym.\ Sci."}
MACRO {jmr} {"J.~Magn.\ Reson."}
MACRO {jmra} {"J.~Magn.\ Reson.\ A"}
MACRO {jmrb} {"J.~Magn.\ Reson.\ B"}
MACRO {sci} {"Science"}
MACRO {nat} {"Nature"}
MACRO {jcch} {"J.~Comput.\ Chem."}
MACRO {cca} {"Croat.\ Chem.\ Acta"}
MACRO {angew} {"Angew.\ Chem., Int.\ Ed."}
MACRO {chemeurj} {"Chem.---Eur.\ J."}

MACRO {poly} {"Polymer"}
MACRO {ajp} {"Am.\ J.\ Phys."}
MACRO {rsi} {"Rev.\ Sci.\ Instrum."}
MACRO {jcp} {"J.~Chem.\ Phys."}
MACRO {cpl} {"Chem.\ Phys.\ Lett."}
MACRO {molph} {"Mol.\ Phys."}
MACRO {pac} {"Pure Appl.\ Chem."}
MACRO {jbc} {"J.~Biol.\ Chem."}
MACRO {tl} {"Tetrahedron Lett."}
MACRO {psisoe} {"Proc.\ SPIE-Int.\ Soc.\ Opt.\ Eng."}
MACRO {prb} {"Phys.\ Rev.\ B:\ Condens.\ Matter Mater. Phys."}
MACRO {jap} {"J.~Appl.\ Phys."}
MACRO {pnac} {"Proc.\ Natl.\ Acad.\ Sci.\ U.S.A."}
MACRO {bba} {"Biochim.\ Biophys.\ Acta"}
MACRO {nar} {"Nucleic.\ Acid Res."}

READ

%    \end{macrocode}
% The nature of the initialise code depends on whether we need to
% support |natbib|.  First the simple case is handled.
% \iffalse
%<*!nat>
% \fi
%    \begin{macrocode}
STRINGS { longest.label }

INTEGERS { number.label longest.label.width }

FUNCTION {initialize.longest.label}
{ "" 'longest.label :=
  #1 'number.label :=
  #0 'longest.label.width :=
}

FUNCTION {longest.label.pass}
{ number.label int.to.str$ 'label :=
  number.label #1 + 'number.label :=
  label width$ longest.label.width >
    { label 'longest.label :=
      label width$ 'longest.label.width :=
    }
    'skip$
  if$
}

EXECUTE {initialize.longest.label}

ITERATE {longest.label.pass}

%    \end{macrocode}
% \iffalse
%</!nat>
% \fi
% Now the |natbib| system is sorted out, basically by copying from
% |plainnat.bst|.
% \iffalse
%<*nat>
% \fi
%    \begin{macrocode}
INTEGERS { len }

FUNCTION {chop.word}
{ 's :=
  'len :=
  s #1 len substring$ =
    { s len #1 + global.max$ substring$ }
    's
  if$
}

FUNCTION {format.lab.names}
{ 's :=
  s #1 "{vv~}{ll}" format.name$
  s num.names$ duplicate$
  #2 >
    { pop$ bbl.etal space.connect }
    { #2 <
        'skip$
        { s #2 "{ff }{vv }{ll}{ jj}" format.name$ "others" =
            { bbl.etal space.connect }
            { bbl.and space.connect s #2 "{vv~}{ll}" format.name$ space.connect }
          if$
        }
      if$
    }
  if$
}

FUNCTION {author.key.label}
{ author empty$
    { key empty$
        { cite$ #1 #3 substring$ }
        'key
      if$
    }
    { author format.lab.names }
  if$
}

FUNCTION {author.editor.key.label}
{ author empty$
    { editor empty$
        { key empty$
            { cite$ #1 #3 substring$ }
            'key
          if$
        }
        { editor format.lab.names }
      if$
    }
    { author format.lab.names }
  if$
}

FUNCTION {author.key.organization.label}
{ author empty$
    { key empty$
        { organization empty$
            { cite$ #1 #3 substring$ }
            { "The " #4 organization chop.word #3 text.prefix$ }
          if$
        }
        'key
      if$
    }
    { author format.lab.names }
  if$
}

FUNCTION {editor.key.organization.label}
{ editor empty$
    { key empty$
        { organization empty$
            { cite$ #1 #3 substring$ }
            { "The " #4 organization chop.word #3 text.prefix$ }
          if$
        }
        'key
      if$
    }
    { editor format.lab.names }
  if$
}

FUNCTION {calc.short.authors}
{ type$ "book" =
  type$ "inbook" =
  or
    'author.editor.key.label
    { type$ "proceedings" =
        'editor.key.organization.label
        { type$ "manual" =
            'author.key.organization.label
            'author.key.label
          if$
        }
      if$
    }
  if$
  'short.list :=
}

FUNCTION {calc.label}
{ calc.short.authors
  short.list
  "("
  *
  year duplicate$ empty$
  short.list key field.or.null = or
     { pop$ "" }
     'skip$
  if$
  *
  'label :=
}

ITERATE {calc.label}

STRINGS { longest.label last.label next.extra }

INTEGERS { longest.label.width last.extra.num number.label }

FUNCTION {initialize.longest.label}
{ "" 'longest.label :=
  #0 int.to.chr$ 'last.label :=
  "" 'next.extra :=
  #0 'longest.label.width :=
  #0 'last.extra.num :=
  #0 'number.label :=
}

FUNCTION {forward.pass}
{ last.label label =
    { last.extra.num #1 + 'last.extra.num :=
      last.extra.num int.to.chr$ 'extra.label :=
    }
    { "a" chr.to.int$ 'last.extra.num :=
      "" 'extra.label :=
      label 'last.label :=
    }
  if$
  number.label #1 + 'number.label :=
}

EXECUTE {initialize.longest.label}

ITERATE {forward.pass}

%    \end{macrocode}
% \iffalse
%</nat>
% \fi

FUNCTION {begin.bib}
{ preamble$ empty$
    'skip$
    { preamble$ write$ newline$ }
  if$
  "\providecommand{\url}[1]{\texttt{#1}}"
  write$ newline$
  "\providecommand{\refin}[1]{\\ \textbf{Referenced in:} #1}"
  write$ newline$
%<nat>  "\providecommand{\natexlab}[1]{#1}"
%<nat>  write$ newline$
%<!nat>  "\begin{thebibliography}{"  longest.label  * "}" *
%<nat>  "\begin{thebibliography}{" number.label int.to.str$ * "}" *
  write$ newline$
}

EXECUTE {begin.bib}

EXECUTE {init.state.consts}

ITERATE {call.type$}

FUNCTION {end.bib}
{ newline$
  "\end{thebibliography}" write$ newline$
}

EXECUTE {end.bib}
%</bib>
%<*database>
@BOOK{Coghill2006,
  title = {{T}he {ACS} {S}tyle {G}uide},
  publisher = {{O}xford {U}niversity {P}ress, {I}nc. and
               {T}he {A}merican {C}hemical {S}ociety},
  year = {2006},
  editor = {Coghill, Anne M. and Garson, Lorrin R.},
  address = {{N}ew {Y}ork},
  edition = {3},
  subtitle = {{E}ffective {C}ommunication of {S}cientific {I}nformation},
}

@MISC{ACS2007,
  url = {http://pubs.acs.org/books/references.shtml},
}
%</database>
%<*jawltxdoc>
% The following is convenient method for collecting together package
% loading, formatting commands and new macros used to format |dtx|
% files written by the current author.  It is based on the similar
% files provided by Will Robertson in his packages and Heiko Oberdiek
% as a stand-alone package.  Notice that it is not intended for other
% users: there is no error checking!  However, it is covered by the
% LPPL in the same way as the rest of this package.
%
\NeedsTeXFormat{LaTeX2e}
\ProvidesPackage{jawltxdoc}
  [2007/10/14 v1.0b]
% First of all, a number of support packages are loaded.
\usepackage[T1]{fontenc}
\usepackage[english,UKenglish]{babel}
\usepackage[scaled=0.95]{helvet}
\usepackage[version=3]{mhchem}
\usepackage[final]{microtype}
\usepackage[osf]{mathpazo}
\usepackage{booktabs,array,url,graphicx,courier,unitsdef}
\usepackage{upgreek,ifpdf,listings}
% If using PDFLaTeX, the source will be attached to the PDF. This
% is basically the system used by Heiko Oberdiek, but with a check
% that PDF mode is enabled.
\ifpdf
  \usepackage{embedfile}
  \embedfile[%
    stringmethod=escape,%
    mimetype=plain/text,%
    desc={LaTeX docstrip source archive for package `\jobname'}%
    ]{\jobname.dtx}
\fi
\usepackage{\jobname}
\usepackage[numbered]{hypdoc}
%
% To typeset examples, a new environment is needed.  The code below
% is based on that in used by |listings|, but is modified to get
% better formatting for this context.  The formatting of the output
% is basically that in Will Robertson's |dtx-style| file.
\newlength\LaTeXwidth
\newlength\LaTeXoutdent
\newlength\LaTeXgap
\setlength\LaTeXgap{1em}
\setlength\LaTeXoutdent{-0.15\textwidth}
\def\typesetexampleandcode{%
  \begin{list}{}{%
    \setlength\itemindent{0pt}
    \setlength\leftmargin\LaTeXoutdent
    \setlength\rightmargin{0pt}
  }
  \item
    \setlength\LaTeXoutdent{-0.15\textwidth}
    \begin{minipage}[c]{\textwidth-\LaTeXwidth-\LaTeXoutdent-\LaTeXgap}
      \lst@sampleInput
    \end{minipage}%
    \hfill%
    \begin{minipage}[c]{\LaTeXwidth}%
      \hbox to\linewidth{\box\lst@samplebox\hss}%
    \end{minipage}%
  \end{list}
}
\def\typesetcodeandexample{%
  \begin{list}{}{%
    \setlength\itemindent{0pt}
    \setlength\leftmargin{0pt}
    \setlength\rightmargin{0pt}
  }
  \item
    \begin{minipage}[c]{\LaTeXwidth}%
      \hbox to\linewidth{\box\lst@samplebox\hss}%
    \end{minipage}%
    \lst@sampleInput
  \end{list}
}
\def\typesetfloatexample{%
  \begin{list}{}{%
    \setlength\itemindent{0pt}
    \setlength\leftmargin{0pt}
    \setlength\rightmargin{0pt}
  }
  \item
    \lst@sampleInput
    \begin{minipage}[c]{\LaTeXwidth}%
      \hbox to\linewidth{\box\lst@samplebox\hss}%
    \end{minipage}%
  \end{list}
}
\def\typesetcodeonly{%
  \begin{list}{}{%
    \setlength\itemindent{0pt}
    \setlength\leftmargin{0pt}
    \setlength\rightmargin{0pt}
  }
  \item
    \begin{minipage}[c]{\LaTeXwidth}%
      \hbox to\linewidth{\box\lst@samplebox\hss}%
    \end{minipage}%
  \end{list}
}
\edef\LaTeXexamplefile{\jobname.tmp}
\lst@RequireAspects{writefile}
\newbox\lst@samplebox
\lstnewenvironment{LaTeXexample}[1][\typesetexampleandcode]{%
  \let\typesetexample#1
  \global\let\lst@intname\@empty
  \setbox\lst@samplebox=\hbox\bgroup
  \setkeys{lst}{language=[LaTeX]{TeX},tabsize=4,gobble=2,%
    breakindent=0pt,basicstyle=\small\ttfamily,basewidth=0.51em,%
    keywordstyle=\color{blue},%
% Notice that new keywords should be added here. The list is simply
% macro names needed to typeset documentation of the package
% author.
    morekeywords={bibnote,citenote,bibnotetext,bibnotemark,%
      thebibnote,bibnotename,includegraphics,schemeref,%
      floatcontentsleft,floatcontentsright,floatcontentscentre,%
      schemerefmarker,compound,schemerefformat,color,%
      startchemical,stopchemical,chemical,setupchemical,bottext,%
      listofschemes}}
  \lst@BeginAlsoWriteFile{\LaTeXexamplefile}
}{%
  \lst@EndWriteFile\egroup
  \setlength\LaTeXwidth{\wd\lst@samplebox}
  \typesetexample%
}
\def\lst@sampleInput{%
  \MakePercentComment\catcode`\^^M=10\relax
  \small%
  {\setkeys{lst}{SelectCharTable=\lst@ReplaceInput{\^\^I}%
    {\lst@ProcessTabulator}}%
    \leavevmode \input{\LaTeXexamplefile}}%
  \MakePercentIgnore%
}
\hyphenation{PDF-LaTeX}
%</jawltxdoc>
%\fi

%
% Documentation:
%    (a) Without write18 enabled:
%          pdflatex achemso.dtx
%          bibtex8 --wolfgang achemso.aux
%          makeindex -s gind.ist achemso.idx
%          makeindex -s gglo.ist -o achemso.gls  achemso.glo
%          pdflatex achemso.dtx
%          makeindex -s gind.ist achemso.idx
%          makeindex -s gglo.ist -o achemso.gls  achemso.glo
%          pdflatex achemso.dtx
%    (b) With write18 enabled:
%          pdflatex achemso.dtx
%          bibtex8 --wolfgang achemso.aux
%          pdflatex achemso.dtx
%          pdflatex achemso.dtx
%
% Installation:
%     Copy achemso.sty and the achmes*.bst files to a location
%     searched by TeX, and if required by your TeX installation,
%     run the appropriate command to build a hash of files
%     (texhash, mpm --update-db, etc.)
%
% Note:
%     The jawltxdoc.sty file is not needed for installation,
%     only for building the documentation.  It may be deleted.
%
%<*ignore>
% This is all taken verbatim from Heiko Oberdiek's packages
\begingroup
  \def\x{LaTeX2e}%
\expandafter\endgroup
\ifcase 0\ifx\install y1\fi\expandafter
         \ifx\csname processbatchFile\endcsname\relax\else1\fi
         \ifx\fmtname\x\else 1\fi\relax
\else\csname fi\endcsname
%</ignore>
%<*install>
\input docstrip.tex
\keepsilent
\askforoverwritefalse
\preamble
 ----------------------------------------------------------------
 The achemso package - LaTeX and BibTeX support for American
 Chemical Society publications
 Maintained by Joseph Wright
 E-mail: joseph.wright@morningstar2.co.uk
 Released under the LaTeX Project Public License v1.3 or later
 See http://www.latex-project.org/lppl.txt
 ----------------------------------------------------------------

\endpreamble
\Msg{Generating achemso files:}
\usedir{tex/latex/contib/achemso}
\generate{\file{\jobname.ins}{\from{\jobname.dtx}{install}}
          \file{\jobname.sty}{\from{\jobname.dtx}{package}}
          \file{jawltxdoc.sty}{\from{\jobname.dtx}{jawltxdoc}}
}
\declarepostamble\bibtexable
\endpostamble
\usedir{bibtex/bst/achemso}
\generate{\usepostamble\bibtexable
          \file{achemso.bst}{\from{achemso.dtx}{bib}}
          \file{achemnat.bst}{\from{achemso.dtx}{bib,nat}}
          \file{achemsol.bst}{\from{achemso.dtx}{bib,list}}
          \file{achemlnt.bst}{\from{achemso.dtx}{bib,list,nat}}
}
\generate{\usepostamble\empty\usepreamble\empty
          \file{achemso.bib}{\from{achemso.dtx}{database}}
}
\endbatchfile
%</install>
%<*ignore>
\fi
% Will Robertson's trick
\immediate\write18{makeindex -s gind.ist -o \jobname.ind  \jobname.idx}
\immediate\write18{makeindex -s gglo.ist -o \jobname.gls  \jobname.glo}
%</ignore>
%<*driver>
\PassOptionsToClass{a4paper}{article}
\documentclass{ltxdoc}
\EnableCrossrefs
\CodelineIndex
\RecordChanges
%\OnlyDescription
% The various formatting commands used in this file are collected
% together in |jawltxdoc|.
\usepackage{jawltxdoc}
\begin{document}
  \DocInput{\jobname.dtx}
\end{document}
%</driver>
% \fi
%
% \CheckSum{105}
%
% \CharacterTable
%  {Upper-case    \A\B\C\D\E\F\G\H\I\J\K\L\M\N\O\P\Q\R\S\T\U\V\W\X\Y\Z
%   Lower-case    \a\b\c\d\e\f\g\h\i\j\k\l\m\n\o\p\q\r\s\t\u\v\w\x\y\z
%   Digits        \0\1\2\3\4\5\6\7\8\9
%   Exclamation   \!     Double quote  \"     Hash (number) \#
%   Dollar        \$     Percent       \%     Ampersand     \&
%   Acute accent  \'     Left paren    \(     Right paren   \)
%   Asterisk      \*     Plus          \+     Comma         \,
%   Minus         \-     Point         \.     Solidus       \/
%   Colon         \:     Semicolon     \;     Less than     \<
%   Equals        \=     Greater than  \>     Question mark \?
%   Commercial at \@     Left bracket  \[     Backslash     \\
%   Right bracket \]     Circumflex    \^     Underscore    \_
%   Grave accent  \`     Left brace    \{     Vertical bar  \|
%   Right brace   \}     Tilde         \~}
%
%\GetFileInfo{\jobname.sty}
%
%\changes{v1.0}{1998/06/01}{Initial release of package by Mats
%   Dahlgren}
%\changes{v2.0}{2007/01/17}{Re-write of package by Joseph Wright}
%\changes{v2.0}{2007/01/17}{Several improvements to BibTeX style
%  files}
%\changes{v2.0}{2007/01/17}{License changed to LPPL}
%\changes{v2.1}{2007/02/15}{Updated documentation to reflect 3rd
%  edition of ACS Style Guide}
%\changes{v2.1}{2007/02/15}{BibTeX style improved to reflect 3rd
%  edition of ACS Style Guide}
%\changes{v2.2}{2007/06/05}{Added \texttt{natbib} support}
%\changes{v2.2a}{2007/07/08}{Fixed separation of editor names}
%\changes{v2.2a}{2007/07/08}{Bug fixes to \texttt{natbib} and list
% support}
%\changes{v2.2a}{2007/07/08}{\texttt{title} field included in output
% for \texttt{incollection} records}
%\changes{v2.2b}{2007/07/09}{Bug fix to name formatting}
%\changes{v2.2d}{2007/10/16}{Added \textsc{url} field to
%  \texttt{misc} output}
%\changes{v2.2d}{2007/10/16}{Package design improved}
%
%\DoNotIndex{\@biblabel,\@eha,\@gobble,\@ifpackageloaded,\@ifundefined}
%\DoNotIndex{\bibliographystyle,\bibname,\citeform,\citeleft}
%\DoNotIndex{\citenumfont,\citeright,\DeclareOption,\def,\else,\emph}
%\DoNotIndex{\fi,\ifx,\NeedsTeXFormat,\newcommand,\newif}
%\DoNotIndex{\OptionNotUsed,\PackageError,\PackageWarning}
%\DoNotIndex{\ProcessOptions,\ProvidesPackage,\refname,\relax}
%\DoNotIndex{\renewcommand,\RequirePackage,\textit,}
%
% \title{\texttt{achemso} --- LaTeX and BibTeX support for American
%   Chemical Society publications%
%   \thanks{This file describes version \fileversion, last revised
%           \filedate.}}
% \author{Joseph Wright%
%   \thanks{E-mail: joseph.wright@morningstar2.co.uk}}
% \date{Released \filedate}
%
%\maketitle
%
%\begin{abstract}
% The |achemso| package provides a BibTeX style in accordance with
% the requirements of the journals of the American Chemical Society,
% along with a supporting LaTeX package file. Also provided is a
% BibTeX style file to be used for bibliography database listings.
%\end{abstract}
%
% \section{Introduction}
%
% Synthetic chemists do not, in the main, use LaTeX for the
% preparation of journal articles. Some journals, mainly in the
% physical chemistry area, do accept LaTeX submissions.  Given the
% clear advantages of LaTeX over other methods, it would be
% nice to be able to use LaTeX for preparing reports. Thus the need
% for BibTeX styles for chemistry is real. The package |achemso|
% provides for a BibTeX style and other support for articles and
% reports in the style of the American Chemical Society (ACS).
%
% As describe in \emph{The ACS Style Guide} \cite{Coghill2006},
% almost all ACS publications use the same style for the formatting
% of references.  The reproduction of this style is the aim of the
% BibTeX style file provided here.  However, the ACS use different
% citation styles in different publications.  The |achemso| package
% provides support for the two numerical systems: superscript
% and italic in-text citations.  The majority of ACS journals use
% the superscript method (Table \ref{tbl:journals-super}), with a
% smaller number using the italic system (Table
% \ref{tbl:journals-inline}). The journal \emph{Biochemistry} does
% not use the standard ACS style for references, and so is not
% covered by the |achemso| package.
% \begin{table}
%   \centering
%   \small
%   \begin{tabular}{>{\itshape}l>{\itshape}l}
%     \toprule
%     \upshape{Journal Title} & \upshape{\emph{CASSI} Abbreviation} \\
%     \midrule
%     Accounts of Chemical Research & Acc.~Chem.~Res. \\
%     Analytical Chemistry & Anal.~Chem. \\
%     Biomacromolecules & Biomacromolecules \\
%     Chemical Reviews & Chem.~Rev. \\
%     Chemistry of Materials & Chem.~Mater. \\
%     Crystal Growth \& Design & Cryst.~Growth Des. \\
%     Energy \& Fuels & Energy Fuels \\
%     Industrial \& Engineering Chemistry Research & Ind.~Eng.~Chem.~Res. \\
%     Inorganic Chemistry & Inorg.Chem. \\
%     Journal of the American Chemical Society & J.~Am.~Chem.~Soc. \\
%     Journal of Chemical and Engineering Data & J.~Chem.~Eng.~Data \\
%     Journal of Chemical Theory and Computation & J.~Chem.~Theory Comput. \\
%     Journal of Chemical Information and Modeling & J.~Chem.~Inf.~Model. \\
%     Journal of Combinatorial Chemistry & J.~Comb.~Chem. \\
%     Journal of Medicinal Chemistry & J.~Med.~Chem. \\
%     Journal of Natural Products & J.~Nat.~Prod. \\
%     The Journal of Organic Chemistry & J.~Org.~Chem. \\
%     The Journal of Physical Chemistry A & J.~Phys.~Chem.~A \\
%     The Journal of Physical Chemistry B & J.~Phys.~Chem.~B \\
%     The Journal of Physical Chemistry C & J.~Phys.~Chem.~C \\
%     Journal of Proteome Research & J.~Proteome Res. \\
%     Langmuir & Langmuir \\
%     Macromolecules & Macromolecules \\
%     Molecular Pharmaceutics & Mol.~Pharm. \\
%     Nano Letters & Nano Lett. \\
%     Organic Letters & Org.~Lett. \\
%     Organic Process Research \& Design & Org.~Process Res.~Dev. \\
%     Organometallics & Organometallics \\
%     \bottomrule
%   \end{tabular}
%   \caption{Journals using the ACS reference style with superscript citations}
%   \label{tbl:journals-super}
% \end{table}
% \begin{table}
%   \small
%   \centering
%   \begin{tabular}{>{\itshape}l>{\itshape}l}
%     \toprule
%     \upshape{Journal Title} & \upshape{\emph{CASSI} Abbreviation} \\
%     \midrule
%     ACS Chemical Biology & ACS Chem.~Biol. \\
%     Bioconjugate Chemistry & Bioconjugate Chem. \\
%     Biotechnology Progress & Biotechnol.~Prog. \\
%     Chemical Research in Toxicology & Chem.~Res.~Toxicol. \\
%     Environmental Science and Technology & Envirn.~Sci.~Technol. \\
%     Journal of Agricultural and Food Chemistry & J.~Agric.~Food Chem. \\
%     \bottomrule
%   \end{tabular}
%   \caption{Journals using the ACS reference style with in-text citations}
%   \label{tbl:journals-inline}
% \end{table}
%
% This package consists of two BibTeX files (|achemso.bst|
% and |achemsol.bst|) along with a small LaTeX file |achemso.sty|.
% The naming of the package is slightly unusual, but follows from
% the need to pick a unique name.  To quote the documentation to the
% first version:
% \begin{quote}
%   there is already a LaTeX 2.09 and
%   BibTeX style package called |acsarticle| and
%   |acs.bst|, which are not ``ACS'' as in `American Chemical
%   Society' (rather, this package is
%   formatting the output according to the instructions of
%   \emph{Advances in Control Systems}).  Hence, \emph{this}
%   new package had to be given another name.  The name of choice
%   was then |achemso|, which is made from the words
%   ``\emph{A}merican \emph{Chem}ical \emph{So}ciety.''
% \end{quote}
%
% \subsection{Change of maintainer}
%
% This package was initially released by Mats Dahlgren.  He no
% longer has time to devote to LaTeX development.  With his permission,
% the package has therefore been taken over by Joseph
% Wright, the maintainer of the the |rsc| package.  The majority of
% the package has been rebuilt and the BibTeX style file has been
% totally overhauled.  Any mistakes are entirely the fault of the
% new maintainer!
%
% \section{The BibTeX style files}
%
% The BibTeX style files implement the bibliographic style specified
% by the ACS in \emph{The ACS Style Guide} \cite{Coghill2006},
% on the ACS website \cite{ACS2007}, and in current ACS publications.
% Some of this information can be contradictory, and \emph{The ACS
% Style Guide} sometimes gives more than one option as a model.
% In order to resolve cases where several possibilities are available
% current editions of the \emph{Journal of the American Chemical
% Society} have been consulted; the current consensus there has been
% taken as the correct approach.  In addition to the problem
% of picking the correct style, some of the BibTeX record types are
% difficult to match to standard references in ACS journals.  The
% ``best guess'' has been taken with these.
%
% \subsection{Additional record types}
%
% In general, the database record types supported here follow those
% in the standard BibTeX style files.  Four additional record types
% are provided:
% \begin{description}
%   \item[|patent|] A patent: formatting is similar to other record
%       types.  The data entry for this record type follows the
%       pattern used in |rsc.bst|: |journal| is used to hold
%       the patent type (\emph{e.g.}~``U.S.~Patent''), with the
%       patent number given in |pages|. Whilst this format is
%       non-standard, it is relatively easy to use and implement!
%   \item[|submitted|] Articles submitted to journals but not
%       yet accepted: appends ``submitted'' in a suitable fashion
%       to the entry.
%   \item[|inpress|] Articles in press: appends ``in press'' or,
%       if available, the DOI number assigned to the article.
%   \item[|remark|] A note with no other information to be
%       included.  Output consists purely of the |note| field.
% \end{description}
%
% \subsection{BibTeX database entry requirements}
%
% The requirements for entries in the BibTeX database are slightly
% different using |achemso.bst| to the standard style files. This
% is mainly because some fields are not cited in
% ACS bibliographies.  In particular, journal articles do not
% require a title (the |title| field is ignored).  Articles
% in books and ``collections'' only need the title of the book.
% If a chapter title is given for an |incollection| record, it will
% be printed, but not in the case of an |inbook| record.
%
% \subsection{References to software}
%
% Referencing software is always a little difficult.  The style files
% provided here follow the normal LaTeX convention of using the
% |manual| record type to cite software.  The only requirement is a
% |title|, but fields such as |organization| may be used for more
% detail.  The |edition| field is used to format the software version
% correctly: this will automatically be prefixed with ``version'' by
% the style file.
%
% \subsection{The \texttt{annotate} field}
%
% The standard BibTeX styles use the |note| field for notes to be
% added to the citation.  However, it is common to want personal
% notes about references.  This is catered for using the |annotate|
% field.  The style |achemso| ignores the |annotate| field, whilst
% the |achemsol| style appends the |annotate| information to the
% bibliographic output.  Thus |achemsol| is intended for use in
% database maintenance, whilst |achemso| is for production
% bibliographies.
%
% \DescribeMacro{\refin}
% For use in the |annotate| field the macro \cmd{\refin}
% is defined in |achemso.bst| and |achemsol.bst|.
% The command takes a single argument \marg{text}, and
% gives the output \textbf{Referenced in: text}.
% This command takes one argument (normally text) which is
% preceded by the text ``\textbf{Referenced in:} \meta{text}''.
% The \cmd{\refin} command is intended for tracking citations
% ``backward'' through the database.  For example, this could be
% used to link citations in a database to the writer's own papers.
%
% \subsection{Predefined journal abbreviations}
%
% A number of journal abbreviations are defined in the |.bst| files.
% The abbreviations cover a number ACS journals, several other
% physical chemistry publications and other journals listed as
% highly cited by \emph{Chem.\ Abs.}\ The interested user should
% consult the |.bst| files for full details.
%
% \subsection{\texttt{natbib} support}
%
% As of version 2.2, a |natbib| compatible style file, |achemnat| is
% provided.  The style file provides the appropriate option,
% |natbib|, to load this BibTeX file along with |natbib|, setting up
% the appropriate options.
%
% \section{The LaTeX Package}
%
% The current version of |achemso.sty| is a complete
% re-implementation of the functionality of the original file,
% designed to ensure greater compatibility with other packages. The
% only change for the user is that the bibliography section does
% \emph{not} start a new page when using the |article| document
% class. However, the package now supports all of the standard
% classes, and so the |report| class may be used to ensure a new
% page is started.
%
% \DescribeMacro{\bibliographystyle}
% Loading the |achemso| package adds the appropriate
% \cmd{\bibliographystyle} command to the LaTeX source.  As a result,
% subsequent \cmd{\bibliographystyle} statements will be ignored:
% a suitable warning is given.  The format of citations is altered
% (using the |cite| or |natbib| package as appropriate), and the
% package ensures that the bibliography will be named ``References''
%  in all standard document types.\footnote{This only works if the
% \texttt{babel} package is \emph{not} loaded.  Users wanting a
% system which works with \texttt{babel} should look at the
% \texttt{chemstyle package}. }
%
%  The |achemso| package has five options:
% \begin{description}
%   \item[|note|] If the bibliography contains notes as well
%       as citations, then the section heading should be
%       ``References and Notes''.  This is altered by the
%       |note| package option.
%   \item[|number|] This option numbers the bibliography
%       section (using the |tocbibind| package), and causes it to
%       be entered in the Table of Contents.
%   \item[|list|] This option is intended for creating a listing
%       of the entire BibTeX database.  The BibTeX style is
%       changed to |achemsol|, which will output the additional
%       database field |annotate|, intended for personal notes
%       about a particular database entry.  It also adds the
%       BibTeX key for each citation as a marginal note to the
%       output, using the |showkeys| package.
%    \item[|notsuper|] Switches from superscript citations
%       (\emph{e.g.}~Author \emph{et al.}$^3$) to
%        in-text ones in italics (\emph{e.g.}~Author
%        \emph{et al.}~(\emph{3})). There is a |super|
%        option for completeness, which simply gives the default
%        behaviour.
%     \item[|natbib|] Uses |natbib| rather than |cite| for citation
%        formatting; this also loads the |achemnat| style in place
%        of |achemso|.
% \end{description}
%
% \StopEventually{\bibliography{achemso}}
%
% \section{The Package Code}
%
%  The package code is not very complicated.  For the
%  interested reader(s), it is presented here.
%
% The usual setup code is executed.
%    \begin{macrocode}
%<*package>
\NeedsTeXFormat{LaTeX2e}
\ProvidesPackage{achemso}
  [2007/10/16 v2.2d LaTeX and BibTeX support for American
     Chemical Society publications]
%    \end{macrocode}
% \begin{macro}{\ACS@sctnnmbr}
% \begin{macro}{\ACS@lst}
% \begin{macro}{\ACS@note}
% \changes{v2.0}{2007/01/17}{Boolean values made internal to package}
% \begin{macro}{\ACS@super}
% \changes{v2.1}{2007/02/15}{New Boolean for citation control}
% \begin{macro}{\ACS@natbib}
% \changes{v2.2a}{2007/07/08}{New Boolean for |natbib| support}
% Boolean values are used to handle the options.
%    \begin{macrocode}
\newif \ifACS@sctnnmbr \ACS@sctnnmbrfalse
\newif \ifACS@list     \ACS@listfalse
\newif \ifACS@note     \ACS@notefalse
\newif \ifACS@super    \ACS@supertrue
\newif \ifACS@natbib   \ACS@natbibfalse
%    \end{macrocode}
% \end{macro}
% \end{macro}
% \end{macro}
% \end{macro}
% \end{macro}
% The options are processed.
%\changes{v2.2d}{2007/10/16}{Added \texttt{notes} option}
%    \begin{macrocode}
\DeclareOption{note}{\ExecuteOptions{notes}}
\DeclareOption{notes}{\ACS@notetrue}
\DeclareOption{number}{\ACS@sctnnmbrtrue}
\DeclareOption{super}{\ACS@supertrue}
\DeclareOption{list}{\ACS@listtrue}
\DeclareOption{notsuper}{\ACS@superfalse}
\DeclareOption{natbib}{\ACS@natbibtrue}
\DeclareOption*{\OptionNotUsed}
\ProcessOptions
%    \end{macrocode}
% \changes{v2.1}{2007/02/15}{|cite| package is loaded with different
% options depending on citation style requested}
% \changes{v2.2a}{2007/07/08}{|natbib| support added}
% The |cite| package is loaded to sort and compress references
% correctly. Depending upon the package option given, citations are
% either superscript or italic and in parentheses.
%    \begin{macrocode}
\ifACS@natbib
  \ifACS@super
    \RequirePackage[numbers,sort&compress,super]{natbib}
  \else
%    \end{macrocode}
% For in-line citations with |natbib|, we have to do a
% bit of work to get things to look right.  |natbib| uses
% \cmd{\citenumfont} to format the numbers, but it is not defined
% by default, so we have to use \cmd{\newcommand}.
%    \begin{macrocode}
    \RequirePackage[numbers,sort&compress,round]{natbib}
    \newcommand*{\citenumfont}{\textit}
  \fi
\else
  \ifACS@super
%    \end{macrocode}
%\changes{v2.2c}{2007/08/22}{Use the \texttt{overcite} alias for
%  \texttt{cite} as ACS have very old LaTeX system}
%    \begin{macrocode}
    \RequirePackage[nospace]{overcite}
  \else
%    \end{macrocode}
% Again in-line citations need some format changes.  In the case of
% |cite|, everything is defined initially.  Thus we can use
% \cmd{\renewcommand} for everything.
%    \begin{macrocode}
    \RequirePackage{cite}
    \renewcommand\citeleft{(}
    \renewcommand\citeright{)}
    \renewcommand\citeform[1]{\emph{#1}}
  \fi
\fi
%    \end{macrocode}
% If the |babel| package is loaded, the |note| option does not
% work.  So it is disabled here with a suitable warning.
%    \begin{macrocode}
\@ifpackageloaded{babel}
  {\ACS@notefalse\PackageWarning{achemso}%
    {babel package loaded - note option disabled}}
  {\relax}
%    \end{macrocode}
% \begin{macro}{\ACS@biberror}
% The function \cmd{\ACS@biberror} is defined here to provide an
% easy way of generating a warning if there is no name for a
% bibliography section.  This will only happen with non-standard
% class files.
%     \begin{macrocode}
\def\ACS@biberror{\PackageError{achemso}%
  {No bibliography name command defined}\@eha}
%    \end{macrocode}
% \end{macro}
% \begin{macro}{\refname}
% \begin{macro}{\bibname}
% The |note| option renames the references section to
% ``References and Notes''.  This applies for all standard
% document classes.
% The term ``Bibliography'' is not used in chemistry, the value of
% \cmd{\bibname} is redefined here in all cases where it exists.
%     \begin{macrocode}
\@ifundefined{refname}{%
  \@ifundefined{bibname}{%
    \ACS@biberror
  }{%
    \ifACS@note
      \renewcommand*{\bibname}{References and Notes}
    \else
      \renewcommand*{\bibname}{References}
    \fi
  }
}{%
  \ifACS@note
    \renewcommand*{\refname}{References and Notes}
  \fi
}
%    \end{macrocode}
% \end{macro}
% \end{macro}
% If the |number| option is set, the |tocbibind| package is
% used to number the bibliography.
% \changes{v2.0}{2007/01/17}{Switched to using \texttt{tocbibind}
% to produce number bibliography}
%    \begin{macrocode}
\ifACS@sctnnmbr
  \RequirePackage[numbib]{tocbibind}
\fi
%    \end{macrocode}
% \begin{macro}{\bibliographystyle}
% Depending on the package option, the bibliography style
% will either be |achemso| or |achemsol|.  The later is intended
% for listing the entire database.  The |list| option of the
% package selects this, and for listing also generates boxed
% labels for each reference.  The |showkeys| package provides
% this functionality.  If |natbib| is asked for, then the appropriate
% style files are used in place of the standard ones.
% \changes{v2.0}{2007/01/17}{Replaced custom code with
% \texttt{showkeys} package}
%    \begin{macrocode}
\ifACS@list
  \ifACS@natbib
    \bibliographystyle{achemlnt}
  \else
    \bibliographystyle{achemsol}
  \fi
  \RequirePackage[notcite]{showkeys}
\else
  \ifACS@natbib
    \bibliographystyle{achemnat}
  \else
    \bibliographystyle{achemso}
  \fi
\fi
%    \end{macrocode}
% \end{macro}
% \begin{macro}{\@biblabel}
% In order to re-format the bibliography labels, the easiest
% method is to redefine the \cmd{\@biblabel} macro from the LaTeX
% kernel.
%    \begin{macrocode}
\def\@biblabel#1{#1.}
%    \end{macrocode}
% \end{macro}
% \begin{macro}{\ACS@bibwarning}
% \begin{macro}{\bibliographystyle}
% To ensure that additional \cmd{\bibliographystyle} commands in the
% source are killed off.  The \cmd{\ACS@bibwarning} provides a clean
% method of generating the warning message.
% \changes{v2.0}{2007-01-17}{Command ignored in document body}
%    \begin{macrocode}
\def\ACS@bibwarning{\PackageWarning{achemso}%
  {Additional bibliographystyle command ignored}}
\def\bibliographystyle{\ACS@bibwarning\@gobble}
%    \end{macrocode}
% \end{macro}
% \end{macro}
% The package is complete.
%    \begin{macrocode}
%</package>
%    \end{macrocode}
% \PrintChanges
% \PrintIndex
% \Finale
% \iffalse
%<*bib>
ENTRY
  { address
%<list>    annotate
    author
    booktitle
    chapter
    doi
    edition
    editor
    howpublished
    institution
    journal
%<nat>    key
    note
    number
    organization
    pages
    publisher
    school
    series
    title
    type
    url
    volume
    year
  }
  {}
  {
  label
%<nat>    extra.label
%<nat>    short.list
  }

INTEGERS { output.state before.all mid.sentence after.sentence }
INTEGERS { after.block after.item author.or.editor }
INTEGERS { separate.by.semicolon }

FUNCTION {init.state.consts}
{ #0 'before.all :=
  #1 'mid.sentence :=
  #2 'after.sentence :=
  #3 'after.block :=
  #4 'after.item :=
}

FUNCTION {add.comma}
{ ", " * }

FUNCTION {add.semicolon}
{ "; " * }

%    \end{macrocode}
% For authors/editors we need to be able to add either a semi-colon
% or a comma.  This is done using a switching function, defined here.
%    \begin{macrocode}

FUNCTION {add.comma.or.semicolon}
{ #1 separate.by.semicolon =
    'add.semicolon
    'add.comma
  if$
}

FUNCTION {add.colon}
{ ": " * }

STRINGS { s t }

FUNCTION {output.nonnull}
{ 's :=
  output.state mid.sentence =
    { add.comma write$ }
    { output.state after.block =
      { add.semicolon write$
        newline$
        "\newblock " write$
      }
      { output.state before.all =
          'write$
          { output.state after.item =
            { " " * write$ }
            { add.period$ " " * write$ }
          if$
          }
        if$
        }
      if$
      mid.sentence 'output.state :=
    }
  if$
  s
}

FUNCTION {output}
{ duplicate$ empty$
    'pop$
    'output.nonnull
  if$
}

FUNCTION {output.check}
{ 't :=
  duplicate$ empty$
    { pop$ "Empty " t * " in " * cite$ * warning$ }
    'output.nonnull
  if$
}

%    \end{macrocode}
% For the standard file types, |output.bibitem| can come here.
% The same is not true for styles supporting |natbib|, and so
% |output.bibitem| occurs later for those styles.
% \iffalse
%<*!nat>
% \fi
%    \begin{macrocode}
FUNCTION {output.bibitem}
{ newline$
  "\bibitem{" write$
  cite$ write$
  "}" write$
  newline$
  ""
  before.all 'output.state :=
}

%    \end{macrocode}
% \iffalse
%</!nat>
% \fi
%    \begin{macrocode}
FUNCTION {new.block}
{ output.state before.all =
    'skip$
    { after.block 'output.state := }
  if$
}

FUNCTION {new.sentence}
{ output.state after.block =
    'skip$
    { output.state before.all =
        'skip$
        { after.sentence 'output.state := }
      if$
    }
  if$
}
%<*list>

FUNCTION {add.note}
{ annotate empty$
    'skip$
    { new.block
      "{\footnotesize " annotate * "}" * output }
  if$
}
%</list>

FUNCTION {fin.entry}
%<list>{ add.note
%<list>  add.period$
%<!list>{ add.period$
  write$
  newline$
}
FUNCTION {not}
{   { #0 }
    { #1 }
  if$
}

FUNCTION {and}
{   'skip$
    { pop$ #0 }
  if$
}

FUNCTION {or}
{   { pop$ #1 }
    'skip$
  if$
}

FUNCTION {field.or.null}
{ duplicate$ empty$
    { pop$ "" }
    'skip$
  if$
}

FUNCTION {emphasize}
{ duplicate$ empty$
    { pop$ "" }
    { "\emph{" swap$ * "}" * }
  if$
}

FUNCTION {boldface}
{ duplicate$ empty$
    { pop$ "" }
    { "\textbf{" swap$ * "}" * }
  if$
}

FUNCTION {paren}
{ duplicate$ empty$
    { pop$ "" }
    { "(" swap$ * ")" * }
  if$
}

FUNCTION {bbl.and}
{ "and" }

FUNCTION {bbl.chapter}
{ "Chapter" }

FUNCTION {bbl.editor}
{ "Ed." }

FUNCTION {bbl.editors}
{ "Eds." }

FUNCTION {bbl.edition}
{ "ed." }

FUNCTION {bbl.etal}
{ "et~al." }

FUNCTION {bbl.in}
{ "In" }

FUNCTION {bbl.inpress}
{ "in press" }

FUNCTION {bbl.msc}
{ "M.Sc.\ thesis" }

FUNCTION {bbl.page}
{ "p" }

FUNCTION {bbl.pages}
{ "pp" }

FUNCTION {bbl.phd}
{ "Ph.D.\ thesis" }

FUNCTION {bbl.submitted}
{ "submitted for publication" }

FUNCTION {bbl.techreport}
{ "Technical Report" }

FUNCTION {bbl.version}
{ "version" }

FUNCTION {bbl.volume}
{ "Vol." }

FUNCTION {bbl.first}
{ "1st" }

FUNCTION {bbl.second}
{ "2nd" }

FUNCTION {bbl.third}
{ "3rd" }

FUNCTION {bbl.fourth}
{ "4th" }

FUNCTION {bbl.fifth}
{ "5th" }

FUNCTION {bbl.st}
{ "st" }

FUNCTION {bbl.nd}
{ "nd" }

FUNCTION {bbl.rd}
{ "rd" }

FUNCTION {bbl.th}
{ "th" }

FUNCTION {eng.ord}
{ duplicate$ "1" swap$ *
  #-2 #1 substring$ "1" =
     { bbl.th * }
     { duplicate$ #-1 #1 substring$
       duplicate$ "1" =
         { pop$ bbl.st * }
         { duplicate$ "2" =
             { pop$ bbl.nd * }
             { "3" =
                 { bbl.rd * }
                 { bbl.th * }
               if$
             }
           if$
          }
       if$
     }
   if$
}

FUNCTION{is.a.digit}
{ duplicate$ "" =
    {pop$ #0}
    {chr.to.int$ #48 - duplicate$
     #0 < swap$ #9 > or not}
  if$
}

FUNCTION{is.a.number}
{
  { duplicate$ #1 #1 substring$ is.a.digit }
    {#2 global.max$ substring$}
  while$
  "" =
}

FUNCTION {extract.num}
{ duplicate$ 't :=
  "" 's :=
  { t empty$ not }
  { t #1 #1 substring$
    t #2 global.max$ substring$ 't :=
    duplicate$ is.a.number
      { s swap$ * 's := }
      { pop$ "" 't := }
    if$
  }
  while$
  s empty$
    'skip$
    { pop$ s }
  if$
}

FUNCTION {bibinfo.check}
{ swap$
  duplicate$ missing$
    { pop$ pop$
      ""
    }
    { duplicate$ empty$
        {
          swap$ pop$
        }
        { swap$
          pop$
        }
      if$
    }
  if$
}

FUNCTION {convert.edition}
{ extract.num "l" change.case$ 's :=
  s "first" = s "1" = or
    { bbl.first 't := }
    { s "second" = s "2" = or
        { bbl.second 't := }
        { s "third" = s "3" = or
            { bbl.third 't := }
            { s "fourth" = s "4" = or
                { bbl.fourth 't := }
                { s "fifth" = s "5" = or
                    { bbl.fifth 't := }
                    { s #1 #1 substring$ is.a.number
                        { s eng.ord 't := }
                        { edition 't := }
                      if$
                    }
                  if$
                }
              if$
            }
          if$
        }
      if$
    }
  if$
  t
}

FUNCTION {tie.or.space.connect}
{ duplicate$ text.length$ #3 <
    { "~" }
    { " " }
  if$
  swap$ * *
}

FUNCTION {space.connect}
{ " " swap$ * * }

INTEGERS { nameptr namesleft numnames }

FUNCTION {format.names}
{ 's :=
  #1 'nameptr :=
  s num.names$ 'numnames :=
  numnames 'namesleft :=
  numnames #15 >
    { s #1 "{vv~}{ll,}{~f.}{,~jj}" format.name$ 't :=
      t bbl.etal space.connect
    }
    {
       { namesleft #0 > }
       { s nameptr "{vv~}{ll,}{~f.}{,~jj}" format.name$ 't :=
           nameptr #1 >
             { namesleft #1 >
               { add.comma.or.semicolon t * }
               { numnames #2 >
                 { "" * }
                 'skip$
               if$
               t "others," =
                 { bbl.etal space.connect }
                 { add.comma.or.semicolon t * }
               if$
               }
             if$
             }
           't
         if$
         nameptr #1 + 'nameptr :=
         namesleft #1 - 'namesleft :=
         }
     while$
  }
  if$
}

FUNCTION {format.authors}
{ author empty$
    { "" }
    { #1 'author.or.editor :=
      #1 'separate.by.semicolon :=
      author format.names
    }
  if$
}

FUNCTION {format.editors}
{ editor empty$
    { "" }
    { #2 'author.or.editor :=
      #0 'separate.by.semicolon :=
      editor format.names
      add.comma
      editor num.names$ #1 >
        { bbl.editors }
        { bbl.editor }
      if$
      *
    }
  if$
}

FUNCTION {n.separate.multi}
{ 't :=
  ""
  #0 'numnames :=
  t text.length$ #4 > t is.a.number and
    {
      { t empty$ not }
      { t #-1 #1 substring$ is.a.number
          { numnames #1 + 'numnames := }
          { #0 'numnames := }
        if$
        t #-1 #1 substring$ swap$ *
        t #-2 global.max$ substring$ 't :=
        numnames #4 =
          { duplicate$ #1 #1 substring$ swap$
            #2 global.max$ substring$
            "," swap$ * *
            #1 'numnames :=
          }
          'skip$
        if$
      }
      while$
    }
    { t swap$ * }
  if$
}

FUNCTION {format.bvolume}
{ volume empty$
    { "" }
    { bbl.volume volume tie.or.space.connect }
  if$
}

FUNCTION {format.title.noemph}
{ 't :=
  t empty$
    { "" }
    { t }
  if$
}

FUNCTION {format.title}
{ 't :=
  t empty$
    { "" }
    { t emphasize }
  if$
}

FUNCTION {format.number.series}
{ volume empty$
    { number empty$
       { series field.or.null }
       { series empty$
         { "There is a number but no series in " cite$ * warning$ }
         { series number space.connect }
       if$
       }
      if$
    }
    { "" }
  if$
}

FUNCTION {format.url}
{ url empty$
    { "" }
    { new.sentence "\url{" url * "}" * }
  if$
}

% The specialised |output.bibitem| needed for |natbib| support now
% follows, along with the various support macros it needs.
% \iffalse
%<*nat>
% \fi
%    \begin{macrocode}
FUNCTION {format.full.names}
{'s :=
  #1 'nameptr :=
  s num.names$ 'numnames :=
  numnames 'namesleft :=
    { namesleft #0 > }
    { s nameptr
      "{vv~}{ll}" format.name$ 't :=
      nameptr #1 >
        {
          namesleft #1 >
            { ", " * t * }
            {
              numnames #2 >
                { "," * }
                'skip$
              if$
              t "others" =
                { bbl.etal * }
                { bbl.and space.connect t space.connect }
              if$
            }
          if$
        }
        't
      if$
      nameptr #1 + 'nameptr :=
      namesleft #1 - 'namesleft :=
    }
  while$
}

FUNCTION {author.editor.full}
{ author empty$
    { editor empty$
        { "" }
        { editor format.full.names }
      if$
    }
    { author format.full.names }
  if$
}

FUNCTION {author.full}
{ author empty$
    { "" }
    { author format.full.names }
  if$
}

FUNCTION {editor.full}
{ editor empty$
    { "" }
    { editor format.full.names }
  if$
}

FUNCTION {make.full.names}
{ type$ "book" =
  type$ "inbook" =
  or
    'author.editor.full
    { type$ "proceedings" =
        'editor.full
        'author.full
      if$
    }
  if$
}

FUNCTION {output.bibitem} { newline$
  "\bibitem[" write$
  label write$
  ")" make.full.names duplicate$ short.list =
     { pop$ }
     { * }
   if$
  "]{" * write$
  cite$ write$
  "}" write$
  newline$
  ""
  before.all 'output.state :=
}

%    \end{macrocode}
% \iffalse
%</nat>
% \fi
%    \begin{macrocode}
FUNCTION {n.dashify}
{ 't :=
  ""
    { t empty$ not }
    { t #1 #1 substring$ "-" =
    { t #1 #2 substring$ "--" = not
        { "--" *
          t #2 global.max$ substring$ 't :=
        }
        {   { t #1 #1 substring$ "-" = }
        { "-" *
          t #2 global.max$ substring$ 't :=
        }
          while$
        }
      if$
    }
    { t #1 #1 substring$ *
      t #2 global.max$ substring$ 't :=
    }
      if$
    }
  while$
}

FUNCTION {format.date}
{ year empty$
    { "" }
    { year boldface }
  if$
}

FUNCTION {format.bdate}
{ year empty$
    { "There's no year in " cite$ * warning$ }
    'year
  if$
}

FUNCTION {either.or.check}
{ empty$
    'pop$
    { "Can't use both " swap$ * " fields in " * cite$ * warning$ }
  if$
}

FUNCTION {format.edition}
{ edition duplicate$ empty$
    'skip$
    { convert.edition
      bbl.edition bibinfo.check
      " " * bbl.edition *
    }
  if$
}

INTEGERS { multiresult }

FUNCTION {multi.page.check}
{ 't :=
  #0 'multiresult :=
    { multiresult not
      t empty$ not
      and
    }
    { t #1 #1 substring$
      duplicate$ "-" =
      swap$ duplicate$ "," =
      swap$ "+" =
      or or
        { #1 'multiresult := }
        { t #2 global.max$ substring$ 't := }
      if$
    }
  while$
  multiresult
}

FUNCTION {format.pages}
{ pages empty$
    { "" }
    { pages multi.page.check
      { bbl.pages pages n.dashify tie.or.space.connect }
      { bbl.page pages tie.or.space.connect }
    if$
    }
  if$
}


FUNCTION {format.pages.required}
{ pages empty$
    { ""
      "There are no page numbers for " cite$ * warning$
      output
    }
    { pages multi.page.check
      { bbl.pages pages n.dashify tie.or.space.connect }
      { bbl.page pages tie.or.space.connect }
    if$
    }
  if$
}

FUNCTION {format.pages.nopp}
{ pages empty$
    { ""
      "There are no page numbers for " cite$ * warning$
      output
    }
    { pages multi.page.check
      { pages n.dashify space.connect }
      { pages space.connect }
    if$
    }
  if$
}

FUNCTION {format.pages.patent}
{ pages empty$
    { "There is no patent number for " cite$ * warning$ }
    { pages multi.page.check
      { pages n.dashify }
      { pages n.separate.multi }
      if$
    }
  if$
}

FUNCTION {format.vol.pages}
{ volume emphasize field.or.null
  duplicate$ empty$
    { pop$ format.pages.required }
    { add.comma pages n.dashify * }
  if$
}

FUNCTION {format.chapter.pages}
{ chapter empty$
    'format.pages
    { type empty$
    { bbl.chapter }
    { type "l" change.case$ }
      if$
      chapter tie.or.space.connect
      pages empty$
    'skip$
    { add.comma format.pages * }
      if$
    }
  if$
}

FUNCTION {format.title.in}
{ 's :=
  s empty$
    { "" }
    { editor empty$
      { bbl.in s format.title space.connect }
      { bbl.in s format.title space.connect
        add.semicolon format.editors *
      }
    if$
    }
  if$
}

FUNCTION {format.pub.address}
{ publisher empty$
    { "" }
    { address empty$
        { publisher }
        { publisher add.colon address *}
      if$
    }
  if$
}

FUNCTION {format.school.address}
{ school empty$
    { "" }
    { address empty$
        { school }
        { school add.colon address *}
      if$
    }
  if$
}

FUNCTION {format.organization.address}
{ organization empty$
    { "" }
    { address empty$
        { organization }
        { organization add.colon address *}
      if$
    }
  if$
}

FUNCTION {format.version}
{ edition empty$
    { "" }
    { bbl.version edition tie.or.space.connect }
  if$
}

FUNCTION {empty.misc.check}
{ author empty$ title empty$ howpublished empty$
  year empty$ note empty$ url empty$
  and and and and and
    { "all relevant fields are empty in " cite$ * warning$ }
    'skip$
  if$
}

FUNCTION {empty.doi.note}
{ doi empty$ note empty$ and
    { "Need either a note or DOI for " cite$ * warning$ }
    'skip$
  if$
}

FUNCTION {format.thesis.type}
{ type empty$
    'skip$
    { pop$
      type emphasize
    }
  if$
}

FUNCTION {article}
{ output.bibitem
  format.authors "author" output.check
  after.item 'output.state :=
  journal emphasize "journal" output.check
  after.item 'output.state :=
  format.date "year" output.check
  volume empty$
    { ""
      format.pages.nopp output
    }
    { format.vol.pages output }
  if$
  note output
  fin.entry
}

FUNCTION {book}
{ output.bibitem
  author empty$
    { booktitle empty$
        { title format.title "title" output.check }
        { booktitle format.title "booktitle" output.check }
      if$
      format.edition output
      new.block
      editor empty$
        { "Need either an author or editor for " cite$ * warning$ }
        { "" format.editors * "editor" output.check }
      if$
    }
    { format.authors output
      after.item 'output.state :=
      "author and editor" editor either.or.check
      booktitle empty$
        { title format.title "title" output.check }
        { booktitle format.title "booktitle" output.check }
      if$
      format.edition output
    }
  if$
  new.block
  format.number.series output
  new.block
  format.pub.address "publisher" output.check
  format.bdate "year" output.check
  new.block
  format.bvolume output
  pages empty$
    'skip$
    { format.pages output }
  if$
  note output
  fin.entry
}

FUNCTION {booklet}
{ output.bibitem
  format.authors output
  after.item 'output.state :=
  title format.title "title" output.check
  howpublished output
  address output
  format.date output
  note output
  fin.entry
}

FUNCTION {inbook}
{ output.bibitem
  author empty$
    { title format.title "title" output.check
      format.edition output
      new.block
      editor empty$
      { "Need at least an author or an editor for " cite$ * warning$ }
      { "" format.editors * "editor" output.check }
    if$
    }
    { format.authors output
      after.item 'output.state :=
      title format.title.in "title" output.check
      format.edition output
    }
  if$
  new.block
  format.number.series output
  new.block
  format.pub.address "publisher" output.check
  format.bdate "year" output.check
  new.block
  format.bvolume output
  format.chapter.pages "chapter and pages" output.check
  note output
  fin.entry
}

FUNCTION {incollection}
{ output.bibitem
  author empty$
    { booktitle format.title "booktitle" output.check
      format.edition output
      new.block
      editor empty$
        { "Need at least an author or an editor for " cite$ * warning$ }
        { "" format.editors * "editor" output.check }
      if$
    }
    { format.authors output
      after.item 'output.state :=
      title empty$
        'skip$
        { title format.title.noemph output }
      if$
      after.sentence 'output.state :=
      booktitle format.title.in "booktitle" output.check
      format.edition output
    }
  if$
  new.block
  format.number.series output
  new.block
  format.pub.address "publisher" output.check
  format.bdate "year" output.check
  new.block
  format.bvolume output
  format.chapter.pages "chapter and pages" output.check
  note output
  fin.entry
}

FUNCTION {inpress}
{ output.bibitem
  format.authors "author" output.check
  after.item 'output.state :=
  journal emphasize "journal" output.check
  doi empty$
    {  bbl.inpress output }
    {  after.item 'output.state :=
       format.date output
       "DOI:" doi tie.or.space.connect output
    }
  if$
  note output
  fin.entry
}

FUNCTION {inproceedings}
{ output.bibitem
  format.authors "author" output.check
  after.item 'output.state :=
  title empty$
    'skip$
    { title format.title.noemph output
      after.sentence 'output.state :=
    }
  if$
  booktitle format.title output
  address output
  format.bdate "year" output.check
  pages empty$
    'skip$
    { new.block
      format.pages output }
  if$
  note output
  fin.entry
}

FUNCTION {manual}
{ output.bibitem
  format.authors output
  after.item 'output.state :=
  title format.title "title" output.check
  format.version output
  new.block
  format.organization.address output
  format.bdate output
  note output
  fin.entry
}

FUNCTION {mastersthesis}
{ output.bibitem
  format.authors "author" output.check
  after.item 'output.state :=
  bbl.msc format.thesis.type output
  format.school.address "school" output.check
  format.bdate "year" output.check
  note output
  fin.entry
}

FUNCTION {misc}
{ output.bibitem
  format.authors output
  after.item 'output.state :=
  title empty$
    'skip$
    { title format.title output }
  if$
  howpublished output
  year output
  format.url output
  note output
  fin.entry
  empty.misc.check
}

FUNCTION {patent}
{ output.bibitem
  format.authors "author" output.check
  after.item 'output.state :=
  journal "journal" output.check
  after.item 'output.state :=
  format.pages.patent "pages" output.check
  format.bdate "year" output.check
  note output
  fin.entry
}

FUNCTION {phdthesis}
{ output.bibitem
  format.authors "author" output.check
  after.item 'output.state :=
  bbl.phd format.thesis.type output
  format.school.address "school" output.check
  format.bdate "year" output.check
  note output
  fin.entry
}

FUNCTION {proceedings}
{ output.bibitem
  title format.title.noemph "title" output.check
  address output
  format.bdate "year" output.check
  pages empty$
    'skip$
    { new.block
      format.pages output }
  if$
  note output
  fin.entry
}

FUNCTION {remark}
{ output.bibitem
  note "note" output.check
  fin.entry
}

FUNCTION {submitted}
{ output.bibitem
  format.authors "author" output.check
  bbl.submitted output
  fin.entry
}

FUNCTION {techreport}
{ output.bibitem
  format.authors "author" output.check
  after.item 'output.state :=
  title format.title "title" output.check
  new.block
  type empty$
    'bbl.techreport
    'type
  if$
  number empty$
    'skip$
    { number tie.or.space.connect }
  if$
  output
  format.pub.address output
  format.bdate "year" output.check
  pages empty$
    'skip$
    { new.block
      format.pages output }
  if$
  note output
  fin.entry
}

FUNCTION {unpublished}
{ output.bibitem
  format.authors "author" output.check
  after.item 'output.state :=
  journal empty$
    'skip$
    { journal emphasize "journal" output.check }
  if$
  doi empty$
    {  note output }
    {  after.item 'output.state :=
       format.date output
       "DOI:" doi tie.or.space.connect output
    }
  if$
  fin.entry
  empty.doi.note
}

FUNCTION {conference} {inproceedings}

FUNCTION {other} {patent}

FUNCTION {default.type} {misc}

MACRO {jan} {"Jan."}
MACRO {feb} {"Feb."}
MACRO {mar} {"Mar."}
MACRO {apr} {"Apr."}
MACRO {may} {"May"}
MACRO {jun} {"June"}
MACRO {jul} {"July"}
MACRO {aug} {"Aug."}
MACRO {sep} {"Sept."}
MACRO {oct} {"Oct."}
MACRO {nov} {"Nov."}
MACRO {dec} {"Dec."}

MACRO {acchemr} {"Acc.\ Chem.\ Res."}
MACRO {aacsa} {"Adv.\ {ACS} Abstr."}
MACRO {anchem} {"Anal.\ Chem."}
MACRO {bioch} {"Biochemistry"}
MACRO {bicoc} {"Bioconj.\ Chem."}  % ***
MACRO {bitech} {"Biotechnol.\ Progr."}  % ***
MACRO {chemeng} {"Chem.\ Eng.\ News"}
MACRO {chs} {"Chem.\ Health Safety"} % ***
MACRO {crt} {"Chem.\ Res.\ Toxicol."} % ***
MACRO {chemrev} {"Chem.\ Rev."} % ***
MACRO {cmat} {"Chem.\ Mater."} % ***
MACRO {chemtech} {"{CHEMTECH}"} % ***
MACRO {enfu} {"Energy Fuels"} % ***
MACRO {envst} {"Environ.\ Sci.\ Technol."}
MACRO {iecf} {"Ind.\ Eng.\ Chem.\ Fundam."}
MACRO {iecpdd} {"Ind.\ Eng.\ Chem.\ Proc.\ Des.\ Dev."}
MACRO {iecprd} {"Ind.\ Eng.\ Chem.\ Prod.\ Res.\ Dev."}
MACRO {iecr} {"Ind.\ Eng.\ Chem.\ Res."} % ***
MACRO {inor} {"Inorg.\ Chem."}
MACRO {jafc} {"J.~Agric.\ Food Chem."}
MACRO {jacs} {"J.~Am.\ Chem.\ Soc."}
MACRO {jced} {"J.~Chem.\ Eng.\ Data"}
MACRO {jcics} {"J.~Chem.\ Inf.\ Comput.\ Sci."}
MACRO {jmc} {"J.~Med.\ Chem."}
MACRO {joc} {"J.~Org.\ Chem."}
MACRO {jps} {"J.~Pharm.\ Sci."}
MACRO {jpcrd} {"J.~Phys.\ Chem.\ Ref.\ Data"} % ***
MACRO {jpc} {"J.~Phys.\ Chem."}
MACRO {jpca} {"J.~Phys.\ Chem.~A"}
MACRO {jpcb} {"J.~Phys.\ Chem.~B"}
MACRO {lang} {"Langmuir"}
MACRO {macro} {"Macromolecules"}
MACRO {orgmet} {"Organometallics"}
MACRO {orglett} {"Org.\ Lett."}

MACRO {jft} {"J.~Chem.\ Soc., Faraday Trans."}
MACRO {jft1} {"J.~Chem.\ Soc., Faraday Trans.\ 1"}
MACRO {jft2} {"J.~Chem.\ Soc., Faraday Trans.\ 2"}
MACRO {tfs} {"Trans.\ Faraday Soc."}
MACRO {jcis} {"J.~Colloid Interface Sci."}
MACRO {acis} {"Adv.~Colloid Interface Sci."}
MACRO {cs} {"Colloids Surf."}
MACRO {csa} {"Colloids Surf.\ A"}
MACRO {csb} {"Colloids Surf.\ B"}
MACRO {pcps} {"Progr.\ Colloid Polym.\ Sci."}
MACRO {jmr} {"J.~Magn.\ Reson."}
MACRO {jmra} {"J.~Magn.\ Reson.\ A"}
MACRO {jmrb} {"J.~Magn.\ Reson.\ B"}
MACRO {sci} {"Science"}
MACRO {nat} {"Nature"}
MACRO {jcch} {"J.~Comput.\ Chem."}
MACRO {cca} {"Croat.\ Chem.\ Acta"}
MACRO {angew} {"Angew.\ Chem., Int.\ Ed."}
MACRO {chemeurj} {"Chem.---Eur.\ J."}

MACRO {poly} {"Polymer"}
MACRO {ajp} {"Am.\ J.\ Phys."}
MACRO {rsi} {"Rev.\ Sci.\ Instrum."}
MACRO {jcp} {"J.~Chem.\ Phys."}
MACRO {cpl} {"Chem.\ Phys.\ Lett."}
MACRO {molph} {"Mol.\ Phys."}
MACRO {pac} {"Pure Appl.\ Chem."}
MACRO {jbc} {"J.~Biol.\ Chem."}
MACRO {tl} {"Tetrahedron Lett."}
MACRO {psisoe} {"Proc.\ SPIE-Int.\ Soc.\ Opt.\ Eng."}
MACRO {prb} {"Phys.\ Rev.\ B:\ Condens.\ Matter Mater. Phys."}
MACRO {jap} {"J.~Appl.\ Phys."}
MACRO {pnac} {"Proc.\ Natl.\ Acad.\ Sci.\ U.S.A."}
MACRO {bba} {"Biochim.\ Biophys.\ Acta"}
MACRO {nar} {"Nucleic.\ Acid Res."}

READ

%    \end{macrocode}
% The nature of the initialise code depends on whether we need to
% support |natbib|.  First the simple case is handled.
% \iffalse
%<*!nat>
% \fi
%    \begin{macrocode}
STRINGS { longest.label }

INTEGERS { number.label longest.label.width }

FUNCTION {initialize.longest.label}
{ "" 'longest.label :=
  #1 'number.label :=
  #0 'longest.label.width :=
}

FUNCTION {longest.label.pass}
{ number.label int.to.str$ 'label :=
  number.label #1 + 'number.label :=
  label width$ longest.label.width >
    { label 'longest.label :=
      label width$ 'longest.label.width :=
    }
    'skip$
  if$
}

EXECUTE {initialize.longest.label}

ITERATE {longest.label.pass}

%    \end{macrocode}
% \iffalse
%</!nat>
% \fi
% Now the |natbib| system is sorted out, basically by copying from
% |plainnat.bst|.
% \iffalse
%<*nat>
% \fi
%    \begin{macrocode}
INTEGERS { len }

FUNCTION {chop.word}
{ 's :=
  'len :=
  s #1 len substring$ =
    { s len #1 + global.max$ substring$ }
    's
  if$
}

FUNCTION {format.lab.names}
{ 's :=
  s #1 "{vv~}{ll}" format.name$
  s num.names$ duplicate$
  #2 >
    { pop$ bbl.etal space.connect }
    { #2 <
        'skip$
        { s #2 "{ff }{vv }{ll}{ jj}" format.name$ "others" =
            { bbl.etal space.connect }
            { bbl.and space.connect s #2 "{vv~}{ll}" format.name$ space.connect }
          if$
        }
      if$
    }
  if$
}

FUNCTION {author.key.label}
{ author empty$
    { key empty$
        { cite$ #1 #3 substring$ }
        'key
      if$
    }
    { author format.lab.names }
  if$
}

FUNCTION {author.editor.key.label}
{ author empty$
    { editor empty$
        { key empty$
            { cite$ #1 #3 substring$ }
            'key
          if$
        }
        { editor format.lab.names }
      if$
    }
    { author format.lab.names }
  if$
}

FUNCTION {author.key.organization.label}
{ author empty$
    { key empty$
        { organization empty$
            { cite$ #1 #3 substring$ }
            { "The " #4 organization chop.word #3 text.prefix$ }
          if$
        }
        'key
      if$
    }
    { author format.lab.names }
  if$
}

FUNCTION {editor.key.organization.label}
{ editor empty$
    { key empty$
        { organization empty$
            { cite$ #1 #3 substring$ }
            { "The " #4 organization chop.word #3 text.prefix$ }
          if$
        }
        'key
      if$
    }
    { editor format.lab.names }
  if$
}

FUNCTION {calc.short.authors}
{ type$ "book" =
  type$ "inbook" =
  or
    'author.editor.key.label
    { type$ "proceedings" =
        'editor.key.organization.label
        { type$ "manual" =
            'author.key.organization.label
            'author.key.label
          if$
        }
      if$
    }
  if$
  'short.list :=
}

FUNCTION {calc.label}
{ calc.short.authors
  short.list
  "("
  *
  year duplicate$ empty$
  short.list key field.or.null = or
     { pop$ "" }
     'skip$
  if$
  *
  'label :=
}

ITERATE {calc.label}

STRINGS { longest.label last.label next.extra }

INTEGERS { longest.label.width last.extra.num number.label }

FUNCTION {initialize.longest.label}
{ "" 'longest.label :=
  #0 int.to.chr$ 'last.label :=
  "" 'next.extra :=
  #0 'longest.label.width :=
  #0 'last.extra.num :=
  #0 'number.label :=
}

FUNCTION {forward.pass}
{ last.label label =
    { last.extra.num #1 + 'last.extra.num :=
      last.extra.num int.to.chr$ 'extra.label :=
    }
    { "a" chr.to.int$ 'last.extra.num :=
      "" 'extra.label :=
      label 'last.label :=
    }
  if$
  number.label #1 + 'number.label :=
}

EXECUTE {initialize.longest.label}

ITERATE {forward.pass}

%    \end{macrocode}
% \iffalse
%</nat>
% \fi

FUNCTION {begin.bib}
{ preamble$ empty$
    'skip$
    { preamble$ write$ newline$ }
  if$
  "\providecommand{\url}[1]{\texttt{#1}}"
  write$ newline$
  "\providecommand{\refin}[1]{\\ \textbf{Referenced in:} #1}"
  write$ newline$
%<nat>  "\providecommand{\natexlab}[1]{#1}"
%<nat>  write$ newline$
%<!nat>  "\begin{thebibliography}{"  longest.label  * "}" *
%<nat>  "\begin{thebibliography}{" number.label int.to.str$ * "}" *
  write$ newline$
}

EXECUTE {begin.bib}

EXECUTE {init.state.consts}

ITERATE {call.type$}

FUNCTION {end.bib}
{ newline$
  "\end{thebibliography}" write$ newline$
}

EXECUTE {end.bib}
%</bib>
%<*database>
@BOOK{Coghill2006,
  title = {{T}he {ACS} {S}tyle {G}uide},
  publisher = {{O}xford {U}niversity {P}ress, {I}nc. and
               {T}he {A}merican {C}hemical {S}ociety},
  year = {2006},
  editor = {Coghill, Anne M. and Garson, Lorrin R.},
  address = {{N}ew {Y}ork},
  edition = {3},
  subtitle = {{E}ffective {C}ommunication of {S}cientific {I}nformation},
}

@MISC{ACS2007,
  url = {http://pubs.acs.org/books/references.shtml},
}
%</database>
%<*jawltxdoc>
% The following is convenient method for collecting together package
% loading, formatting commands and new macros used to format |dtx|
% files written by the current author.  It is based on the similar
% files provided by Will Robertson in his packages and Heiko Oberdiek
% as a stand-alone package.  Notice that it is not intended for other
% users: there is no error checking!  However, it is covered by the
% LPPL in the same way as the rest of this package.
%
\NeedsTeXFormat{LaTeX2e}
\ProvidesPackage{jawltxdoc}
  [2007/10/14 v1.0b]
% First of all, a number of support packages are loaded.
\usepackage[T1]{fontenc}
\usepackage[english,UKenglish]{babel}
\usepackage[scaled=0.95]{helvet}
\usepackage[version=3]{mhchem}
\usepackage[final]{microtype}
\usepackage[osf]{mathpazo}
\usepackage{booktabs,array,url,graphicx,courier,unitsdef}
\usepackage{upgreek,ifpdf,listings}
% If using PDFLaTeX, the source will be attached to the PDF. This
% is basically the system used by Heiko Oberdiek, but with a check
% that PDF mode is enabled.
\ifpdf
  \usepackage{embedfile}
  \embedfile[%
    stringmethod=escape,%
    mimetype=plain/text,%
    desc={LaTeX docstrip source archive for package `\jobname'}%
    ]{\jobname.dtx}
\fi
\usepackage{\jobname}
\usepackage[numbered]{hypdoc}
%
% To typeset examples, a new environment is needed.  The code below
% is based on that in used by |listings|, but is modified to get
% better formatting for this context.  The formatting of the output
% is basically that in Will Robertson's |dtx-style| file.
\newlength\LaTeXwidth
\newlength\LaTeXoutdent
\newlength\LaTeXgap
\setlength\LaTeXgap{1em}
\setlength\LaTeXoutdent{-0.15\textwidth}
\def\typesetexampleandcode{%
  \begin{list}{}{%
    \setlength\itemindent{0pt}
    \setlength\leftmargin\LaTeXoutdent
    \setlength\rightmargin{0pt}
  }
  \item
    \setlength\LaTeXoutdent{-0.15\textwidth}
    \begin{minipage}[c]{\textwidth-\LaTeXwidth-\LaTeXoutdent-\LaTeXgap}
      \lst@sampleInput
    \end{minipage}%
    \hfill%
    \begin{minipage}[c]{\LaTeXwidth}%
      \hbox to\linewidth{\box\lst@samplebox\hss}%
    \end{minipage}%
  \end{list}
}
\def\typesetcodeandexample{%
  \begin{list}{}{%
    \setlength\itemindent{0pt}
    \setlength\leftmargin{0pt}
    \setlength\rightmargin{0pt}
  }
  \item
    \begin{minipage}[c]{\LaTeXwidth}%
      \hbox to\linewidth{\box\lst@samplebox\hss}%
    \end{minipage}%
    \lst@sampleInput
  \end{list}
}
\def\typesetfloatexample{%
  \begin{list}{}{%
    \setlength\itemindent{0pt}
    \setlength\leftmargin{0pt}
    \setlength\rightmargin{0pt}
  }
  \item
    \lst@sampleInput
    \begin{minipage}[c]{\LaTeXwidth}%
      \hbox to\linewidth{\box\lst@samplebox\hss}%
    \end{minipage}%
  \end{list}
}
\def\typesetcodeonly{%
  \begin{list}{}{%
    \setlength\itemindent{0pt}
    \setlength\leftmargin{0pt}
    \setlength\rightmargin{0pt}
  }
  \item
    \begin{minipage}[c]{\LaTeXwidth}%
      \hbox to\linewidth{\box\lst@samplebox\hss}%
    \end{minipage}%
  \end{list}
}
\edef\LaTeXexamplefile{\jobname.tmp}
\lst@RequireAspects{writefile}
\newbox\lst@samplebox
\lstnewenvironment{LaTeXexample}[1][\typesetexampleandcode]{%
  \let\typesetexample#1
  \global\let\lst@intname\@empty
  \setbox\lst@samplebox=\hbox\bgroup
  \setkeys{lst}{language=[LaTeX]{TeX},tabsize=4,gobble=2,%
    breakindent=0pt,basicstyle=\small\ttfamily,basewidth=0.51em,%
    keywordstyle=\color{blue},%
% Notice that new keywords should be added here. The list is simply
% macro names needed to typeset documentation of the package
% author.
    morekeywords={bibnote,citenote,bibnotetext,bibnotemark,%
      thebibnote,bibnotename,includegraphics,schemeref,%
      floatcontentsleft,floatcontentsright,floatcontentscentre,%
      schemerefmarker,compound,schemerefformat,color,%
      startchemical,stopchemical,chemical,setupchemical,bottext,%
      listofschemes}}
  \lst@BeginAlsoWriteFile{\LaTeXexamplefile}
}{%
  \lst@EndWriteFile\egroup
  \setlength\LaTeXwidth{\wd\lst@samplebox}
  \typesetexample%
}
\def\lst@sampleInput{%
  \MakePercentComment\catcode`\^^M=10\relax
  \small%
  {\setkeys{lst}{SelectCharTable=\lst@ReplaceInput{\^\^I}%
    {\lst@ProcessTabulator}}%
    \leavevmode \input{\LaTeXexamplefile}}%
  \MakePercentIgnore%
}
\hyphenation{PDF-LaTeX}
%</jawltxdoc>
%\fi
