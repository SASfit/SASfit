\section{Other functions}

\subsection{DLS\_Sphere\_RDG}  ~\\

This function has been implemented to simulate the relaxation
signal $g_2(t)-1 = g_1^2(t)$ of a DLS (dynamic light scattering)
measurement. The $Q$ dependent contribution to the relaxation
signal by particles of different radius $R$ is considered by
weighting $g_2(t)-1$ with the form factor of sperical particles in
Raylay-Debye-Gans approximation:
\begin{align}
   I_\text{DLS\_Sphere\_RDG} (t,\eta,T,Q) &=  \int\limits_0^\infty N(R) \; K^2_\text{sp}(Q,R)
   \: e^{-D Q^2t} \; dR
\end{align}
with
\begin{align}
   D        & = \frac{k_BT}{6\pi\eta R} \nonumber \\
   K_\text{sp}(Q,R) & = \frac{4}{3}\pi R^3  \, 3 \frac{\sin QR - QR \cos
   QR}{(QR)^3}\nonumber\\
   R & : \text {radius} \nonumber \\
   T & : \text {temperature} \nonumber \\
   \eta & : \text {viscosity} \nonumber \\
   Q & : \text {scattering vector} \nonumber
\end{align}

%%%%%%%%%%%%%%%%%%%%%%%%%%%%%%%%%%%%%%%%%%%%%%%%%%%%%%%%%%%%%%%%%%%%%%%%%%%%%%%%%%%%%

\clearpage
\subsection{Langevin}  ~\\

This function has been implemented to simulate the magnetisation
curve $M(B)$ of superparamagnetic particles following the Langevin
statistics with a size distribution $N(R)$: Magnetization curve of
superparamagnetic particles with magnetization a $M_s$ at
temperature $T$
\begin{align}
M(B) & = \dfrac{\int\limits_0^\infty N(R)\, \tfrac{4}{3}\pi R^3\,
M_{p}(B,R) \; dR}{\int\limits_0^\infty N(R) \tfrac{4}{3}\pi
R^3\;dR}
\\[3mm]
M_{p}(B,R)  & = M_\infty\left(\coth(\alpha) - \tfrac{1}{\alpha}\right)  \nonumber \\
\alpha       & = \dfrac{B M_s \tfrac{4}{3}\pi R^3}{k_B T}
\nonumber
\end{align}

%%%%%%%%%%%%%%%%%%%%%%%%%%%%%%%%%%%%%%%%%%%%%%%%%%%%%%%%%%%%%%%%%%%%%%%%%%%%%%%%%%%%%%

\clearpage
\subsection{SuperParStroboPsi}  ~\\

In the following the scattering of polarized incident neutrons with polarization analysis (POLARIS) is described.
Experimental measured scattering signal are always a mixture of the spin dependent scattering cross-sections
$I^{\pm\pm}(\BM{Q})$ and $I^{\pm\mp}(\BM{Q})$. The relative contribution of these cross-sections to the measured cross-section
depend on the polarisation of the polarizer $P_{pol}$ the efficiency of the spin flipper $\epsilon$ and the
transmission values $T_\pm$ of the polarization analyzer, which is assumed to be an opaque He-filter.
If the neutron polarization of the polarizer is $P_{pol}\in [-1;1]$ and the spin flipper is off the incident
polarisation on the sample $P_{in}$ is given by
\begin{subequations}
\begin{align}
P_{in} &= P_{pol}= \frac{N_+-N_-}{N_++N_-}=n_+-n_-
\end{align}
with
\begin{align}
n_+&=\frac{N_+}{N_++N_-} \text{ and } n_- =\frac{N_-}{N_++N_-} \\
\Rightarrow n_+ &= \frac{1+P_{pol}}{2} \text{ and } n_- =\frac{1-P_{pol}}{2}
\end{align}
\end{subequations}
$N_\pm$ are the number of incident neutrons with spin polarizations up ($+$) and down ($-$).
After switching on the spin flipper, which works with an efficiency of $\epsilon \in [0;1]$
($\epsilon=0$: flipper off, $\epsilon\lessapprox 1$ : flipper on), one gets
\begin{align}
\begin{split}
n_+(P_{pol},\epsilon) &= \epsilon \;
\frac{1-P_{pol}}{2}+(1-\epsilon) \; \frac{1+P_{pol}}{2} \nonumber \\
&= \frac{1-\epsilon P_{pol}}{2}\quad \text{ and } \\
n_-(P_{pol},\epsilon) &= \epsilon \;
\frac{1+P_{pol}}{2}+(1-\epsilon) \; \frac{1-P_{pol}}{2} \nonumber \\
&= \frac{1+\epsilon P_{pol}}{2}
\end{split}
\end{align}
The efficiency of the analyzer to filter spin up
($+$) or spin down ($-$) neutrons is given in case of an opaque spin filter by its transmission
$T_\pm(t) \in [0;1]$, which can be a function of time, so that the measured scattering cross section
$I_\text{m}(\BM{Q})$ is given by
\newlength\breiteb
\settowidth\breiteb{$\displaystyle {}+{}$}
\begin{align}
\begin{split}
I_\text{m}(\BM{Q}) = & \hspace{\breiteb} n_+(P_{pol},\epsilon) T_+(t) I_{++}(\BM{Q}) + n_+(P_{pol},\epsilon) T_-(t) I_{+-}(\BM{Q}) \\[3mm]
           &+                  n_-(P_{pol},\epsilon) T_-(t) I_{--}(\BM{Q}) + n_-(P_{pol},\epsilon) T_+(t) I_{-+}(\BM{Q})
\label{eq:Im(Q)_POLARIS}
\end{split}
\end{align}

\noindent
The spin dependent scattering intensities of magnetic particles with an orientation distribution
$f\left(\theta,\phi\right)$ of its magnetic moment can easily be calculated in terms of order
parameters $S_l$ if one assumes that the particle
are spherical symmetric or the orientation of the magnetic moment of a particle is not correlated to the
particle orientation. Furthermore it will be assumed that an external magnetic field is applied perpendicular
to the incident neutron beam and that all orientations $\phi$ for a given angle $\theta$, which is defined
as the angle between the magnetisation vector of the particle and the direction of the external field $\BM{B}$
have the same probability, i.e. the orientation distribution only depends on $\theta$ so that
$f\left(\theta,\phi\right)=f\left(\theta\right)$. The orientation probability distribution function can be expanded
in terms of a complete set of orthogonal functions. Expanding it in terms of Legendre polynomials $P_l(\cos\theta)$ gives
\begin{align}
f(\theta) &= \sum_l c_l P_l(\cos\theta) = \sum_l \frac{2l+1}{2} S_l P_l(\cos\theta)
\end{align}
The expansion coefficients can be calculated by
\begin{align}
\begin{split}
c_l &= \frac{2l+1}{2} \int_0^\pi f(\theta)\, P_l(\cos\theta) \sin\theta \;\mathrm{d}\theta\\
\mbox{or} \\
S_l &= \int_0^\pi f(\theta) \, P_l(\cos\theta) \sin\theta\;\mathrm{d}\theta
\end{split}
\end{align}
The first three Legendre polynomials are defined by
\begin{subequations}
\begin{align}
P_0(\cos\theta) &= 1\\
P_1(\cos\theta) &= \cos\theta\ \\
P_2(\cos\theta) &= \frac{1}{2}\left(3\cos^2\theta-1\right)
\end{align}
\end{subequations}
For superparamagnetic particle the orientation probability distribution is given by
\begin{align}
f(\theta) = \frac{\alpha}{4\pi\sinh\alpha} \exp(\alpha\cos\theta)
\end{align}
with $\alpha=BM_pV_p/(k_BT)$ being the Langevin parameter. For this orientation probability distribution
the first order parameters can be calculates as
\begin{subequations}
\label{eq:S_l_Boltzmann}
\begin{align}
S_0 & = 1 \\
S_1 & = L(\alpha) = \coth\alpha - \frac{1}{\alpha} \\
S_2 & = 1-3\frac{L(\alpha)}{\alpha}
\end{align}
\end{subequations}
The scattering from a system of many particles is obtained
by summing up the scattering amplitudes of all precipitates
weighted by the phase shift at each particle position.
In the decoupling approach scattering intensity is given by
\begin{align} \label{eq:I_ijSQ}
\frac{\mathrm{d}\sigma_{{\pm\pm}\atop{\pm\mp}}}{\mathrm{d}\Omega}(\BM{Q}) &=
N\left\{
\left\langle \abs{F_{{\pm\pm}\atop{\pm\mp}}(\BM{Q})}^2\right\rangle
+ \abs{\left\langle F_{{\pm\pm}\atop{\pm\mp}}(\BM{Q})\right\rangle}^2 (S(\BM{Q})-1)
\right\}
\end{align}
which consists of two terms. The first one depends only
on the particle structure and corresponds to the independent
scattering of $N$ particles, while the second one is also a
function of their statistical distribution and reflects the interparticle
interference, which is described by $S(\BM{Q})$. The $\langle\rangle$ indicates
an average over all possible configurations, sizes and orientations of the
magnetic moments of the particles. The spin dependent scattering amplitudes
$F_{{\pm\pm}\atop{\pm\mp}}(\BM{Q})$ can be calculated from the nuclear and magnetic
amplitudes
\begin{align}
F_{\pm\pm}(\BM{Q}) & = F_N(\BM{Q}) \mp F_{M_\perp x}(\BM{Q}) \\
F_{\pm\mp}(\BM{Q}) & = -\left(F_{M_\perp y}(\BM{Q}) \pm \imath F_{M_\perp z}(\BM{Q})\right)
\end{align}
The nuclear scattering amplitude is proportional to the nuclear excess scattering $\beta_N=F_N(Q=0)$
and the nuclear form factor $f_N(\BM{Q})$
\begin{align}
F_N(\BM{Q}) &= \beta_N f_N(\BM{Q})
\end{align}
The magnetic scattering amplitude $\BM{F}_{M_\perp}(\BM{Q})$ can be described as a vector,
with
\begin{align}
\BM{F}_{M_\perp}(\BM{Q}) =  \BM{\hat{\mu}}_\perp D_M \mu f_M(\BM{Q})=\BM{\hat{\mu}}_\perp F_M(\BM{Q})
\end{align}
where $f_M(Q)$ is the magnetic form factor, $\BM{\mu}=\BM{M}_pV_p$ the magnetic moment of the particle,
$D_M=\frac{\gamma e}{2\pi\hbar}$, and $\BM{\hat{\mu}}_\perp$ the Halpern-Johnson vector defined as
\begin{align}
\BM{\hat{\mu}}_\perp &= \frac{\BM{Q}}{Q}\times\left(\frac{\BM{\mu}}{\mu}\times\frac{\BM{Q}}{Q}\right)
\end{align}
It is assumed here that the neutron spin polarization is parallel or antiparallel to the axes $\BM{e}_x$
which is the direction perpendicular to the incoming neutron beam.
If only the Halpern-Johnson vector $\BM{\hat{\mu}}_\perp$ depends on the orientation distribution $f(\theta)$ of the
magnetic moments $\BM{\mu}$ of the particles but not the form factor $f_M(\BM{Q})$, which is valid
for spherical symmetric particles or anisotropic shaped particles where the particle shape is not
correlated to the direction of the magnetic moment, the averages in \ref{eq:I_ijSQ}
can be written in terms of the order parameters $S_1$ and $S_2$
\begin{subequations}
\label{eq:F_ijSQ}
\begin{align}
\begin{split}
\left\langle F_{\pm\pm}(\BM{Q})\right\rangle &=  F_N(\BM{Q}) +F_M(\BM{Q})S_1\sin^2\psi
\end{split} \\[3mm]
\left\langle F_{\pm\mp}(\BM{Q})\right\rangle &= F_M(\BM{Q})S_1\sin\psi\cos\psi  \\[3mm]
\begin{split}
\left\langle \abs{F_{\pm\pm}(\BM{Q})}^2\right\rangle &= \abs{F_N(\BM{Q})}^2+\abs{F_M(\BM{Q})}^2 \left[S_2 \sin^4\psi+ \frac{1-S_2}{3} \sin^2\psi\right] \\
                                       & \mp \left[F_M(\BM{Q})F^*_N(\BM{Q})+F^*_M(\BM{Q})F_N(\BM{Q})\right]S_1\sin^2\psi
\end{split} \\[3mm]
\begin{split}
\left\langle \abs{F_{\pm\mp}(\BM{Q})}^2\right\rangle &=\abs{F_M(\BM{Q})}^2 \left[2\;\frac{1-S_2}{3}-S_2\sin^4\psi+\frac{4S_2-1}{3}\sin^2\psi\right]
\end{split}
\end{align}
\end{subequations}
The spin-flip and spin-nonflip cross-section $\frac{\mathrm{d}\sigma_{{\pm\pm}\atop{\pm\mp}}}{\mathrm{d}\Omega}(\BM{Q})$
can be obtained by combining \ref{eq:I_ijSQ} and \ref{eq:F_ijSQ}. The cross-sections
without polarization analysis $I_\pm(\BM{Q})$ and for unpolarized neutrons $I_{unp}(\BM{Q})$ are given by
\begin{subequations} \label{eq:I_iSQ_I_unpSQ}
\begin{align}
\begin{split}
    I_{\pm}(\BM{Q}) &= I_{\pm\pm}(\BM{Q})+I_{\pm\mp}(\BM{Q}) \\[2mm]
                    &= \Bigl[\abs{F_N(\BM{Q})}^2 + \abs{F_M(\BM{Q})}^2S_1^2\sin^2\psi \\
                    & \quad  \mp \left[F_M(\BM{Q})F^*_N(\BM{Q})+F^*_M(\BM{Q})F_N(\BM{Q})\right]S_1\sin^2\psi\Bigr] S(\BM{Q}) \\
                    & \quad \abs{F_M(\BM{Q})}^2\left(\frac{2}{3}\left(1-S_2\right)+\left(S_2-S_1^2\right)\sin^2\psi \right)
\end{split} \\[3mm]
\begin{split}
    I_{unp}(\BM{Q}) &= \frac{1}{2}\left(I_{+}(\BM{Q})+I_{-}(\BM{Q})\right)\\[2mm]
                    &=  \left(\abs{F_N(\BM{Q})}^2 + \abs{F_M(\BM{Q})}^2S_1^2\sin^2\psi\right) S(\BM{Q}) \\
                    & \quad + \abs{F_M(\BM{Q})}^2\left(\frac{2}{3}\left(1-S_2\right)+\left(S_2-S_1^2\right)\sin^2\psi\right)
\end{split}
\end{align}
\end{subequations}

In the case of a Boltzmann orientation distribution
$f(\theta)=\exp\left(\frac{\BM{B\mu}}{k_BT}\right)=\exp\left(\frac{B\mu\cos\theta}{k_BT}\right)$
the order parameter $S_l$ already have been given in eq.\ \ref{eq:S_l_Boltzmann}
and the spin dependent intensities can be written as
\begin{subequations}
\begin{align}
\frac{I_{\pm\pm}(\BM{Q})}{N} &=
\abs{F_M(\BM{Q}) L(\alpha)\,\sin^2\psi
     \pm F_N(\BM{Q})}^2 S(\BM{Q}) \\
&+
\abs{F_M(\BM{Q})}^2\, \left( \frac{L(\alpha)}{\alpha}\,
\sin^2\psi - \left( L^2(\alpha)-1+3\,\frac{L(\alpha)}{\alpha}\right)
\, \sin^4\psi \right) \nonumber \\
\nonumber \\
\frac{I_{\mp\pm}(\BM{Q})}{N} &=
\left( \sin^2\psi - \sin^4\psi \right)
L^2(\alpha)
\abs{F_M(\BM{Q})}^2 S(\BM{Q}) \\
&+
\abs{F_M(\BM{Q})}^2  \left( \left( \sin^4\psi-\sin^2\psi \right)
\left( L^2(\alpha)-1+3\, \frac{L(\alpha)}{\alpha}\right) +
(2-\sin^2\psi)\, \frac{L(\alpha)}{\alpha}\right) \nonumber
\end{align}
\end{subequations}
$\psi$ is the angle between $\BM{Q}$ and the horizontal axis in the plane of the detector.
$L(\alpha)=\coth(\alpha)-\frac{1}{\alpha}$ is the Langevin function. In the case of a static field
the superparamagnetic particle are thermodynamic equilibrium and $\alpha$ is given by
\begin{align}
\alpha&=\frac{BM_PV_P}{k_B T},
\end{align}
with $M_P$ being the particle magnetization, $V_P$ the particle volume,
$T$ the temperature in Kelvin and $k_B$ the Boltzmann constant.

In our experiments we applied an oscillating magnetic field to the sample described by:
\begin{subequations}
\begin{align}
B(t,\nu;d_\text{SD},\lambda,\rho_0) &= B_1 - B_0
\cos(\phi(t,\nu);\dots) \\
\phi(t,\nu;d_\text{SD},\lambda,\rho_0) &=2\pi\nu(t-d_\text{SD}\lambda/3956)+\rho_0
\end{align}
\end{subequations}
where $t$ in [s] is the time between neutron detection and the trigger signal
from the frequency generator, $d_\text{SD}$ in [m] is the sample
detector distance, $\lambda $ in [\AA] the neutron wavelength, and
$\nu$ in [Hz] the frequency of the oscillating magnetic field. As
$t$ is defined here as the time of the neutron detection one has
therefore to correct the phase in the argument for the magnetic
field with an additional phase term. The term
$t_\text{SD}=d_\text{SD}\lambda/3956$ takes into account the flight
time $t_\text{SD}$ of the neutrons between the sample and the
detector. The term  $\rho_0$ accounts for any other
additional constant phase shift between trigger signal and the
magnetic field due to phase shifts in the amplifier.
If the neutron polarization can follow adiabatically the varying magnetic field needs to be verified
experimentally. Therefore we introduce here a parameter $a_\text{ad} \in [0;1]$ which takes into account
wether ($a_\text{ad}=1$) or not ($a_\text{ad}=0$) the neutron spin adiabatically follows the change
of of magnetic field direction ($\text{sgn}(B(t))$).
\begin{align}
\BM{r}_\text{ad} =
\left(
  \begin{array}{c}
     a_\text{ad} \\
     (1-a_\text{ad}) \\
  \end{array}
\right)
\text{ and }
\BM{s}_\text{ad} =
\left(
  \begin{array}{c}
     \text{sgn}(B(t)) \\
     1 \\
  \end{array}
\right)
\end{align}
The measured intensity than reads as
\begin{subequations}
\begin{align}
I_\text{m}(\BM{Q},t) =&
\sum_{i=1,2} r_{ad,i} \Bigg[ \\
&\hspace{\breiteb} n_+\left(s_{ad,i}P_{pol},\epsilon\right)
                        A_+ I_{++}(\BM{Q},t)
+                 n_+\left(s_{ad,i}P_{pol},\epsilon\right)
                        A_- I_{+-}(\BM{Q},t) \nonumber \\
&+                 n_-\left(s_{ad,i}P_{pol},\epsilon\right)
                        A_- I_{--}(\BM{Q},t)
+                 n_-\left(s_{ad,i}P_{pol},\epsilon\right)
                        A_+ I_{-+}(\BM{Q},t)
\Bigg]  \nonumber \\[3mm]
= &\sum_{i=1,2} \; \sum_{k,l=+,-}
r_{ad,i}\; n_k\left(s_{ad,i}P_{pol},\epsilon\right)\; A_l \; I_{kl}(\BM{Q},t)
\end{align}
with
\begin{align}
\begin{split}
A_\pm &= \left(\frac{1+s_{ad,2}}{2}r_{ad,1} + \frac{1+s_{ad,1}}{2}r_{ad,2}\right) T_\pm(t) \\
    &+ \left(\frac{1-s_{ad,2}}{2}r_{ad,1} + \frac{1-s_{ad,1}}{2}r_{ad,2}\right) T_\mp(t)
\end{split}
\end{align}
\end{subequations}
To calculate the time dependent scattering cross section a model for the time evolution of the
orientation distribution of the magnetic moments $f(\theta,t)$ is needed, from which the time
dependent order parameter $S_1(t)$ and $S_2(t)$ can be obtained, as well as a model for the
time evolution of the structure factor $S(\BM{Q},t)$.
\begin{subequations}
\begin{align}
\frac{\mathrm{d}M}{\mathrm{d}t} &= -\frac{M(t)-M_\infty L(\alpha'(t))}{\tau} \\
M(t=0) &= M_0 \\
\alpha'(t) &= \alpha_0\cos(\omega t+\phi_0)+\alpha_1 \\
\alpha_0 &= \frac{-B_0\mu}{k_BT} \\
\alpha_1 &= \frac{B_1\mu}{k_BT} \\
\omega &= 2\pi\nu \\
\phi_0 &= \varphi_0 - \frac{\omega d_\text{SD}\lambda}{3956}
\end{align}
\end{subequations}
%
%\begin{align}
%M(t) &= \textstyle e^{-\frac{t}{\tau}} \left(\int_0^t
%M_\infty e^{\frac{t'}{\tau}} \frac{\coth\left(\alpha_0\cos(\omega t'+\varphi_0)+\alpha_1\right)\alpha_0\cos(\omega t'+\varphi_0)
%+\coth(\alpha_0 \cos(\omega t'+\varphi_0)+\alpha_1) \alpha_1-1}{\tau (\alpha_0 \cos(\omega t'+\varphi_0)+\alpha_1)}
%\mathrm{d}t'+M_0\right)
%\end{align}
In the limit of small values for the Langevin parameter $\alpha$ the Langevin function can be approximated
by
\begin{align}
    L(\alpha) \rightarrow \frac{\alpha}{3}
\end{align}
for which the differential equation has an analytical solution
\begin{align}
M(t) &= e^{-\frac {t}{\tau}} \left[ M_0- M_{\infty}\left(\frac{\alpha_0}{3}\;\frac{
\cos \left( \phi_0 \right) +\omega\tau\, \sin\left( \phi_0 \right) }{1+{\omega}^{2}{\tau^{2}}}
+\frac{\alpha_1}{3}\right)\right] \nonumber \\
&+M_\infty \left(\frac{\alpha_0}{3} \; \frac{\cos \left( \omega\,t+\phi_0 \right) +
\omega\tau\,\sin \left( \omega\,t+\phi_0 \right) }{1+\omega^2\tau^2}+\frac{\alpha_1}{3}\right) \\
&\underset{t\gg\tau}{\simeq}
M_\infty \left(\frac{\alpha_0}{3} \; \frac{\cos \left( \omega\,t+\phi_0 -\frac{\pi}{2}+\arcsin\left(\frac{1}{\sqrt{1+\omega^2\tau^2}}\right)\right) }{\sqrt{1+\omega^2\tau^2}}+\frac{\alpha_1}{3}\right)
\end{align}

Assuming that the system is at any time in thermodynamic equilibrium with the actual magnetic field $B(t)$
than the time dependent oscillating SANS signal can be described by simply  introducing a time dependent
value for $\alpha(t)$:
\begin{align}
\alpha(t,\lambda,d_\text{SD},\rho_0,\mu_\text{kT}) =
B(t,\lambda,d_\text{SD},\rho_0) \frac{M_PV_P}{k_B T} = B(\dots)\;
\mu_\text{kT}
\end{align}
with
\begin{align}
\mu_\text{kT} = \frac{M_PV_P}{k_B T}
\end{align}
In case that the magnetic moments can not follow the external magnetic field this could be described
by an additional phase $\Delta \phi_\alpha$ between $\alpha$ and $B$ and a damping factor $d_\alpha$ so that
\begin{align}
\alpha(t,\lambda,d_\text{SD},\rho_0-\Delta \phi_\alpha,\mu_\text{kT}) =
d_{\alpha} B(t,\lambda,d_\text{SD},\rho_0-\Delta \phi_\alpha) \frac{M_PV_P}{k_B T}.
\end{align}
In such a case of such
\begin{align}
\BM{r}_\text{ad} =
\left(
  \begin{array}{c}
    (1-a_\text{ad}) \; \text{sgn}(B(t)) \\
    a_\text{ad} \; \text{sign}(B(t)\alpha(t))\\
  \end{array}
\right)
\end{align}
Furthermore we assume here that the size of the form factors
of the magnetic and nuclear scattering are the same except the
scattering contrast that means the ratio of magnetic to nuclear
form factor is $Q$-independent and equal to the squared ratio of
magnetic to nuclear scattering length density
\BE
\frac{F_N^2(Q)}{\tilde{F}_M^2(Q)}  = const = \left( \frac{\Delta
b_\text{nuc}}{\Delta b_\text{mag}}\right)^2
\EE
Therefore the time
dependent signal on the detector for a given direction $\psi$ is given by
\begin{align}
 y(t,\dots) =
\int_{\lambda_0-\Delta\lambda}^{\lambda_0+\Delta\lambda}
    p_\triangle(\lambda) \;\; I_m(t,\lambda,\psi,\dots) \;\;  d\lambda
\end{align}
whereby $p_\triangle(\lambda)$ describes the triangular shaped
wave length distribution of the neutron beam.


\clearpage
\subsection{SuperParStroboPsiSQ}  ~\\

The external applied field is given by: \BE
B(t,d_\text{SD},\lambda,\rho_0) = B_1 - B_0
\cos(2\pi\nu(t-d_\text{SD}\lambda/3956)+\rho_0) \EE where $t$ in
[s] is the time between neutron detection and the trigger signal
from the frequency generator, $d_\text{SD}$ in [m] is the sample
detector distance, $\lambda $ in [\AA] the neutron wavelength, and
$\nu$ in [Hz] the frequency of the oscillating magnetic field. As
$t$ is defined here as the time of the neutron detection one has
therefore to correct the phase in the argument for the magnetic
field with an addition phase term. The term
$t_\text{SD}=d_\text{SD}\lambda/3956$ takes account for the flight
time $t_\text{SD}$ of the neutrons between the sample and the
detector. The term  $\rho_0$ takes account for any other
additional constant phase shift between trigger signal and the
magnetic field due to phase shifts in the amplifier.

The scattering intensity of superparamagnetic particles including a structure factor
$S(Q)$is given by
\begin{align}
I(Q) &= \underbrace{\tilde{F}_M^2(Q) \; 2
\frac{L(\alpha)}{\alpha} + F_N^2(Q)S(Q)}_{\displaystyle A(Q)} \\
     &+ \underbrace{\tilde{F}_M^2(Q) \left[\left(1-3 \frac{L(\alpha)}{\alpha}-L^2(\alpha)\right)+L^2(\alpha)S(Q)\right]}_{\displaystyle B(Q)} \sin^2\Psi
     \nonumber
\end{align}
Assuming that the system is in equilibrium faster than $1/\nu$
than the above equation can be used to describe the time dependent
oscillating SANS signal simply by introducing a time dependent
value for $\alpha$:
\BE
\alpha(t,\lambda,d_\text{SD},\rho_0,\mu_\text{kT}) =
B(t,\lambda,d_\text{SD},\rho_0) \frac{M_PV_P}{k_B T} = B(\dots)\;
\mu_\text{kT}
\EE
with
\BE
\mu_\text{kT} = \frac{M_PV_P}{k_B T}
\EE
Furthermore we assume here that the size of the form factors
of the magnetic and nuclear scattering are the same except the
scattering contrast that means the ratio of magnetic to nuclear
form factor is $Q$-independent and equal to the squared ratio of
magnetic to nuclear scattering length density
\BE
\frac{F_N^2(Q)}{\tilde{F}_M^2(Q)}  = const = \left( \frac{\Delta
b_\text{nuc}}{\Delta b_\text{mag}}\right)^2
\EE
Therefore the time
dependent signal on the detector for a given direction $\Psi$ and
integrated over $Q$ in this direction is given by
\begin{align}
i(t,\dots)
    &= \int_{Q_\text{min}}^{Q_\text{min}} I(Q,t,\lambda,\Psi,d_\text{SD},\rho_0) dQ  \\
    &= c\left[2\frac{L(\alpha)}{\alpha}
              + \frac{F_N^2(Q)}{\tilde{F}_M^2(Q)} S(Q)
              + \left[\left(1-3 \frac{L(\alpha)}{\alpha}-L^2(\alpha)\right)+L^2(\alpha)S(Q)\right]\sin^2\Psi\right] \nonumber \\
    &= c\left[2\frac{L(\alpha)}{\alpha}
              + \left( \frac{\Delta b_\text{nuc}}{\Delta b_\text{mag}}\right)^2 S(Q)
              + \left[\left(1-3 \frac{L(\alpha)}{\alpha}-L^2(\alpha)\right)+L^2(\alpha)S(Q)\right]\sin^2\Psi\right] \nonumber
\end{align}
\BE y(t,\dots) =
\int_{\lambda_0-\Delta\lambda}^{\lambda_0+\Delta\lambda}
    p_\triangle(\lambda) \;\; i(t,\lambda,\Psi,d_\text{SD},\rho_0) \;\;  d\lambda
\EE whereby $p_\triangle(\lambda)$ describes the triangular shaped
wave length distribution of the neutron beam.

\subsection{SuperParStroboPsiSQBt}  ~\\
The same as {\tt SuperParStroboPsiSQ} except that the structure factor $S(Q,t)$ becomes field
dependent:
\begin{align}
S(Q,t) = 1+\left[S(Q)-1\right] \abs{\frac{B(t,d_\text{SD},\lambda,\rho_0)}{\abs{B_1}+\frac{2}{\pi}\abs{B_0}}}
\end{align}


\subsection{SuperParStroboPsiSQLx}  ~\\
The same as {\tt SuperParStroboPsiSQ} except that the structure factor $S(Q,t)$ becomes field
dependent:
\begin{align}
S(Q,t) = 1+\left[S(Q)-1\right] \abs{L(\alpha)}
\end{align}



\subsection{SuperParStroboPsiSQL2x}  ~\\
The same as {\tt SuperParStroboPsiSQ} except that the structure factor $S(Q,t)$ becomes field
dependent:
\begin{align}
S(Q,t) = 1+\left[S(Q)-1\right] L^2(\alpha)
\end{align}
