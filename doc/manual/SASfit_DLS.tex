
\chapter{Basic Analysis of Dynamic Light Scattering Data}
\label{basicDLS}
In a typical dynamic light scattering (DLS) or photo correlation scattering (PCS)
experiment, the autocorrelation function $G^{(2)}(\tau)$ of the intensity scattered
by dispersed particles is determined as a function of the delay $\tau$. $G^{(2)}(\tau)$
is related to the modulus of the normalized field autocorrelation function $g_1(\tau)$
by a Siegert relationship
\begin{align}
G^{(2)}(\tau) & = A g^2_1(\tau)  + B.
\end{align}
Here $B$ is a background term often designated as the baseline and $A$ can be considered
as another instrumental factor. The time dependence of $g_1(\tau)$ is related to the
dynamics of the dispersed particles. For particles in Brownian motion, the time decay of
$g_1(\tau)$ is determined by the diffusion coefficient of the dispersed particles. In particular,
for monodisperse samples $g_1(\tau)$ is an exponentially decaying function:
\begin{align}
g_1(\tau) & = \exp(-\Gamma\tau) \label{g1}\\
\text{ or } & \nonumber \\
G^{(2)}(\tau)  & = A \exp(-2\Gamma\tau) + B
\label{G2}
\end{align}
where the decay rate $\Gamma$ is linked to the particles' diffusion coefficient $D$
by $\Gamma=D Q^2$, where $Q$ is the modulus of the scattering vector
\begin{align}
Q = \frac{4\pi m_1}{\lambda_0} \sin(\theta/2)
\end{align}
$m_1$ is the refraction index of the solution, $\lambda_0$ the wavelength in vacuo
of the incident light and $\theta$ the scattering angle. At the end the Stokes-Einstein
expression for the diffusion coefficient is used to get an average particle radius
$R_\text{DLS}$
\begin{align}
D = \frac{kT}{6\pi\eta R_\text{DLS}}
\label{DiffRDLS}
\end{align}
where $k$ is Boltzmann's constant, $T$ the absolute temperature, $\eta$ the viscosity
of the dispersion medium and $R_\text{DLS}$ the particle radius (only valid for noninteracting
particles).

\newpage
\section{Cumulant Analysis}
The formulas in eq.\ \ref{g1} to \ref{DiffRDLS} are valid for monodisperse
dispersions only. For polydisperse dispersions the cumulants method of Koppel (1972) is widely used,
which assumes a multi-exponential behaviour so that $g_1(\tau)$ and $G^{(2)}(\tau)$
can be written in a series expansion as:
\begin{align}
 g_1(\tau)  &= \exp\left(-\Gamma_1 \tau + \frac{\Gamma_2 \tau^2}{2} - \frac{\Gamma_3 \tau^3}{6} + \cdots \right)\\
 G^{(2)}(\tau) &=  A \exp\left(- 2\Gamma_1 \tau + \Gamma_2 \tau^2 - \frac{\Gamma_3 \tau^3}{3} + \cdots \right) + B
\label{G2Cumulant}
\end{align}
\texttt{SASfit} assumes for the cumulant fit-routine that the function $G^{(2)}(\tau)$ is supplied. As normally no error
bar is available from the correlator a robust least square procedure is implemented.

\vspace{5mm}
\noindent REFERENCE:\\
Dennis E. Koppel, Analysis of Macromolecular Polydispersity in
Intensity Correlation Spectroscopy: The Method of Cumulants, The
Journal of Chemical Physics, Vol.1, No. 11 (1972), 4815- 4820

\newpage
\section{Double Decay Cumulant Analysis}
\begin{align}
G^{(2)}(\tau) &=  A \left[p \, e^{- 2\Gamma_{a,1}\, \tau + \Gamma_{a,2}\, \tau^2}
                     +(1-p) \, e^{- 2\Gamma_{b,1}\, \tau + \Gamma_{b,2}\, \tau^2}\right] + B
\label{G2DoubleCumulantDecay}
\end{align}

\newpage
\section{Fit of Double Stretched Exponentials}
\begin{align}
G^{(2)}(t) = A \left\{p \, \exp\left(\left[\frac{t}{\tau_1}\right]^{\gamma_1}\right)
+ (1-p) \, \exp\left(\left[\frac{t}{\tau_2}\right]^{\gamma_2}\right)
\right\} + B \label{G2DoubleExpDecay}
\end{align}


\newpage
\subsection{The least squares minimiser and the robust least squares
procedure}
The function to be minimised is
$$
\chi^2 = \sum_i \left( \frac{r_i}{\Delta I_i} \right)^2
$$
where the residual is defined as
$$
r_i = I_i - I_{i,th}.
$$
Here $I_i$  is the intensity correlation function $G^{(2)}(t_i)$
at time $t_i$ with already subtracted baseline as received from a correlator, $I_{i,th}$ is the value
of the cumulant fit according to eq.\ \ref{G2Cumulant} or a double exponential decay according to
eq.\ \ref{G2DoubleExpDecay}. Normally no error values
$\Delta I_i$ are supplied from the correlator so that all data points are weighted
the same. A robust fitting with bisquare weights is implemented which uses an
iteratively reweighted least squares algorithm, and follows the procedure:
\begin{enumerate}
\item Fit the model by an unweighted least
squares (that is,  $\chi$).
\item Standardize the residuals via $u_i=r_i/(Ks)$. Here $K$
is a tuning constant equal to 4.685, and $s$ is the robust variance
given by $MAD/0.6745$, where $MAD$ is the median absolute deviation of
the residuals
$$
MAD = \sum_{i=1}^N \frac{1}{N}\abs{I_{i,th}-I_i}
$$
\item Compute the robust weights $w_i$ as a function of the standardized residuals $u_i$.
The bisquare weights are given by
$$
w_i =
\begin{cases}
\left(1-u_i^2 \right)^2 & \abs{u_i} < 1 \\
0                       & \abs{u_i} \geq 1
\end{cases}
$$
\item  Re-do the fit using the weighted minimiser:
$$
\chi^2 = \sum_i w_i \left( \frac{r_i}{\Delta I_i} \right)^2
$$
\item The fit converges when the MAD changes by no more than the fraction set by
{\verb"residual_tolerance"} (which has been chosen to be $10^{-8}$). Otherwise, perform the next iteration of the
fitting procedure by returning to the first step.
\end{enumerate}


