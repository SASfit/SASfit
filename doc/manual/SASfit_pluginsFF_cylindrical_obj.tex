%%%%%%%%%%%%%%%%%%%%%%%%%%%%%%%%%%%%%%%%%%%%%%%%%%%%%%%%%%%%%%%%%%%%%%%%
\clearpage
\section{Cylindrical objects}


%%%%%%%%%%%%%%%%%%%%%%%%%%%%%%%%%%%%%%%%%%%%%%%%%%%%%%%%%%%%%%%%%%%%%%%%%

%\clearpage
\subsection{Helical structures} ~\\
Several approaches for describing the small angle scattering signal from randomly oriented helical structures has been published.
In \cite{Franklin1956,Puigjaner1974} a model of a double helix with a round cross section for each strand has been developed. For a fanlike cross section of a double helix a solution has been given by \cite{Schmidt1970,Pringle1971}. Fukuda et al.\ \cite{Fukuda2002} have extended the model to an arbitrary shaped cross section. In a model developed by Lebedev et al.\ \cite{Lebedev2003} it is assumed, that beads are arranged on a helical path assuming a single strand. In all models the helix is straight and its length dimension is large compared to all other characteristic length of the helix.
In \cite{Benham1980} a solution of a single helical strand which can have a secondary coiling has been described. In this paper the cross section of the helical strand is assumed to be infinitesimal thin.

~\\
\subsubsection{Fanlike helix} ~\\
\begin{figure}[htb]
\begin{center}
  \subfigure[top on view]{\label{fig:beadshelixside}\includegraphics[width=0.45\textwidth]{../images/form_factor/cylindrical_obj/fanlike_helicesXS.png}}
\hfill
  \subfigure[side view]{\label{fig:beadshelixside}\includegraphics[width=0.45\textwidth]{../images/form_factor/cylindrical_obj/fanlike_helices3D.png}}
\end{center}
\caption{Double helix with strands of round cross sections.} \label{fig:roundhelix}
\end{figure}
\cite{Schmidt1970,Pringle1971,Fukuda2002,Teixeira2010}
\begin{align}
\begin{split}
I(Q) &= P_\text{rod}(Q,H) \left(\left(\eta_\text{h}-\eta_\text{solv}\right)\omega R^2\left(1-a^2\right)\right)^2 \\
&\times \sum_{n=0}^\infty \epsilon_n \left( \cos(n\varphi/2) \frac{\sin(n\omega/2)}{n\omega/2} g_n\left(Q,R,a\right)\right)^2
\end{split} \\
g_n\left(Q,R,a\right) &= \frac{2}{R^2\left(1-a^2\right)} \int_{aR}^R r' J_n\left(Qr'\left(1-q_n^2\right)\right)\mathrm{d}r' \\
q_n &=
\begin{cases}
\begin{array}{rcl}
\frac{2\pi n}{QP} & \text{for} & Q\geq \frac{2\pi n}{P}\\
1 & \text{for} & Q < \frac{2\pi n}{P}
\end{array}
\end{cases} \\
P_\text{rod}(Q,H) &= H^2 \left(2\frac{\mathrm{Si}(QH)}{(QH)}-\left(\frac{\sin(QH/2)}{QH/2}\right)^2\right)
\end{align}
with $\epsilon_0=1$ for $n=0$ and $\epsilon_n=2$ for $n\geq 1$. The sum converges very fast and for small $Q$-values the first few terms are already sufficient. However, {\tt SASfit} is continuing the sum until the relative change of the sum is less than $10^{-10}$.
The forward scattering of the model is normalized to the squared scattering length density contrast and squared total volume of the helix so that
\begin{align}
I(Q=0) &= \left(H\left(\eta_\text{h}-\eta_\text{solv}\right)\omega R^2\left(1-a^2\right)\right)^2
\end{align}

\vspace{5mm}

\underline{Input Parameters for model \texttt{fanlike helix}:}\\
\begin{description}
\item[\texttt{R}] external helix radius $R$
\item[\texttt{a}] inner helix radius $aR$, with $0\leq a\leq 1$
\item[\texttt{omega}] angular of the sector of material $\omega$
\item[\texttt{phi}] angle between the two sectors of matrerial $\varphi$
\item[\texttt{dummy}] not used
\item[\texttt{P}] height of one helix period $P$
\item[\texttt{H}] total length of helix $H$
\item[\texttt{eta\_h}] scattering length density of helix $\eta_\text{h}$
\item[\texttt{dummy}] not used
\item[\texttt{eta\_solv}] scattering length density of solvent $\eta_\text{solv}$
\end{description}

\noindent\underline{Note:}
\begin{itemize}
\item The helix is assumed to be stiff and long so that it scattering intensity can be factorized in a cross-section contribution and a shape contribution, whereas the shape contribution can be described by an infinitesimal thin rod of length $H$.
\item $R$, $P$, and $H$ are only physical for values larger than 0.
\item The model is an approximation for the limit $H \gg P$ and $H \gg R$.
\item $0\leq a\leq 1$
\end{itemize}

\phantom{.}~\\
\subsubsection{Helix with round cross-section} ~\\
\begin{figure}[htb]
\begin{center}
  \subfigure[top on view]{\label{fig:beadshelixside}\includegraphics[width=0.45\textwidth]{../images/form_factor/cylindrical_obj/roundhelices1.png}}
\hfill
  \subfigure[side view]{\label{fig:beadshelixside}\includegraphics[width=0.45\textwidth]{../images/form_factor/cylindrical_obj/round_helices3D.png}}
\end{center}
\caption{Double helix with strands of round cross sections. The cross-sections are round in the plane perpendicular to the helix axis.} \label{fig:roundhelix}
\end{figure}
\cite{Franklin1956,Puigjaner1974,Fukuda2002}
\begin{align}
I(Q) &= P_\text{rod}(Q,H) \sum_{n=0}^{\infty} \epsilon_n \left( A_{1n}^2 + A_{2n}^2+2A_{1n}A_{2n}\cos n\alpha\right)
\label{eq:roundhelix1}
\end{align}
where
\begin{align}
A_{in} &= J_n\left(\delta_i Q_\perp\right) \pi R_i^2 \left(\eta_i-\eta_\mathrm{solv}\right)\frac{2J_1\left(R_i Q_\perp\right)}{R_i Q_\perp} \\
Q_\perp^2 &= Q^2-\left[\frac{2\pi n}{P}\right]^2
\end{align}
with $\epsilon_0=1$ for $j=0$ and $\epsilon_j=2$ for $j\geq 1$. Furthermore $R_1$ and $R_2$ are the radii of the round helical strands.
$\delta_1$ and $\delta_2$ are the distances of the strands to the helix axis.
The pitch of the helix is denoted by $P$, whereas $\alpha$ is the angle between the two strands.

\phantom{.}~\\
\subsubsection{Beads model of a single helical strand} ~\\
\cite{Lebedev2003,Avdeev2013}

\begin{figure}[htb]
\begin{center}
  \subfigure[side view]{\label{fig:beadshelixside}\includegraphics[width=0.45\textwidth]{../images/form_factor/cylindrical_obj/beads_helix_model_1.png}}
\hfill
  \subfigure[top on view]{\label{fig:beadshelixside}\includegraphics[width=0.45\textwidth]{../images/form_factor/cylindrical_obj/beads_helix_model_2.png}}
%\includegraphics[width=0.4\textwidth,height=0.74\textwidth]{../images/form_factor/cylindrical_obj/beads_helix_model.png}
\end{center}
\caption{Bead model of a helix. In figure a) bead helices with different bead radius and different number of beads per turn are shown} \label{fig:beadshelix}
\end{figure}

\begin{align}
I(Q) &= P_\text{rod}(Q,H) \sum_{j=-\infty}^{\infty} \abs{\frac{nH}{P}\Psi_j\left(QD,\frac{2\pi j}{PQ}\right) \Phi(Q,R)}^2
\label{eq:beadshelix1}
\end{align}
with
\begin{align}
\Psi_j\left(QD,\frac{2\pi j}{PQ}\right) &=
\begin{cases}
\begin{array}{rcl}
J_j\left(\frac{QD}{2}\sqrt{1-\left(\frac{2\pi j}{PQ}\right)}\right) & \text{for} & Q\geq \frac{2\pi\abs{j}}{P}\\
0 & \text{for} & Q < \frac{2\pi\abs{j}}{P}
\end{array}
\end{cases}\\
\Phi(Q,R) &= 3\frac{4\pi}{3}R^3\left(\eta_\text{b}-\eta_\text{solv}\right)\frac{\sin(QR)-QR\cos{QR}}{\left(QR\right)^3} \\
P_\text{rod}(Q,H) &= H^2 \left(2\frac{\mathrm{Si}(QH)}{(QH)}-\left(\frac{\sin(QH/2)}{QH/2}\right)^2\right)
\end{align}
where $\mathrm{Si}(x)=\int_0^x\frac{\sin t}{t}\mathrm{d}t$ is the sine integral. $P_\text{rod}(Q,H)$ is the form factor of an infinitesimal thin rod of length $H$.
$J_j$ is the regular cylindrical Bessel function, for which $J_{j}(x)=(-1)^\abs{j}J_{-j}(x)$ if $j\in \mathbb{Z}$ (integer value) and $j \neq 0$.
Therefore the sum in eq.\ \ref{eq:beadshelix1} can be written as
\begin{align}
I(Q) &= P_\text{rod}(Q,H) \sum_{j=0}^{\infty} \epsilon_j \abs{\frac{nH}{P}\Psi_j\left(QD,\frac{2\pi j}{PQ}\right) \Phi(Q,R)}^2
\label{eq:beadshelix2}
\end{align}
with $\epsilon_0=1$ for $j=0$ and $\epsilon_j=2$ for $j\geq 1$. The sum converges very fast and for small $Q$-values the first two term are already sufficient. However, {\tt SASfit} is continuing the sum until either the argument below the square root becomes negative or the relative change of the sum is less than $10^{-10}$.

The forward scattering of the model is normalized so that
\begin{align}
I(Q=0) &= \left(H\frac{nH}{P}\frac{4\pi}{3}R^3\left(\eta_\text{b}-\eta_\text{solv}\right)\right)^2
\end{align}

\vspace{5mm}

\underline{Input Parameters for model \texttt{beads helix}:}\\
\begin{description}
\item[\texttt{R}] radius of monomer units/beads $R$
\item[\texttt{D}] mean diameter of helix $D$
\item[\texttt{n}] number of monomer beads per turn $n$
\item[\texttt{dummy}] not used
\item[\texttt{dummy}] not used
\item[\texttt{P}] height of one helix period $P$
\item[\texttt{H}] total length of helix $H$
\item[\texttt{eta\_b}] scattering length density of monomer beads $\eta_\text{b}$
\item[\texttt{dummy}] not used
\item[\texttt{eta\_solv}] scattering length density of solvent $\eta_\text{solv}$
\end{description}

\noindent\underline{Note:}
\begin{itemize}
\item The helix is assumed to be stiff and long so that it scattering intensity can be factorized in a cross-section contribution and a shape contribution, whereas the shape contribution can be described by an infinitesimal thin rod of length $H$.
\item $R$, $P$, $D$, $H$, and $n$ are only physical for values larger than 0.
\item The model is an approximation for the limit $H \gg P$ and $H \gg D$.
\end{itemize}

\phantom{.}~\\
\subsubsection{straight superhelix} ~\\

\begin{figure}[htb]
\begin{center}
\includegraphics[width=0.497\textwidth,height=0.529\textwidth]{../images/form_factor/cylindrical_obj/straightSuperhelix.png}
\end{center}
\caption{infinitesimal thin single stranded helix} \label{fig:straightsuperhelix}
\end{figure}

Benham et al.\ \cite{Benham1980} describe single helices with a secondary helical curvature, which they call coiled superhelices. They assume in their analysis an infinitesimal thin helical strand. The straight superhelix is their starting geometry without a secondary coiling. The expression for a random oriented straight superhelix of finite length can be written exact in form of a single integral and is given by
\begin{align}
I(Q) &= 2 \int_0^{\mathcal{L}} \frac{(\mathcal{L}-w)\sin Q\psi}{Q\psi} \mathrm{d}w \\
\psi &= + \sqrt{2 R_1^2\left[1-\cos\left(\frac{w}{\sqrt{R_1^2+a^2)}}\right) \right]+\frac{a^2w^2}{R_1^2+a^2}} \\
\mathcal{L} &= 2\pi \frac{H}{P} \sqrt{R_1^2+a^2} \\
a &= \frac{P}{2\pi}
\end{align}
$\mathcal{L}$ is the arclength of the helix, $R_1$ the distance to the helix axis and $P$ the pitch of the helix. The intensity is normalized to the squared arclength of the helix. i.e. $I(Q=0)=\mathcal{L}^2$.
\vspace{5mm}

\underline{Input Parameters for model \texttt{beads helix}:}\\
\begin{description}
\item[\texttt{R\_1}] distance to the helix axis $R_1$
\item[\texttt{dummy}] not used
\item[\texttt{dummy}] not used
\item[\texttt{dummy}] not used
\item[\texttt{dummy}] not used
\item[\texttt{P}] height of one helix period $P$
\item[\texttt{H}] total length of helix $H$
\item[\texttt{dummy}] not used
\item[\texttt{dummy}] not used
\item[\texttt{dummy}] not used
\end{description}

\noindent\underline{Note:}
\begin{itemize}
\item the model assumes an infinitesimal thin single helical strand.
\item $R$, $P$, $D$, $H$, and $n$ are only physical for values larger than 0.
\item The model is an approximation for the limit $H \gg P$ and $H \gg D$.
\end{itemize} 