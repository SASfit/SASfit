\chapter{Plugin functions}
\label{ch:plugins}

%%%%%%%%%%%%%%%%%%%%%%%%%%%%%%%%%%%%%%%%%%%%%%%%%%%%%%%%%%%%%%%%%%%%%%%%%

%\clearpage
\section{local planar obj.} ~\\
\label{plugin:LocalPlanar)}
In subsection
\ref{subsect:DiblockCopolymerMicelles} the form factor of micelles
is described which consist of diblock copolymers. The formalism
for the form factor can be easily extended to shapes with local
planar or local cylindrical geometry
\cite{PedersenJApplCryst2000,Neugebauer1943} for which the the
form factor can be factorized in cross-section $P_{cs}(Q)$ term
and an overall shape term $P'(Q)$ (see also
~\ref{sect:RodLikeMicelles} and ~\ref{sect:WormLikeMicelles}
\begin{align}
I(Q) &=P'(Q) P_{cs}(Q).
\end{align}
\subsection{Unilamellear vesicles: \texttt{ULV+Chains(RW)}}
\label{plugin:ULV+Chains(RW)}

%%%%%%%%%%%%%%%%%%%%%%%%%%%%%%%%%%%%%%%%%%%%%%%%%%%%%%%%%%%%%%%%%%%%%%%%%

\clearpage
\section{JuelichCoreShell} ~\\

This model considers a dense core and original two shells
\cite{Willner2000}. Besides, it considers two different density
profiles: a parabolic and a star-like profile for the second shell.
\begin{align}
\eta_\text{shell}(r) & \propto r^{-x} \quad \text{for starlike profile $x=4/3$}\\
\eta_\text{shell}(r) & \propto 1-\left(\frac{r}{L_p}\right)^{2}
\quad \text{for parabolic profile of thickness $L_p$}
\end{align}
Model parameters:
\begin{description}
\item[$b_\text{solv}$]  scattering length density of the solvent
\item[$I_0$] forward scattering
\item[$M_\text{core}$] molecular weight of core (g/mol)
\item[$M_\text{brush}$] molecular weight brush (g/mol)
\item[$\rho_\text{core}$] mass density of core matter (g/cm$^3$)
\item[$\rho_\text{brush}$] mass density of brush matter (g/cm$^3$)
\item[$b_\text{core}$] scattering length density of core material (cm$^{-2}$)
\item[$b_\text{brush}$] scattering length density of brush material (cm$^{-2}$)
\item[$N_\text{agg}$] aggregation number (real number)
\item[$d_c^+$] extra radius of core (compared to compact)
\item[$p_{12}$] relative distribution of shell amount in
(1$^\text{st}$shell:2$^\text{nd}$shell) ($0\ldots\infty$)
\item[$d_1^+$] extra radius of first shell (compared to compact)
\item[$d_2^+$] extra radius of second shell (compared to compact)
\item[$\sigma_c$] core smearing
\item[$\sigma_1$] smearing of 1$^\text{st}$ shell
\item[$\sigma_2$] smearing of 2$^\text{nd}$ shell
\item[$x_\text{star}$] relative distribution of
parbolic:starlike profile in 2$^\text{nd}$ shell, one has to put a
very high value in order to consider only a star-like profile.
\item[$\gamma$] for star-like profile the exponent is $4/3$ and for a constant profile chose 0
\item[$L_p$] thickness of parabolic brush (must fit in 2$^\text{nd}$ shell!)
\end{description}
%parnam(16)= 'f_brush' scattering length density correction factor brush
%parnam(17)= 'f_core' scattering length density correction factor core
\begin{align}
I(Q) = %\frac{amplitu}{V_c+V_b}
\left[\Delta b_c F_c + \Delta b_b (F_1+F_2)\right]^2
\end{align}
\begin{align}
\Delta b_c &= b_\text{core} - b_\text{solv}  (1-f_\text{core}) \\
\Delta b_b &= b_\text{brush} - b_\text{solv} (1-f_\text{brush})
\end{align}
$V_c$ and $V_b$ are the core and shell bulk volumes respectively.
\\

\noindent \textbf{Mass Conservation:} \\
From the given values of the molecular weights of the two blocks and
their densities, and an assumed aggregation number $N_\text{agg}$,
the bulk volumes of the core and the shell, $V_c$ and $V_b$, can be
calculated.

\noindent \textbf{Core:}
\begin{align}
& \text{bulk core volume:} \quad          V_c = \frac{N_\text{agg} M_\text{core}}{\rho_\text{core}N_a} \\
& \text{minimal radius of core:} \quad    R_c^0 = \left(\frac{3}{4\pi} V_c\right)^{1/3} \\
& \text{effective core radius:} \quad     R_c = R_c^0 + d_c^+ \\
& \text{swollen core volume:} \quad       V_{sc} = \frac{4}{3}\pi R_c^3 \\
& \text{swelling factor:} \quad           s_c = \frac{V_{sc}}{V_c}
\end{align}

\noindent \textbf{Shell:}
\begin{align}
& \text{bulk shell volume:} \quad         V_b = \frac{N_\text{agg}
M_\text{shell}}{\rho_\text{shell}N_a}
\end{align}
The relative amount of shell material in the first shell
$f_\text{shell1}$ is controlled by the parameter $p_{12}$, so that
the portion of the second shell $f_\text{shell2}$ can be obtained
through:
\begin{align}
f_\text{shell1} &= \frac{p_{12}}{1+p_{12}} \\
f_\text{shell2} &= 1-f_\text{shell1}
\end{align}

\noindent \textbf{Shell 1:}
\begin{align}
& \text{portion of the total shell volume in first shell:} \quad  V_{s1} = f_\text{shell1} V_b \\
& \text{minimal radius of shell:} \quad    R_{c1} = \left(\frac{3}{4\pi} (V_{sc}+V_{s1})\right)^{1/3} \\
& \text{effective core radius:} \quad     R_1 = R_{c1} + d_1^+ \\
& \text{swollen volume of first shell:} \quad       V_{s1s} = \frac{4}{3}\pi R_1^3 \\
& \text{swelling factor:} \quad           s_{s1} =
\frac{V_{s1s}-V_{sc}}{V_{s1}}
\end{align}

\noindent \textbf{Shell 2:}
\begin{align}
& \text{portion of the total shell volume in second shell:} \quad  V_{s2} = f_\text{shell2} V_b \\
& \text{minimal radius of shell:}        \quad R_{c2} = \left(\frac{3}{4\pi} (V_{s1s}+V_{s2})\right)^{1/3} \\
& \text{effective core radius:}          \quad R_2 = R_{c2} + d_2^+ \\
& \text{swollen volume of second shell:} \quad V_{s2s} = \frac{4}{3}\pi R_2^3 \\
& \text{swelling factor:}                \quad s_{s2} = \frac{V_{s2s}-V_{s1s}}{V_{s2}}\\
& \text{fraction of star-like density profile in 2$^\text{nd}$
shell:} \quad  f_\text{star} =2
\frac{\arctan(\abs{p_\text{star}})}{\pi}
\end{align}

\noindent Together with the profile functions $\Phi_c(r,R_c)$,
$\Phi_1(r,R_1,R_2)$, $\Phi_2(r,R_1,R_2,f_\text{star})$ and
\begin{align}
f_\text{Fermi}(x) = \frac{1}{1+\exp(x)}
\end{align}
the volumes of the core and two shells and the corresponding form
factor are determined by numerical integration. \sloppy
\\

\noindent \textbf{Profiles:}
\begin{align}
\Phi_c(r,R_c) &=    f_\text{Fermi}(r-R_c) \, dr
\end{align}
\begin{align}
\Phi_1(r,R_1,R_2) &=   (1-f_\text{Fermi}(r-R_1)) \, \,
f_\text{Fermi}(r-R_2) \, dr
\end{align}
for  $r<R_1$
\begin{align}
\Phi_2(r,R_1,R_2,f_\text{star},\gamma) =   & (1-f_\text{Fermi}(r-R_1)) \, \, f_\text{Fermi}(r-R_2) \nonumber \\
\times & \left[(1-f_\text{star}) +
\frac{f_\text{star}}{R_1^\gamma}\right]
\end{align}
for $r>R_1$
\begin{align}
\Phi_2(r,R_1,R_2,f_\text{star},\gamma,L_p) =  &  (1-f_\text{Fermi}(r-R_1)) \, \, f_\text{Fermi}(r-R_2) \nonumber \\
\times & \left[(1-f_\text{star})
\left(1-\left(\frac{r-R_1}{L_p}\right)^2 \right) +
\frac{f_\text{star}}{r^\gamma} \right]
\end{align}


\vspace{5mm}
\noindent \underline{Input Parameters for model \texttt{JuelichCoreShell}:}
\begin{description}
\item[\texttt{C}] scaling constant $C$
\item[\texttt{Mcore}] molecular weight core (g/mol) $M_\text{core}$
\item[\texttt{Mbrush}] molecular weight brush (g/mol) $M_\text{brush}$
\item[\texttt{rho\_core}] mass density core matter (g/cm$^3$) $\rho_\text{core}$
\item[\texttt{rho\_brush}] mass density brush matter (g/cm$^3$) $\rho_\text{brush}$
\item[\texttt{b\_core}] scattering length density of core material (cm$^{-2}$) $b_\text{core}$
\item[\texttt{b\_brush}] scattering length density of brush material (cm$^{-2}$) $b_\text{brush}$
\item[\texttt{Nagg}] aggregation number $N_\text{agg}$
\item[\texttt{d1\_plus}] extra radius of shell1=core (compared to compact) $d_c^+$
\item[\texttt{part23}] relative distribution of shell amount in
                (1$^\text{st}$shell:2$^\text{nd}$shell) ($0\ldots\infty$) $p_{12}$
\item[\texttt{d2\_plus}] extra radius of first shell2 (compared to compact) $d_1^+$
\item[\texttt{d3\_plus}] extra radius of second shell3 (compared to compact) $d_2^+$
\item[\texttt{sigma1}] core smearing $\sigma_c$
\item[\texttt{sigma2}] smearing of 1$^\text{st}$ shell2 $\sigma_1$
\item[\texttt{sigma3}] smearing of 2$^\text{nd}$ shell3 $\sigma_2$
\item[\texttt{partstar}] relative distribution of parbolic:starlike profile in shell3 $x_\text{star}$;
        one usually puts a very high value in order to consider only a star-like profile.
\item[\texttt{gamma}] for star-like profile the exponent is $\gamma=4/3$ and
    for a constant profile $\gamma=0$
\item[\texttt{lparabol}] thickness of parabolic brush $L_p$ (must fit in shell3!)
\item[\texttt{f\_brush}] scattering length density correction factor brush
\item[\texttt{f\_core}] scattering length density correction factor core
\item[\texttt{rhosolv}] scattering length density of solvent $b_\textrm{solv}$
\end{description}


%%%%%%%%%%%%%%%%%%%%%%%%%%%%%%%%%%%%%%%%%%%%%%%%%%%%%%%%%%%%%%%%%%%%%%%%%%%%%%%%%%%%%%%%
\clearpage
\section{Ferrofluids} \hspace{1pt}
\label{sec:ferrofluid}

%%%%%%%%%%%%%%%%%%%%%%%%%%%%%%%%%%%%%%%%%%%%%%%%%%%%%%%%%%%%%%%%%%%%%%%%%%%%%%%%%%%%%%%%
\clearpage
\section{LogNorm\_fp} \hspace{1pt}
\label{sec:sd_lognorm_fp}

The \texttt{LogNorm} distribution is a continuous distribuion in which the logarithm of a variable
has a normal distribution.
\begin{subequations}
\begin{align}
\text{LogNorm}(x,N,\sigma,p,\mu) &=  \frac{N}{c_\text{LN}}
                                    \frac{1}{x^{p}}\,
                                    \exp\!\!\left(-\frac{\ln(x/\mu)^2}{2\sigma^2}\right) \\
c_\text{LN} &= \sqrt{2\pi}\,\sigma \,\mu^{1-p}
\,\exp\!\!\left((1-p)^2\frac{\sigma^2}{2}\right)
\label{eq:LogNorm}
\end{align}
\end{subequations}
where $\sigma$ is the width parameter, $p$ a shape parameter, $\mu$ is the location parameter.
$c_\text{LN}$ is choosen so that $\int_0^\infty\! \text{LogNorm}(x,\mu,\sigma,p)\,dR = N$
The mode of the distribution is defined as
\begin{align}
x_\text{mode} = \mu e^{-p \sigma^2}
\end{align}
and the $n^\text{th}$ moment $\langle X^n\rangle$ of the \texttt{LogNorm} distribution as
\begin{align}
\langle X^n\rangle = \frac{\int X^n\, \textrm{LogNorm}(X)\, dX}{\int \textrm{LogNorm}(X)\, dX} =
\mu^n \, e^{\frac{1}{2} \sigma^2 n (2 - 2 p + n)}.
\label{eq:nMoment:LogNorm}
\end{align}

Instead of using the parameter $N$ (particle number density) another
Log-Normal size distribution namely {\tt LogNorm\_fp} with the
volume fraction $f_p$ as a parameter has been implemented.
Using the volume fraction as a scaling parameter requires that the intensity is
given in units of cm$^{-1}$ and the scattering vector in nm$^{-1}$. Furthermore
the scattering contrast needs to be supplied in units of cm$^{-2}$. More details
about absolute intensity can be found in chapter \ref{ch:absint}.
The volume fraction $f_p$ can be obtained from the \texttt{LogNorm}-distribution
(eq.\ \ref{eq:LogNorm}) by integrating over the particle volume $V_P$. In case
of spheres we get
\begin{align}
f_p &= 10^{21} \int_0^\infty \mathrm{LogNorm}(R,N,\sigma,p,\mu) V_P(R) \, dR \label{eq:fpMomentsV} \\
    &= 10^{21} \int_0^\infty \mathrm{LogNorm}(R,N,\sigma,p,\mu) \frac{4}{3}\pi R^3 \, dR = 10^{21}
    N \frac{4}{3}\pi \langle X^3 \rangle .
\end{align}
The scaling factor $10^{21}$ depends on the actual units. More details are
given in section \ref{sec:volumefraction}.


For other shapes than spheres the corresponding volume of the object has to be used in eq.\ \ref{eq:fpMomentsV}.
In case of cylinders the volume is given by $V_\text{cyl}=\pi R^2 L$. Depending whether the radius $R$ or the
cylinder length $L$ has a size distribution the volume fraction $f_p$ is calculated differently namely
in case for a radius distribution by
\begin{align}
f_p &= 10^{21} \int_0^\infty \mathrm{LogNorm}(R) V_\text{cyl}(R,L) \, dR \label{eq:fpMomentsV} \\
    &= 10^{21} \int_0^\infty \mathrm{LogNorm}(R) \pi R^2L \, dR = 10^{21} N \pi L \langle X^2 \rangle
\end{align}
and in case of a length distribution by
\begin{align}
f_p % &= 10^{21} \int_0^\infty \mathrm{LogNorm}(L) V_\text{cyl}(R,L) \, dL  \\
    &= 10^{21} \int_0^\infty \mathrm{LogNorm}(L) \pi R^2L \, dL = 10^{21} N \pi R^2 \langle X \rangle .
\end{align}
As the cylinder volume depends on $R^2$  and $L$ either the second or the first moment of the distribution
function is involved in calculating the volume fraction depending which parameter has a distribution.
For a spherical shell a sum of different moments has to be used as listed in table \ref{tab:volumefraction}.

\begin{table}
  \centering
  \scriptsize
\rotatebox{270}{
  \setlength\doublerulesep{0pt}
\begin{tabular}{|>{\columncolor[gray]{1.0}[0.8\tabcolsep][0.8\tabcolsep]} c%
                |>{\columncolor[gray]{1.0}[0.8\tabcolsep][0.8\tabcolsep]} l%
                |>{\columncolor[gray]{1.0}[0.8\tabcolsep][0.8\tabcolsep]} c%
                |>{\columncolor[gray]{1.0}[0.8\tabcolsep][0.8\tabcolsep]} c%
                |>{\columncolor[gray]{1.0}[0.8\tabcolsep][0.8\tabcolsep]} c%
                |>{\columncolor[gray]{1.0}[0.8\tabcolsep][0.8\tabcolsep]} c%
                |>{\columncolor[gray]{1.0}[0.8\tabcolsep][0.8\tabcolsep]} c%
                |>{\columncolor[gray]{1.0}[0.8\tabcolsep][0.8\tabcolsep]} c|}
 \rowcolor[gray]{0.7}
 shape &  form factor &  distrib. & \texttt{length2} & \texttt{length3}
                             &  volume  & $V$ & $N(f_p)$\\
 \rowcolor[gray]{0.7}
  & &  param.\ & & & & &  \\
  \hline\hline
1      &  \texttt{Sphere} & $R$  & not used & not used & whole sph. &
            \mbox{\tiny$\frac{4}{3}\pi R^3$} &
            $  \frac{f_p}{10^{21}} \frac{3}{4\pi} \, \frac{1}{\langle X^3 \rangle}$ \\
 \rowcolor[gray]{0.95}
2 &  \texttt{Cylinder}    & $R$  & $L$ & not used & whole cyl. &
            \mbox{\tiny$ \pi R^2 L$} &
            $  \frac{f_p}{10^{21}} \frac{1}{\pi} \, \frac{1}{\langle X^2 \rangle L}$ \\
 \rowcolor[gray]{0.95}
3 &  \texttt{Cylinder}   & $L$   & $R$ & not used & whole cyl. &
            \mbox{\tiny$\pi R^2 L$} &
            $  \frac{f_p}{10^{21}} \frac{1}{\pi} \, \frac{1}{R^2 \langle X^1 \rangle}$ \\
4 &  \texttt{Sph.Sh.iii}    & $R$ & $\Delta R$ & not used & core+shell&
            \mbox{\tiny$ 4\pi \left(R^2 \Delta R +R \Delta R^2+\frac{1}{3}\Delta R^3+\frac{1}{3} R^3\right)$} &
            $  \frac{f_p}{10^{21}} \frac{1}{4\pi}\,
            \frac{1}{
            \frac{1}{3}\langle X^3 \rangle +
            \langle X^2 \rangle \Delta R+
            \langle X^1 \rangle \Delta R^2+
            \langle X^0 \rangle \frac{\Delta R^3}{3}}$\\
5 &  \texttt{Sph.Sh.iii}    & $\Delta R$ & $R$ & not used & core+shell &
            \mbox{\tiny$ 4\pi \left(R^2 \Delta R +R \Delta R^2+\frac{1}{3}\Delta R^3+\frac{1}{3} R^3\right)$} &
            $  \frac{f_p}{10^{21}} \frac{1}{4\pi}\,
            \frac{1}{
            \frac{1}{3}R^3\langle X^0\rangle +
            R^2 \langle X^1 \rangle+
            R \langle X^2 \rangle +
            \frac{1}{3}\langle X^3 \rangle}$ \\
 \rowcolor[gray]{0.95}
6 &  \texttt{Sph.Sh.iii}    & $R$ & $\Delta R$ & not used & core &
            \mbox{\tiny$ \frac{4}{3}\pi R^3$} &
            $  \frac{f_p}{10^{21}} \frac{3}{4\pi} \, \frac{1}{\langle X^3 \rangle}$\\
 \rowcolor[gray]{0.95}
7 &  \texttt{Sph.Sh.iii}    & $\Delta R$ & $R$ & not used & core &
            \mbox{\tiny$ \frac{4}{3}\pi R^3$} &
            $  \frac{f_p}{10^{21}} \frac{3}{4\pi} \, \frac{1}{R^3 \langle X^0 \rangle}$\\
8 &  \texttt{Sph.Sh.iii}    & $R$ & $\Delta R$ & not used & shell &
            \mbox{\tiny$ 4\pi \left(R^2 \Delta R +R \Delta R^2+\frac{1}{3}\Delta R^3\right)$} &
            $  \frac{f_p}{10^{21}} \frac{1}{4\pi}\,
            \frac{1}{\langle X^2 \rangle \Delta R+
            \langle X^1 \rangle \Delta R^2+
            \langle X^0 \rangle \frac{\Delta R^3}{3}}$ \\
9 &  \texttt{Sph.Sh.iii}    & $\Delta R$ & $R$ & not used & shell &
            \mbox{\tiny$ 4\pi \left(R^2 \Delta R +R \Delta R^2+\frac{1}{3}\Delta R^3\right)$} &
            $  \frac{f_p}{10^{21}} \frac{1}{4\pi}\,
            \frac{1}{R^2 \langle X^1 \rangle+
            R \langle X^2 \rangle +
            \frac{1}{3}\langle X^3 \rangle}$ \\
 \rowcolor[gray]{0.95}
10 &  \texttt{CylShell1}    & $R$ & $\Delta R$ & $L$ & core+shell &
            \mbox{\tiny$ \pi L\left( \Delta R^2 + 2R \Delta R +R^2\right)$} &
            $  \frac{f_p}{10^{21}} \frac{1}{\pi}\,
            \frac{1}{
            L\left( \Delta R^2 \langle X^0 \rangle + 2\langle X^1 \rangle \Delta R +\langle X^2 \rangle\right)
            }$ \\
 \rowcolor[gray]{0.95}
11 &  \texttt{CylShell1}    & $\Delta R$ & $R$ & $L$ &  core+shell  &
            \mbox{\tiny$ \pi L\left( \Delta R^2 + 2R \Delta R +R^2\right)$} &
            $  \frac{f_p}{10^{21}} \frac{1}{\pi}\,
            \frac{1}{
            L\left( \langle X^2 \rangle + 2R \langle X^1 \rangle +R^2\langle X^0 \rangle\right)
            }$\\
 \rowcolor[gray]{0.95}
12 &  \texttt{CylShell1}    & $L$ & $R$ & $\Delta R$ &  core+shell  &
            \mbox{\tiny$ \pi L\left( \Delta R^2 + 2R \Delta R +R^2\right)$} &
            $  \frac{f_p}{10^{21}} \frac{1}{\pi}\,
            \frac{1}{
            \langle X^1 \rangle \left( \Delta R^2 + 2R \Delta R +R^2\right)
            }$\\
13 &  \texttt{CylShell1}    & $R$ & $\Delta R$ & $L$ & core&
            \mbox{\tiny$ \pi L R^2$} &
            $  \frac{f_p}{10^{21}} \frac{1}{\pi} \, \frac{1}{\langle X^2 \rangle L}$ \\
14 &  \texttt{CylShell1}    & $\Delta R$ & $R$ & $L$ &  core  &
            \mbox{\tiny$ \pi L R^2$} &
            $  \frac{f_p}{10^{21}} \frac{1}{\pi} \, \frac{1}{R^2 L \langle X^0 \rangle}$\\
15 &  \texttt{CylShell1}    & $L$ & $R$ & $\Delta R$ &  core  &
            \mbox{\tiny$ \pi L R^2$} &
            $  \frac{f_p}{10^{21}} \frac{1}{\pi} \, \frac{1}{R^2 \langle X^1 \rangle}$\\
 \rowcolor[gray]{0.95}
16 &  \texttt{CylShell1}    & $R$ & $\Delta R$ & $L$ & shell  &
            \mbox{\tiny$ \pi L\left( \Delta R^2 + 2R \Delta R\right)$} &
            $  \frac{f_p}{10^{21}} \frac{1}{\pi} \,
            \frac{1}{L\left( \Delta R^2\langle X^0 \rangle + 2\langle X^1 \rangle \Delta R\right)}$  \\
 \rowcolor[gray]{0.95}
17 &  \texttt{CylShell1}    & $\Delta R$ & $R$ & $L$ &  shell  &
            \mbox{\tiny$ \pi L\left( \Delta R^2 + 2R \Delta R\right)$} &
            $  \frac{f_p}{10^{21}} \frac{1}{\pi} \,
            \frac{1}{L\left( \langle X^2 \rangle + 2R \langle X^1 \rangle\right)}$  \\
 \rowcolor[gray]{0.95}
18 &  \texttt{CylShell1}    & $L$ & $R$ & $\Delta R$ &  shell  &
            \mbox{\tiny$ \pi L\left( \Delta R^2 + 2R \Delta R\right)$} &
            $  \frac{f_p}{10^{21}} \frac{1}{\pi} \,
            \frac{1}{\langle X^1 \rangle\left( \Delta R^2 + 2R \Delta R\right)}$  \\
\hline
\end{tabular}
}

\vspace{3mm}

  \caption{The number density $N$ expressed in terms of volume fraction $f_p$ and moments
  $\langle X^n\rangle$ of the distribution function for
  some particle shapes and different parameters having a distribution.
 The factor $10^{21}$ is needed due to unit
conversion. It is assumed that the radius is given in nm, the
intensity in cm$^{-1}$ and the scattering length densities in
cm$^{-2}$.}
\label{tab:volumefraction}
\end{table}
