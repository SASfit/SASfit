\clearpage
\section{Magnetic Scattering}
\label{sect:MagScatt}

In the case of magnetic moments in the sample, the neutron undergoes
a magnetic interaction in addition to the nuclear interaction. The
corresponding interaction potential is given by
$$
V(\BM{r}) = -\BM{\mu}_N \cdot \BM{B}(\BM{r})
\quad \text{with} \quad \BM{\mu}_N=\gamma\frac{e \hbar}{2m_N
c}\BM{\sigma}
$$
where  $\BM{\mu}_N=\gamma\frac{e \hbar}{2m_N c}\BM{\sigma}$ is the
magnetic dipole moment of the neutron, \BM{\sigma} the Pauli spin
operator, $\gamma = -1.913$ the gyromagnetic ratio and
$\BM{B}(\BM{r})$ the magnetic field induced by an atom at the
position of the neutron. The latter has two components, one
induced by the magnetic dipole moment  $\BM{\mu}_S$ of the
electrons, denoted $\BM{B}_S(\BM{r})$, and one by their orbital
moment $\BM{\mu}_L$, denoted $\BM{B}_L(\BM{r})$. The (weak)
magnetic interaction $V(\BM{r}) =
-\BM{\mu}_N\cdot(\BM{B}_S(\BM{r})+\BM{B}_L(\BM{r}))$
 can as well be
treated in first Born approximation, resulting in the magnetic
scattering amplitude, in analogy to the nuclear scattering
amplitude, given by the Fourier transform of the magnetic
interaction potential $\mathcal {F}[V(\BM{r})]$: \BE b_M
=-\frac{m_N}{2\pi\hbar^2}\, \BM{\mu}_N \cdot \int d^3r \,
e^{\imath\BM{Qr}}\, (\BM{B}_S(\BM{r})+\BM{B}_L(\BM{r})). \EE An
additional static magnetic field $\BM{H}(\BM{r})$ at the point of
local magnetization $\BM{H}(\BM{r})$ (stemming from
$\BM{B}_S(\BM{r})+\BM{B}_L(\BM{r})$) induces a total local magnetic
induction of
$$\BM{B}(\BM{r})=\mu_0(\BM{H}(\BM{r})+\BM{M}(\BM{r}))$$
and the Fourier transform of   yields \BE \BM{B}(\BM{Q}) = \mu_0\,
\frac{\BM{Q}\times [\BM{M}(\BM{Q})\times\BM{Q}]}{Q^2}
             = \mu_0\, \BM{M}_{\bot}(\BM{Q})
             = \mu_0\, \BM{M}(\BM{Q}) \sin(\angle(\BM{Q,M}))
\EE where $\BM{M}(\BM{Q}) = \int d^3r \exp(\imath\BM{Q}\cdot\BM{r})
\BM{M}(\BM{r})$ , with  $\BM{M}(\BM{r})$ given in units of A/m.
$\BM{M}_{\bot}(\BM{Q}) = \BM{Q}\times [\BM{M}(\BM{Q})\times\BM{Q}] /
Q^2$ is the magnetization component perpendicular to the scattering
vector $Q$. The magnetic scattering length then is \BE b_M = D_M\,
\mu_0\, \mbox{\boldmath$\sigma$}\cdot \BM{M}_{\bot}(\BM{Q}) \mbox{~
with ~} D_M = \frac{m_N}{2\pi\hbar^2}\, \mu_N = 2.3161 \times
10^{14} \, \frac{1}{\mbox{Vs}}. \EE For the differential scattering
cross section one finally obtains \BE
\frac{d\sigma_M}{d\Omega}(\BM{Q}) = \frac{D_M^2}{N}\abs{\mu_0
\BM{M}_{\bot}(\BM{Q})}^2 \EE In the presence of a preferred
direction, for example induced by an external magnetic field, the
magnetic scattering depends on the spin state   of the neutrons. Let
the z-axis be the preferred direction, and let ($+$) and ($-$)
denote the neutron spin polarizations parallel and antiparallel to
the z-axis, then the scattering is described by four scattering
processes: two processes where the incident states ($+$) and ($-$)
remain unchanged ($++$ and $--$), the so-called 'non-spin-flip'
processes, and two processes where the spin is flipped ($+-$ and
$-+$), the 'spin-flip' processes. Keeping in mind that the nuclear
scattering does not flip the neutron spin, the four related
scattering lengths are

\begin{align}
b_{\pm\pm} &= b_N \mp D_M\, \mu_0\, M_{\bot x} \\
b_{\pm\mp} &= - D_M\, \mu_0\, (M_{\bot z} \pm \imath \, M_{\bot
y}).
\end{align}
Hereby $b_N$ is the nuclear scattering length. For an unpolarized
neutron beam (which may be taken as composed of 50\% ($+$) and
50\% ($-$) polarization) the square of the modulus of the
scattering length is
 \BE (b_{++}^2+b_{--}^2+b_{+-}^2+b_{-+}^2)/2 = b_N^2+D_M^2
\mu_0^2 M_{\bot}^2. \EE The differential cross section of the
unpolarized neutron beam can therefore be described by the sum of
the nuclear and the magnetic cross section, without any cross
terms.

%%%%%%%%%%%%%%%%%%%%%%%%%%%%%%%%%%%%%%%%%%%%%%%%%%%%%%%%%%%%%%%%%%%%%%%%%%%%%%%%%%

\clearpage
\subsection{Magnetic Saturation}

\subsubsection{MagneticShellAniso}
\label{sect:MagShellAniso}
~\\

\begin{align}
 K(Q,R,\Delta\eta) & = \frac{4}{3}\pi R^3 \Delta\eta \, 3 \frac{\sin QR - QR \cos
 QR}{(QR)^3} \\
   K_\text{sh}(Q&,R,\Delta R,\Delta\eta_\text{sh},\Delta\eta_\text{c}) = K(Q,R+\Delta R,\Delta\eta_\text{sh})-K(Q,R,\Delta\eta_\text{c})\\
   K_\text{NUC}(Q) &=
   K_\text{sh}(Q,R,\Delta R,\eta_{\text{sh NUC}}-\eta_{\text{m NUC}}, \eta_{\text{c NUC}}-\eta_{\text{m NUC}})\\
   K_\text{MAG}(Q) &=
   K_\text{sh}(Q,R,\Delta R,\eta_{\text{sh MAG}}-\eta_{\text{m MAG}}, \eta_{\text{c MAG}}-\eta_{\text{m MAG}})\\
   I(Q) &=
     \frac{1-p}{2}
     \left(K^2_\text{MAG}(Q)+2K_\text{NUC}(Q)K_\text{MAG}(Q)\right)
     \nonumber \\
   &+ \frac{1+p}{2} \left(K^2_\text{MAG}(Q)-2K_\text{NUC}(Q)K_\text{MAG}(Q)\right)\\
    &= K^2_\text{MAG}(Q) - 2p K_\text{NUC}(Q)K_\text{MAG}(Q)
\end{align}
\begin{align}
p            &: \text{neutron polarization, } p\in[-1:1] \nonumber \\
R            &: \text{radius of particle core} \nonumber \\
\Delta R     &: \text{thickness of particle shell} \nonumber \\
\eta_{\text{sh NUC}} &: \text{nuclear scattering length density of particle shell} \nonumber \\
\eta_{\text{m NUC}}  &: \text{nuclear scattering length density of matrix} \nonumber \\
\eta_{\text{c NUC}}  &: \text{nuclear scattering length density of particle core} \nonumber \\
\eta_{\text{sh MAG}} &: \text{magnetic scattering length density of particle shell} \nonumber \\
\eta_{\text{m MAG}}  &: \text{magnetic scattering length density of matrix} \nonumber \\
\eta_{\text{c MAG}}  &: \text{magnetic scattering length density
of particle core} \nonumber
\end{align}

%%%%%%%%%%%%%%%%%%%%%%%%%%%%%%%%%%%%%%%%%%%%%%%%%%%%%%%%%%%%%%%%%%%%%%%%%%%%%%%%%%%%%

\clearpage
\subsubsection{MagneticShellCrossTerm}
\label{sect:MagShellCrossTerm}
~\\

\begin{align}
 K(Q,R,\Delta\eta) & = \frac{4}{3}\pi R^3 \Delta\eta \, 3 \frac{\sin QR - QR \cos
 QR}{(QR)^3} \\
   K_\text{sh}(Q&,R,\Delta R,\Delta\eta_\text{sh},\Delta\eta_\text{c}) = K(Q,R+\Delta R,\Delta\eta_\text{sh})-K(Q,R,\Delta\eta_\text{c})\\
   K_\text{NUC}(Q) &=
   K_\text{sh}(Q,R,\Delta R,\eta_{\text{sh NUC}}-\eta_{\text{m NUC}}, \eta_{\text{c NUC}}-\eta_{\text{m NUC}})\\
   K_\text{MAG}(Q) &=
   K_\text{sh}(Q,R,\Delta R,\eta_{\text{sh MAG}}-\eta_{\text{m MAG}}, \eta_{\text{c MAG}}-\eta_{\text{m MAG}})\\
   I(Q) &=4p K_\text{NUC}(Q)K_\text{MAG}(Q)
\end{align}
\begin{align}
p            &: \text{neutron polarization, } p\in[-1:1] \nonumber \\
R            &: \text{radius of particle core} \nonumber \\
\Delta R     &: \text{thickness of particle shell} \nonumber \\
\eta_{\text{sh NUC}} &: \text{nuclear scattering length density of particle shell} \nonumber \\
\eta_{\text{m NUC}}  &: \text{nuclear scattering length density of matrix} \nonumber \\
\eta_{\text{c NUC}}  &: \text{nuclear scattering length density of particle core} \nonumber \\
\eta_{\text{sh MAG}} &: \text{magnetic scattering length density of particle shell} \nonumber \\
\eta_{\text{m MAG}}  &: \text{magnetic scattering length density of matrix} \nonumber \\
\eta_{\text{c MAG}}  &: \text{magnetic scattering length density
of particle core} \nonumber
\end{align}

%%%%%%%%%%%%%%%%%%%%%%%%%%%%%%%%%%%%%%%%%%%%%%%%%%%%%%%%%%%%%%%%%%%%%%%%%%%%

\clearpage
\subsubsection{MagneticShellPsi}
\label{sect:MagShellPsi}
~\\

\begin{align}
 K(Q,R,\Delta\eta) & = \frac{4}{3}\pi R^3 \Delta\eta \, 3 \frac{\sin QR - QR \cos
 QR}{(QR)^3} \\
   K_\text{sh}(Q&,R,\Delta R,\Delta\eta_\text{sh},\Delta\eta_\text{c}) = K(Q,R+\Delta R,\Delta\eta_\text{sh})-K(Q,R,\Delta\eta_\text{c})\\
   K_\text{NUC}(Q) &=
   K_\text{sh}(Q,R,\Delta R,\eta_{\text{sh NUC}}-\eta_{\text{m NUC}}, \eta_{\text{c NUC}}-\eta_{\text{m NUC}})\\
   K_\text{MAG}(Q) &=
   K_\text{sh}(Q,R,\Delta R,\eta_{\text{sh MAG}}-\eta_{\text{m MAG}}, \eta_{\text{c MAG}}-\eta_{\text{m MAG}})\\
   I(Q) &= K^2_\text{NUC}(Q)+\left(K^2_\text{MAG}(Q) - 2p
    K_\text{NUC}(Q)K_\text{MAG}(Q)\right) \sin^2\Psi
\end{align}
\begin{align}
p            &: \text{neutron polarization, } p\in[-1:1] \nonumber \\
R            &: \text{radius of particle core} \nonumber \\
\Psi         &: \text{angle between $\BM{Q}$ and $\BM{H}$} \nonumber \\
\Delta R     &: \text{thickness of particle shell} \nonumber \\
\eta_{\text{sh NUC}} &: \text{nuclear scattering length density of particle shell} \nonumber \\
\eta_{\text{m NUC}}  &: \text{nuclear scattering length density of matrix} \nonumber \\
\eta_{\text{c NUC}}  &: \text{nuclear scattering length density of particle core} \nonumber \\
\eta_{\text{sh MAG}} &: \text{magnetic scattering length density of particle shell} \nonumber \\
\eta_{\text{m MAG}}  &: \text{magnetic scattering length density of matrix} \nonumber \\
\eta_{\text{c MAG}}  &: \text{magnetic scattering length density
of particle core} \nonumber
\end{align}

%%%%%%%%%%%%%%%%%%%%%%%%%%%%%%%%%%%%%%%%%%%%%%%%%%%%%%%%%%%%%%%%%%%

\clearpage
\subsection{Superparamagnetic Particles (like ferrofluids)}~\\

\begin{align}
\frac{I_{\pm\pm}(\BM{Q})}{N} &= \abs{F_N(\BM{Q}) \mp \tilde
F_M(\BM{Q}) \left[ L(\alpha)-\gamma\right]\,\sin^2\epsilon}^2
\label{eq:ISQ_ppmm}\\
&+ \abs{\tilde F_M(\BM{Q})}^2\, \left( \frac{L(\alpha)}{\alpha}\,
\sin^2\epsilon - \mathcal{L}(\alpha)
\, \sin^4\epsilon \right) \nonumber \\
\nonumber \\
\frac{I_{\mp\pm}(\BM{Q})}{N} &= \left( \sin^2\epsilon -
\sin^4\epsilon \right) \left[L(\alpha)-\gamma\right]^2
\abs{\tilde F_M(\BM{Q})}^2 %S(\BM{Q})
\label{eq:ISQ_pmmp}\\
&+ \abs{\tilde F_M(\BM{Q})}^2  \left( \left(
\sin^4\epsilon-\sin^2\epsilon \right) \mathcal{L}(\alpha) +
(2-\sin^2\epsilon)\, \frac{L(\alpha)}{\alpha}\right) \nonumber \\
\nonumber \\
I_{\rm unp}(\BM{Q}) &=  \frac{1}{2}
\left(I_{++}(\BM{Q})+I_{+-}(\BM{Q})+I_{--}(\BM{Q})+I_{-+}(\BM{Q})
\right)
\nonumber \\
&= N\, \left(\abs{\tilde F_M(\BM{Q})}^2 \left[
L(\alpha)-\gamma\right]^2\,\sin^2\epsilon
   + \abs{F_N(\BM{Q})}^2 \right)
\label{eq:ISQ_unp} \\
&+ N \, \abs{\tilde F_M(\BM{Q})}^2\, \left( 2\,
\frac{L(\alpha)}{\alpha} - \mathcal{L}(\alpha)\,
\sin^2\epsilon\right) \nonumber \label{eq:ISQ}
\end{align}
Hereby $L(\alpha) = \coth \alpha - \frac{1}{\alpha}$ is the
classical Langevin function with \\
$\alpha = \mu_0(H+M_{\rm
eff})M_s^{cr} V_P/kT$. \\
Furthermore the following functions are defined as:\\
$\mathcal{L}(\alpha)=L^2(\alpha)-1+3\,\frac{L(\alpha)}{\alpha}$, \\
$F_N(\BM{Q})=\Delta\eta\, V_N\, f_N(\BM{Q})$,\\
$\tilde F_M(\BM{Q}) = D_M\, M_s^{cr}\, V_M\,f_M(\BM{Q})$ and \\
$\gamma = M_s^{am}/M_s^{cr}$. \\
$\epsilon$ describes the angle between $\BM{Q}$ and the applied
magnetic field $\BM{B}$. If the magnetic field lies in the plane of
the detector, i.e. perpendicular to the incoming beam direction,
$\epsilon$ is in practice identical to $\Psi$ so that $\cos \epsilon
= \sin\delta\, \cos\Psi \simeq \cos \Psi$ for $\delta \simeq \pi/2$
($\BM{Q}$ in plane of detector for SANS, only for large scattering
angle this will change).


\subsubsection{SuperparamagneticFFpsi}
\label{sect:superparFFpsi}

\subsubsection{SuperparamagneticFFAniso}
\label{sect:superparFFAniso}

\subsubsection{SuperparamagneticFFIso}
\label{sect:superparFFIso}

\subsubsection{SuperparamagneticFFCrossTerm}
\label{sect:superparFFCrossTerm}
