%%%%%%%%%%%%%%%%%%%%%%%%%%%%%%%%%%%%%%%%%%%%%%%%%%%%%%%%%%%%%%%%%%%%%%%%%%%%%%%%%%%%%%%%
\section{LogNorm} \hspace{1pt}
\label{sec:sd_lognorm}
The logbnormal distribution is a heavy-right-trailed distribution defined on a left-bounded interval.
\subsection{LogNorm\_fp} \hspace{1pt}
\label{sec:sd_lognorm_fp}

The \texttt{LogNorm} distribution is a continuous distribution in which the logarithm of a variable
has a normal distribution.
\begin{subequations}
\begin{align}
\text{LogNorm}(x,N,\sigma,p,\mu) &=  \frac{N}{c_\text{LN}}
                                    \frac{1}{x^{p}}\,
                                    \exp\!\!\left(-\frac{\ln(x/\mu)^2}{2\sigma^2}\right) \\
c_\text{LN} &= \sqrt{2\pi}\,\sigma \,\mu^{1-p}
\,\exp\!\!\left((1-p)^2\frac{\sigma^2}{2}\right)
\label{eq:LogNorm}
\end{align}
\end{subequations}
where $\sigma$ is the width parameter, $p$ a shape parameter, $\mu$ is the location parameter.
$c_\text{LN}$ is chosen so that $\int_0^\infty\! \text{LogNorm}(x,\mu,\sigma,p)\,dR = N$
The mode of the distribution is defined as
\begin{align}
x_\text{mode} = \mu e^{-p \sigma^2}
\end{align}
and the $n^\text{th}$ moment $\langle X^n\rangle$ of the \texttt{LogNorm} distribution as
\begin{align}
\langle X^n\rangle = \frac{\int X^n\, \textrm{LogNorm}(X)\, dX}{\int \textrm{LogNorm}(X)\, dX} =
\mu^n \, e^{\frac{1}{2} \sigma^2 n (2 - 2 p + n)}.
\label{eq:nMoment:LogNorm}
\end{align}

Instead of using the parameter $N$ (particle number density) another
Log-Normal size distribution namely {\tt LogNorm\_fp} with the
volume fraction $f_p$ as a parameter has been implemented.
Using the volume fraction as a scaling parameter requires that the intensity is
given in units of cm$^{-1}$ and the scattering vector in nm$^{-1}$. Furthermore
the scattering contrast needs to be supplied in units of cm$^{-2}$. More details
about absolute intensity can be found in chapter \ref{ch:absint}.
The volume fraction $f_p$ can be obtained from the \texttt{LogNorm}-distribution
(eq.\ \ref{eq:LogNorm}) by integrating over the particle volume $V_P$. In case
of spheres we get
\begin{align}
f_p &= 10^{21} \int_0^\infty \mathrm{LogNorm}(R,N,\sigma,p,\mu) V_P(R) \, dR \label{eq:fpMomentsV} \\
    &= 10^{21} \int_0^\infty \mathrm{LogNorm}(R,N,\sigma,p,\mu) \frac{4}{3}\pi R^3 \, dR = 10^{21}
    N \frac{4}{3}\pi \langle X^3 \rangle .
\end{align}
The scaling factor $10^{21}$ depends on the actual units. More details are
given in section \ref{sec:volumefraction}.


For other shapes than spheres the corresponding volume of the object has to be used in eq.\ \ref{eq:fpMomentsV}.
In case of cylinders the volume is given by $V_\text{cyl}=\pi R^2 L$. Depending whether the radius $R$ or the
cylinder length $L$ has a size distribution the volume fraction $f_p$ is calculated differently namely
in case for a radius distribution by
\begin{align}
f_p &= 10^{21} \int_0^\infty \mathrm{LogNorm}(R) V_\text{cyl}(R,L) \, dR \label{eq:fpMomentsA} \\
    &= 10^{21} \int_0^\infty \mathrm{LogNorm}(R) \pi R^2L \, dR = 10^{21} N \pi L \langle X^2 \rangle
\end{align}
and in case of a length distribution by
\begin{align}
f_p % &= 10^{21} \int_0^\infty \mathrm{LogNorm}(L) V_\text{cyl}(R,L) \, dL  \\
    &= 10^{21} \int_0^\infty \mathrm{LogNorm}(L) \pi R^2L \, dL = 10^{21} N \pi R^2 \langle X \rangle .
\end{align}
As the cylinder volume depends on $R^2$  and $L$ either the second or the first moment of the distribution
function is involved in calculating the volume fraction depending which parameter has a distribution.
For a spherical shell a sum of different moments has to be used as listed in table \ref{tab:volumefraction}.

\begin{table}
  \centering
  \scriptsize
\rotatebox{270}{
  \setlength\doublerulesep{0pt}
\begin{tabular}{|>{\columncolor[gray]{1.0}[0.8\tabcolsep][0.8\tabcolsep]} c%
                |>{\columncolor[gray]{1.0}[0.8\tabcolsep][0.8\tabcolsep]} l%
                |>{\columncolor[gray]{1.0}[0.8\tabcolsep][0.8\tabcolsep]} c%
                |>{\columncolor[gray]{1.0}[0.8\tabcolsep][0.8\tabcolsep]} c%
                |>{\columncolor[gray]{1.0}[0.8\tabcolsep][0.8\tabcolsep]} c%
                |>{\columncolor[gray]{1.0}[0.8\tabcolsep][0.8\tabcolsep]} c%
                |>{\columncolor[gray]{1.0}[0.8\tabcolsep][0.8\tabcolsep]} c%
                |>{\columncolor[gray]{1.0}[0.8\tabcolsep][0.8\tabcolsep]} c|}
 \rowcolor[gray]{0.7}
 shape &  form factor &  distrib. & \texttt{length2} & \texttt{length3}
                             &  volume  & $V$ & $N(f_p)$\\
 \rowcolor[gray]{0.7}
  & &  param.\ & & & & &  \\
  \hline\hline
1      &  \texttt{Sphere} & $R$  & not used & not used & whole sph. &
            \mbox{\tiny$\frac{4}{3}\pi R^3$} &
            $  \frac{f_p}{10^{21}} \frac{3}{4\pi} \, \frac{1}{\langle X^3 \rangle}$ \\
 \rowcolor[gray]{0.95}
2 &  \texttt{Cylinder}    & $R$  & $L$ & not used & whole cyl. &
            \mbox{\tiny$ \pi R^2 L$} &
            $  \frac{f_p}{10^{21}} \frac{1}{\pi} \, \frac{1}{\langle X^2 \rangle L}$ \\
 \rowcolor[gray]{0.95}
3 &  \texttt{Cylinder}   & $L$   & $R$ & not used & whole cyl. &
            \mbox{\tiny$\pi R^2 L$} &
            $  \frac{f_p}{10^{21}} \frac{1}{\pi} \, \frac{1}{R^2 \langle X^1 \rangle}$ \\
4 &  \texttt{Sph.Sh.iii}    & $R$ & $\Delta R$ & not used & core+shell&
            \mbox{\tiny$ 4\pi \left(R^2 \Delta R +R \Delta R^2+\frac{1}{3}\Delta R^3+\frac{1}{3} R^3\right)$} &
            $  \frac{f_p}{10^{21}} \frac{1}{4\pi}\,
            \frac{1}{
            \frac{1}{3}\langle X^3 \rangle +
            \langle X^2 \rangle \Delta R+
            \langle X^1 \rangle \Delta R^2+
            \langle X^0 \rangle \frac{\Delta R^3}{3}}$\\
5 &  \texttt{Sph.Sh.iii}    & $\Delta R$ & $R$ & not used & core+shell &
            \mbox{\tiny$ 4\pi \left(R^2 \Delta R +R \Delta R^2+\frac{1}{3}\Delta R^3+\frac{1}{3} R^3\right)$} &
            $  \frac{f_p}{10^{21}} \frac{1}{4\pi}\,
            \frac{1}{
            \frac{1}{3}R^3\langle X^0\rangle +
            R^2 \langle X^1 \rangle+
            R \langle X^2 \rangle +
            \frac{1}{3}\langle X^3 \rangle}$ \\
 \rowcolor[gray]{0.95}
6 &  \texttt{Sph.Sh.iii}    & $R$ & $\Delta R$ & not used & core &
            \mbox{\tiny$ \frac{4}{3}\pi R^3$} &
            $  \frac{f_p}{10^{21}} \frac{3}{4\pi} \, \frac{1}{\langle X^3 \rangle}$\\
 \rowcolor[gray]{0.95}
7 &  \texttt{Sph.Sh.iii}    & $\Delta R$ & $R$ & not used & core &
            \mbox{\tiny$ \frac{4}{3}\pi R^3$} &
            $  \frac{f_p}{10^{21}} \frac{3}{4\pi} \, \frac{1}{R^3 \langle X^0 \rangle}$\\
8 &  \texttt{Sph.Sh.iii}    & $R$ & $\Delta R$ & not used & shell &
            \mbox{\tiny$ 4\pi \left(R^2 \Delta R +R \Delta R^2+\frac{1}{3}\Delta R^3\right)$} &
            $  \frac{f_p}{10^{21}} \frac{1}{4\pi}\,
            \frac{1}{\langle X^2 \rangle \Delta R+
            \langle X^1 \rangle \Delta R^2+
            \langle X^0 \rangle \frac{\Delta R^3}{3}}$ \\
9 &  \texttt{Sph.Sh.iii}    & $\Delta R$ & $R$ & not used & shell &
            \mbox{\tiny$ 4\pi \left(R^2 \Delta R +R \Delta R^2+\frac{1}{3}\Delta R^3\right)$} &
            $  \frac{f_p}{10^{21}} \frac{1}{4\pi}\,
            \frac{1}{R^2 \langle X^1 \rangle+
            R \langle X^2 \rangle +
            \frac{1}{3}\langle X^3 \rangle}$ \\
 \rowcolor[gray]{0.95}
10 &  \texttt{CylShell1}    & $R$ & $\Delta R$ & $L$ & core+shell &
            \mbox{\tiny$ \pi L\left( \Delta R^2 + 2R \Delta R +R^2\right)$} &
            $  \frac{f_p}{10^{21}} \frac{1}{\pi}\,
            \frac{1}{
            L\left( \Delta R^2 \langle X^0 \rangle + 2\langle X^1 \rangle \Delta R +\langle X^2 \rangle\right)
            }$ \\
 \rowcolor[gray]{0.95}
11 &  \texttt{CylShell1}    & $\Delta R$ & $R$ & $L$ &  core+shell  &
            \mbox{\tiny$ \pi L\left( \Delta R^2 + 2R \Delta R +R^2\right)$} &
            $  \frac{f_p}{10^{21}} \frac{1}{\pi}\,
            \frac{1}{
            L\left( \langle X^2 \rangle + 2R \langle X^1 \rangle +R^2\langle X^0 \rangle\right)
            }$\\
 \rowcolor[gray]{0.95}
12 &  \texttt{CylShell1}    & $L$ & $R$ & $\Delta R$ &  core+shell  &
            \mbox{\tiny$ \pi L\left( \Delta R^2 + 2R \Delta R +R^2\right)$} &
            $  \frac{f_p}{10^{21}} \frac{1}{\pi}\,
            \frac{1}{
            \langle X^1 \rangle \left( \Delta R^2 + 2R \Delta R +R^2\right)
            }$\\
13 &  \texttt{CylShell1}    & $R$ & $\Delta R$ & $L$ & core&
            \mbox{\tiny$ \pi L R^2$} &
            $  \frac{f_p}{10^{21}} \frac{1}{\pi} \, \frac{1}{\langle X^2 \rangle L}$ \\
14 &  \texttt{CylShell1}    & $\Delta R$ & $R$ & $L$ &  core  &
            \mbox{\tiny$ \pi L R^2$} &
            $  \frac{f_p}{10^{21}} \frac{1}{\pi} \, \frac{1}{R^2 L \langle X^0 \rangle}$\\
15 &  \texttt{CylShell1}    & $L$ & $R$ & $\Delta R$ &  core  &
            \mbox{\tiny$ \pi L R^2$} &
            $  \frac{f_p}{10^{21}} \frac{1}{\pi} \, \frac{1}{R^2 \langle X^1 \rangle}$\\
 \rowcolor[gray]{0.95}
16 &  \texttt{CylShell1}    & $R$ & $\Delta R$ & $L$ & shell  &
            \mbox{\tiny$ \pi L\left( \Delta R^2 + 2R \Delta R\right)$} &
            $  \frac{f_p}{10^{21}} \frac{1}{\pi} \,
            \frac{1}{L\left( \Delta R^2\langle X^0 \rangle + 2\langle X^1 \rangle \Delta R\right)}$  \\
 \rowcolor[gray]{0.95}
17 &  \texttt{CylShell1}    & $\Delta R$ & $R$ & $L$ &  shell  &
            \mbox{\tiny$ \pi L\left( \Delta R^2 + 2R \Delta R\right)$} &
            $  \frac{f_p}{10^{21}} \frac{1}{\pi} \,
            \frac{1}{L\left( \langle X^2 \rangle + 2R \langle X^1 \rangle\right)}$  \\
 \rowcolor[gray]{0.95}
18 &  \texttt{CylShell1}    & $L$ & $R$ & $\Delta R$ &  shell  &
            \mbox{\tiny$ \pi L\left( \Delta R^2 + 2R \Delta R\right)$} &
            $  \frac{f_p}{10^{21}} \frac{1}{\pi} \,
            \frac{1}{\langle X^1 \rangle\left( \Delta R^2 + 2R \Delta R\right)}$  \\
\hline
\end{tabular}
}

\vspace{3mm}

  \caption{The number density $N$ expressed in terms of volume fraction $f_p$ and moments
  $\langle X^n\rangle$ of the distribution function for
  some particle shapes and different parameters having a distribution.
 The factor $10^{21}$ is needed due to unit
conversion. It is assumed that the radius is given in nm, the
intensity in cm$^{-1}$ and the scattering length densities in
cm$^{-2}$.}
\label{tab:volumefraction}
\end{table}

\clearpage
\subsection{shifted bi-LogNorm}\hspace{1pt}
\label{sec:sd_bilognorm}
\begin{align}
c_i &= \frac{1}{\sqrt{2\pi}\sigma_i \mu_i^{1-p_i}}\exp\!\left(-\tfrac{1}{2}\sigma_i^2(1-p_i)^{2}\right)\\
\mathrm{biLogNorm}(x) &=
\begin{cases}
0, & x \le x_0,\\ 
\sum_{i=0}^1 \frac{N_i\,c_i}{(x-x_0)^{-p_i}}\exp\!\left(-\frac{\ln^2\left(\dfrac{x-x_0}{\mu_i}\right)}{2 \sigma_i^2}\right)
, & x > x_0.
\end{cases}
\end{align}

\subsection{shifted LogNorm}\hspace{1pt}
\label{sec:sd_shiftedlognorm}
\begin{align}
c &= \frac{1}{\sqrt{2\pi} \sigma} \\
\mathrm{LN}_\mathrm{sh}(x) &=
\begin{cases}
0, & x \le x_0,\\ 
\frac{N\,c}{x - x_0}\exp\left(-\frac{\ln^2\left(\frac{x - x_0}{\mu}\right)}{2 \sigma^2}\right)
, & x > x_0.
\end{cases}
\end{align}

%%%%%%%%%%%%%%%%%%%%%%%%%%%%%%%%%%%%%%%%%%%%%%%%%%%%%%%%%%%%%%%%%%%%%%%%%%%%%%%%%%%%%%%%
\section{metalog distribution} \hspace{1pt}
\label{sec:sd_metalog}

The quantile function of the metalog distribution $Q(y)=M_{k}(y)$ is defined as \cite{Keelin2011,Keelin2016,Keelin2019}
\begin{flalign}
%\scriptsize
 M_{k}(y) &=
 \left\{{\begin{array}{ll}
    a_{1}+a_{2}\ln {y \over {1-y}}, &  k=2\\
    a_{1}+a_{2}\ln {y \over {1-y}} +a_{3}(y-\frac12)\ln {y \over {1-y}}, & k=3 \\
    a_{1}+a_{2}\ln {y \over {1-y}} +a_{3}(y-\frac12)\ln {y \over {1-y}}+a_{4}(y-\frac12), & k=4 \\
    M_{k-1}(y)+a_{k}(y-\frac12)^{k-1 \over 2}, &{\mbox{odd }}k\geq 5 \\
    M_{k-1}(y)+a_{k}(y-\frac12)^{{k \over 2}-1}\ln {y \over {1-y}},  &{\mbox{even }}k\geq 6 \\
 \end{array}}\right.
\end{flalign}
and its PDF $p(x)=m_k(y)$ with $x=M_{k}(y)$ reads as
\begin{flalign}
%\scriptsize
 m_{k}(y)&=
\left\{{\begin{array}{ll}
    {y(1-y) \over {a_{2}}}, & k=2 \\
    \left({a_{2} \over {y(1-y)}}+a_{3}\left({y-\frac12 \over {y(1-y)}}+\ln {y \over {1-y}}\right)\right)^{-1}, & k=3\\
    \left({a_{2} \over {y(1-y)}}+a_{3}\left({y-\frac12 \over {y(1-y)}}+\ln {y \over {1-y}}\right)+a_{4}\right)^{-1}, &k=4\\
    \left({1 \over {m_{k-1}(y)}}+a_{k}{k-1 \over 2}(y-\frac12)^{(k-3)/2}\right)^{-1}, & \mbox{odd } k\geq 5\\
    \left({1 \over {m_{k-1}(y)}}+a_{k}\left({(y-\frac12)^{\frac{k}{2}-1} \over {y(1-y)}}+({k \over 2}-1)(y-\frac12)^{(\frac{k}{2}-2)}\ln {y \over {1-y}}\right)\right)^{-1}, &\mbox{even } k\geq 6\\
    \end{array}}\right.
\end{flalign}
The metalog PDF as defined above is an unbounded distribution. In case of describing a size distribution it needs to be bounded (Logit metalog) or at least semi-bounded (bounded below, Log metalog). This can be easily done by a transformation of $z(x) = \ln \left(x-b_l\right)$ for a distribution with a lower bound $b_l$ or $z(x)=\ln\left(\frac{x-b_l}{b_u-x}\right)$ with a lower bound $b_l$ and an upper bound $b_u$, where $z$ is then metalog-distributed. The corresponding quantile functions and PDF then reads as
\begin{eqnarray}
   M_k^{\log}(y) &=&  \left\{{\begin{array}{ll}
    b_l+e^{M_k(y)}, & 0<y<1 \\
    b_l, & y=0\\
    \end{array}}\right.\\
   m_k^{\log}(y) &=&  \left\{{\begin{array}{ll}
    m_k(y)e^{-M_k(y)}, & 0<y<1 \\
    0, & y=0\\
    \end{array}}\right.\\
   M_k^{\mathrm{logit}}(y) &=&  \left\{{\begin{array}{ll}
    \frac{b_l+b_ue^{M_k(y)}}{1+e^{M_k(y)}}, & 0<y<1 \\
    b_l, & y=0\\
    b_u, & y=1\\
    \end{array}}\right.\\
   m_k^{\mathrm{logit}}(y) &=&  \left\{{\begin{array}{ll}
    m_k(y)\frac{\left(1+e^{M_k(y)}\right)^2}{(b_u-b_l)e^{M_k(y)}}, & 0<y<1 \\
    0, & y=0\\
    0, & y=1\\
    \end{array}}\right.
 \end{eqnarray}

\subsection{unbounded metalog distribution}\hspace{1pt}
\label{sec:sd_unboundedmetalog}

\subsection{semi-bounded metalog distribution}\hspace{1pt}
\label{sec:sd_semiboundedmetalog}

\subsection{bounded metalog distribution}\hspace{1pt}
\label{sec:sd_boundedmetalog}

%%%%%%%%%%%%%%%%%%%%%%%%%%%%%%%%%%%%%%%%%%%%%%%%%%%%%%%%%%%%%%%%%%%%%%%%%%%%%%%%%%%%%%%%
\section{Johnson's distribution} \hspace{1pt}
\label{sec:sd_Johnson}

\subsection{Johnson $S_B$}\hspace{1pt}
\label{sec:sd_johnsonSB}

\subsection{Johnson $S_L$}\hspace{1pt}
\label{sec:sd_johnsonSL}

\subsection{Johnson $S_U$}\hspace{1pt}
\label{sec:sd_johnsonSU}

\subsection{Johnson $S_N$}\hspace{1pt}
\label{sec:sd_johnsonSN}
%%%%%%%%%%%%%%%%%%%%%%%%%%%%%%%%%%%%%%%%%%%%%%%%%%%%%%%%%%%%%%%%%%%%%%%%%%%%%%%%%%%%%%%%
\section{generalized Johnson's distribution} \hspace{1pt}
\label{sec:sd_gneralizedJohnson}
